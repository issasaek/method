% !TeX root = method.tex

\begin{textblock*}{0.1\linewidth}(510pt, 60pt)
    \small \version
\end{textblock*}

\title{Loredux-Method}
\author{佐伯 一冴\\\small\url{https://github.com/issasaek}}
\date{}
\maketitle

\begin{abstract}
    信念体系を最適化して,抑制条件が惹起する不快な感情を無害化するメソッドを構築する.メソッドの基本的な適用パターンは,不快な感情が生じたときに,①関連する信念体系の部分を特定化して,②信念体系を部分的に最適化して問題の部分を取り除く,というステップから成り立つ.信念体系の最適化は,それを論理的表記法で統制された言語に規格化する論理的還元の段階と,集合論への存在論的還元を通じて現実を再構築する段階に分けることができる.メソッドの継続的な適用は使用者を日常的意識から解放するだろう.
\end{abstract}

\section{序論}
\label{sec:序論}

本稿では,個人の信念体系を最適化するメソッドが扱われる.メソッドの目的はその使用者の幸福である.ただし,ここで言う幸福は特定の心理状態ではなく,メソッドの継続的な適用の結果到達できると仮定される全体的状態のことだ.このメソッドの構築と適用の両面において,foundation~\cite{foundation}で開発された理論的概念を使用する.そのため本稿はfoundation~\cite{foundation}の内容と記法を前提としている.参照は\kagi{\fndp{1}{1}}のようにページリンク付で行われる.

類似した機能を持つメソッドは,心理療法や自助テクニックから宗教的な教義に至るまで,既に古来から夥しい数が存在する.本稿が別のメソッドをわざわざ考案する理由は,単に,既存のメソッドの中に,
\begin{enumerate}[label=(\arabic*)]
    \item 環境依存性が最低限度に止まる.
    \item 日常的意識の根本的な改変を含む.
    \item 科学的世界観と衝突しない.
\end{enumerate}
という条件を全て充たすものを見つけられなかったからだ.(1)について,メソッドの機能を阻害するような使用者を取り巻く(変更可能性の低い)環境的条件の種類が多いほど,そのメソッドは環境依存的である.この環境依存性を低く保つハードルは高い.それでも本稿では,特定の人間関係や資源の状態を前提としないだけでなく,(牢獄に囚われたり,不治の病に伏せているといった)もっと制限された状況においても可及的に使用できるメソッドを志向している.
また(2)は,神経症的状態を解消してより適応的になるという通常の心理療法が持つ機能を含むとしても,さらにその先へと運んでくれるメソッドが欲しいということだ.信念体系の最適化には,日常的な言語的装置をそれとは異なるビジョンを喚起するように改造する工程が含まれている.宗教的なメソッドも結局はそのような志向を持っているだろう.しかし,それらの多くは(3)を充たさないことで除外される.我々の信念体系には,科学者共同体に受容された科学理論か,または常識化されたその劣化バージョンが部分的に含まれている.それらと矛盾するような信念を新たに組み入れることによって,体系を大きく変更したり,矛盾による混乱を放置するとしたら,それは端的にコストが高すぎるのだ.それを実行するには既にテクノロジーが浸透しすぎており,それゆえに宗教的メソッドが機能を果たせる時期は過ぎ去ってしまった.

以上が本稿のメソッドの基本的なコンセプトだ.メソッドの継続的な適用は「幸福」の語が無意味になるほど日常的意識を改変するかもしれない.その段階ではその語でメソッドの目的を表現することはいよいよ困難になるだろう.日常的意識は多数の言語的傾向性から成り立っている.それらはおそらく,石器時代を通じて当時の環境に対処するよう進化論的に獲得されてきた傾向性に根を持っており,我々は未だにそれに隷属しているのだ.メソッドは新しい言語的装置により信念体系を更新する.そして古い神々を殺すだろう.

\section{基本構造}
\label{sec:基本構造}

メソッドは主に不快な感情をトリガーとして使用する.そしてその感情の生起に関与している信念体系の部分を修正する.しかし,このことが何を意味するのかは説明を要する.それと同時に,メソッドの細部を構築する前にその全体像をここで提示しておこう.

\subsection{抑制条件}
\label{ssec:抑制条件}

ここで「不快な感情」と言われているのは,負の修正条件すなわち抑制条件(の危険)の実現によって引き起こされるような内的状態(情動)のことだ.抑制条件とは,ある条件(適用条件)下での行動によってそれが実現された場合に,それによって,同一行動主体のその条件下の同一行動が抑制されるような条件である.「抑制される」というのは,適用条件によってその行動を遮断するような制御回路(負の制御構造)が,その行動主体に構築される(蓋然性がある),ということを意味している\footnote{修正条件の概念は,学習理論における弁別学習に関する概念を改造したものである.例えば,レイノルズ~\cite[pp.\,8--12]{レイノルズ}を参照.なお,蘇った記憶のような内的な状態も修正条件となり得る.}.
例えば,ある人が受注を受けたプロダクトを期日までに納品できなかった結果,顧客か上司に叱責されるという場面を考えよう.この場合,統制に関与する適用条件としては,納品することによって顧客の重要な利益が侵害される等の特殊事情が存在しない,といった条件が考えられる.そのような条件下において,作為/不作為により期日において顧客がプロダクトを占有していない状態を惹起することが,顧客/上司の叱責を因果的に決定している.
このような因果連鎖の実現によって,以下の因果関係(制御構造)の蓋然性が惹起される,という因果関係が成立しているならば,顧客か上司の叱責は,当該適用条件と行動に対して抑制条件になっている.すなわち,別の顧客に関するものでよいが,
\[
   \text{上記の特殊事情が存在しない → 顧客のプロダクト占有を阻止する因果系列を遮断}
\]
という制御構造.
なお,抑制条件とは逆に,行動を促進する(適用条件によって行動が起動されるという正の制御構造を惹起する)条件は促進条件(促進条件)であるが,その実現が阻止されることも,ある状況と行動に対して抑制条件になり得る.

foundation~\cite{foundation}の記法によると,修正条件が特定の領域$x$と規制類型$y$,それゆえ適用条件$\app{\epsilon}y$や構成要件$\exe{\epsilon}y$に相対的であることが明確になる.すなわち,$ x\in\cty{\epsilon}y $について,$y\in\mathrm{Reg}$の制御方向が,
\[
    \tildel{\trgl{y}}\fap 1 = \barl{(\trgl{y}\fap 0)}\cap\msec{(\arg y)}{0}
\]
ならば($y$はS類型),修正条件$ \orp{\tilde{x}\fap 0,\tildel{\trgl{y}}\fap 0} $は抑制条件である(\fndp{23}{23}).
逆に,$ y $の制御方向が$\tildel{\trgl{y}}\fap 1 = \trgl{y}\fap 0 $ならば($y$はP類型),促進条件である.
いずれにせよ,$ \tilde{x}\fap 0\in\tildel{\trgl{y}}\fap 0 $であるとき修正条件は実現される.そして,行動主体は単に$\tilde{x}\fap 1$で特定される.

ところで,修正条件を因果的に惹起する条件も同じ状況と行動に対して修正条件になり得る\footnote{レイノルズ~\cite[p.\,12]{レイノルズ}.この因果連鎖は最終的に,他の修正条件によって因果的に決定されていない生得的な修正条件に行き着く.}.例えば,上司または顧客による状況把握が別の報告者の行動に依存しているならば,その報告者の報告や認知的状態もまた,同様の修正条件となり得る.
厳密化して言うと,$ a\in b $によって$ \tilde{x}\fap 0\in\tildel{\trgl{y}}\fap 0 $の蓋然性が因果的に惹起された場合,つまり,
\begin{gather}
    a\in b,\\
    (\exists n)[n\subseteq\indx{\epsilon}\con{1}
        \orp{a,b}\to_{\epsilon}\orp{\tilde{x}\fap0,\prob{(\tildel{\trgl{y}}\fap 0)}{(n\uphl\epsilon)}}
    ],
\end{gather}
である場合,$ \tilde{w}\fap 1 = \tilde{x}\fap 1 $である$w$と$z$について,$w\in\cty{\epsilon}z$になり得る.ただし,$y$と$z$は,それぞれの右域から修正領域に関する条件を除外した条件が同一であり,かつ,$ \trgl{z} $は$ \trgl{y} $の修正条件を$b$に交換した系列である.すなわち,
\[
    \mathrm{pur}\,y = \mathrm{pur}\,z\con{1}\tildel{(\tildel{\trgl{y}}\tbinom{0}{b})} = \trgl{z}.
\]
このようにして,$ b = \tildel{\trgl{z}}\fap 0 $について,$ \orp{\tilde{w}\fap 0,b} $が新たに修正条件となる.
(2)の\kagi{$ \indx{\epsilon} $}は有意な蓋然性レベルを表わしている.$ \beta\subseteq\trgl{\arg\epsilon} $である限り,$ \prob{\beta}{(1\uphl\epsilon)}=\beta $である.蓋然性の定式化は\fndp{17}{17}にある.

この修正条件の形成過程は,以下のN類型$v$の執行可能性によって$ w\in \cty{\epsilon}z $が構築される過程として再記述できる.つまり,
\[
    \arg\trgl{v} = 3\con{1}\tildel{\trgl{v}}\fap 1 = \Lambda\con{1}\trgl{v}\fap 0 = b\con{1}
    \tildel{\trgl{v}}\fap 0 = \tildel{\trgl{y}}\fap 0
\]
である規制類型$v$と,$ \tilde{u}\fap 0 = \tilde{x}\fap 0 $である$ u\in\app{\epsilon}v $について,
\[
   \orp{u,\exe{\epsilon}v\cap\enf{\epsilon}v}\to_{\epsilon}\orp{w,\cty{\epsilon}z}.
\]
この点,実際には$ u\notin \exe{\epsilon}v\cap\enf{\epsilon}v $であっても,それの言語的代替(実質的には(2)の言語的代替),すなわち$ u\in \app{\epsilon}v\cap\enf{\epsilon}v $であることを肯定する主体$ \tilde{x}\fap 1 $の言語的傾向性が存在するなら,同一の制御可能性が形成され得る.上記の例で言えば,行動主体が,上司または顧客による状況把握が報告者の行動に依存していると単に思い込んでいるに過ぎない場合であっても,報告者の行動は同様の抑制条件となり得る.

さて,本稿が「不快な感情」と言う語で表示しようとするのは,以上の意味での抑制条件か,またはその危険(蓋然性)の実現によって生じる行動主体の情動である.例によって,抑制条件またはその危険が実際にはなくても,それの言語的代替つまりそれを肯定する行動主体の言語的傾向性があれば,同様の情動が生じる.再び先述の例では,上司か顧客の叱責によって生じる情動的状態は,「恐怖」と呼ばれる状態か「怒り」と呼ばれる状態かもしれない.あるいは「落ち込み」や「憂鬱」と呼ばれる状態でもあり得る.そして,そのような叱責より時間的に前の段階において,叱責がそれに依存する現在の状況自体(単に行動主体が現在の時間的・空間的位置を占めることをも含む)が抑制条件として,同様の情動か「不安」と呼ばれるような別のタイプの情動も生じるだろう.どのタイプの情動が生じるかは,この例では行動主体の認知的状態と信念体系に依存していると思われる.生じる情動のタイプを厳密に区別することは必要ではないが,それに関与している信念体系の部分が異なることを通じて,メソッドの適用において違いが生じる可能性はある.

\subsection{信念体系}
\label{ssec:信念体系}

人$x$の時間$t$における信念体系は,$x$が$t$において肯定する文の集合である.文を肯定する制御構造について,\fndp{37}{37}では次のような規制類型$\eta$を想定した.
$\app{\epsilon}\eta$は文タイプの真偽が質問される状況を特定化し,$ \exe{\epsilon}\eta $はそこにおいてその文を肯定する因果系列を実現する.そして,$\eta$の修正条件はコミュニケーションの成立を示すような質問者の反応を特定化する.
型文字\kagi{$ \tau $}で,$e$が言語的タイプ$p$のトークンであるような$ \orp{e,p} $のクラス(を指示するクラス抽象体の位置)を表わすなら\footnote{
    $ \tau\img\univ\subseteq \mathcal{P}(\timex{\mathbb{R}}{4}) $,そして,任意の$ p\in \breve{\tau} $はある同一の言語に属する.すなわち,$p$は当該言語の原始記号の有限系列であり,原始記号自体はそのトークンの集合である.\fndp{36}{36}を参照.},
\begin{gather*}
    \arg \eta \subseteq\classab{e:(\exists k)(\exists p)(\tilde{e}\fap 2 = \orp{k,p}\con{1}k\subseteq \timex{\mathbb{R}}{4}\con{1}p\in\breve{\tau}\img\univ)},\\
    \tildel{\trgl{\eta}}\fap 2 = \classab{\orp{k,p}:p\in\breve{\tau}\img\univ\con{1}(\exists q)(\orp{k,q}\in\tau\con{1}\text{ $q$は$p$の肯定 })}.
\end{gather*}
$ e\in\arg \eta\con{1}\tilde{e}\fap 2 = \orp{k,p} $とすると,$ \tilde{e}\fap 2\in\tildel{\trgl{\eta}}\fap 2 $であるとき,$r$の肯定であるような言語的タイプのトークン$k$が生成される.例えば,質問「$\phi$ですか」に対する「$\phi$です」.ただし,\kagi{$ \phi $}に$p$を代入する.

すると,文$p$が時区間$t$において$x$の信念体系に属するのは,$p,x$を内部構造に含む$t$内の任意の準拠領域$e$について,$e\in\cs{\epsilon}\eta$であるとき,かつそのときに限られる\footnote{多少考えれば肯定するような文も信念体系に取り込むために,制御構造の(高度の)蓋然性$ \prob{(\cs{\epsilon}\eta)}{n\uphl\epsilon} $に拡張してもよい.ただし,$n\subseteq\indx{\epsilon}$である.}.実際に質問された状況には限られないから$ e\in\app{\epsilon}\eta $である必要はない.しかし,$\app{\epsilon}\eta$から質問に関する条件$ f $を除外した条件を充たさない$e$については,$\cs{\epsilon}\eta$が成立しない可能性がある.この点を踏まえると,
\[
    \tilde{e}\fap 1 = x\con{1}\mathcal{R}\fap(\tilde{e}\fap 2)= p\con{1}
    e\fap 0,\mathcal{L}\fap(\tilde{e}\fap 2)\subseteq t\con{1}e\in\intersect{\classab{z:\arg\eta\subseteq z\neq f}}
\]
である任意の$e$について$ e\in\cs{\epsilon}\eta $であるとき,かつそのときに限り,$p$は$t$において$x$の信念体系に属する.
なお,本心で肯定しているかどうか等の厄介事が当然生じるが,その手の変速事例は,$\eta$の適用条件において,どうにかして処理されていると想像しよう.
関連して,最適化前の信念体系は矛盾している可能性が否定できないが,$p$とその否定を文字通り両方肯定するような言語的傾向性は通常成立しない.

差し当たって,信念体系に属する文$p$は,その真理値が$p$のトークンが生成される状況に左右されない永久文\footnote{クワイン~\cite[p.\,325]{クワインd}.}である必要はない.
この点,「今」「ここ」「私」のような指標的表現は,それが出現する文の使用状況に応じて指示対象が変化する.したがって,指標的表現を含む文は,その真理値がトークン生成状況に相対的であるから永久文ではない.また,1個の名詞や形容詞,例えば「ウサギ」や「寒い」が文として使用される場合,その文は「それはウサギだ」,「今ここは寒い」等の代用とみなすことができる\footnote{クワイン~\cite[p.\,5]{クワインe}.}.
そして,非永久文は,その文のトークンを生成する領域と近接する領域について真または偽であるなら場面文である.他方,トークン生成領域を含むより時間的に長い領域について真または偽であるなら持続文である.例えば持続文「今日の朝刊が届いた」が真であれば,その真理値はその日の終わりまで持続する.

日常言語にはこのような非永久文が頻繁に出現するが,それの真理条件や含意関係を明確化する場合は,指標的表現の指示対象を補って永久化する必要が生じる.それに加えて,真理関数と量化の構成方法が明示化された\fndp{4}{4}の原初的言語やその拡張(\fndp{18}{18})の文へと規格化しなければならない.本稿のメソッドにもこの規格化の工程が含まれている.そこで,文の永久化の手続を一般的な形で述べよう.すなわち,
非永久文$p$について,それに出現する指標的表現を変項の系列$ v\fap 0,v\fap 1,\dots,v\fap n $に置き換えた結果を$q$とする.また,$p$のあるトークン生成において,変項の値(指標的表現の指示対象)がそれぞれ$ k\fap 0,k\fap 1,\dots,k\fap n $であるとする.すると,系列$k$に相対的な$p$の永久化は,形式
\begin{align*}
    (\exists \nu_0)(\exists \nu_1)\dots(\exists \nu_n)(\phi\con{1}\orp{\nu_0,\nu_1,\dots,\nu_n}=\alpha)
\end{align*}
への代入によって得ることができる.すなわち,\kagi{$ (\exists \nu_0)(\exists \nu_1)\dots(\exists \nu_n) $}には,変項$ v\fap 0,v\fap 1,\dots,v\fap n $の存在量化子を順番に代入し,\kagi{$ \nu_0,\nu_1,\dots,\nu_n $}には,$ v\fap 0,v\fap 1,\dots,v\fap n $を\kagi{$ , $}で区切って順次代入する.そして,\kagi{$ \phi $}に$q$を,\kagi{$ \alpha $}に$ k\fap 0,k\fap 1,\dots,k\fap n $からなる順序対を指示するクラス抽象体を代入する.

ところで,最適化前の信念体系に属する文$p$は,価値的・規範的判断のような非記述的な文であり得る.環境を予測し制御するテクノロジーを構成するかその基盤となる言語的装置に寄与しない文は,実在の構造を記述するものではない.一方,信念体系は文の真偽に関する質問刺激に対してそれを肯定する傾向性に基づいている.すると,非記述的な文が信念体系に含まれるのかという問題が生じるように見える.しかし,文が記述的かどうかは,以下のような真理述語の適用可能性とは関係がない.

すなわち,ある言語(の文集合)$\gamma$と,任意の$q\in\gamma$を指示する表現と$q$の翻訳を構成できる言語$\gamma'$を考える.
そして,型文字\kagi{$ T $}で,$\gamma$のメンバーに適用される原始的または複合的な$\gamma'$の述語$t$の位置を表わすものとしよう.すると,任意の閉鎖文$p\in \gamma$について,型式
\[
   (\exists y)(\alpha = y\con{1}Ty)\case{3}{0}{0}P
\]
の\kagi{$ \alpha $}に$p$を指示する表現を,\kagi{$ P $}に$p$の翻訳である$\gamma'$の文を代入した結果が($\gamma'$において)真であるなら,$t$は$\gamma$の真理述語である.
$\gamma$を内包的な文を含む日常言語の文の集合とするなら,価値的・規範的判断にもこのような真理述語を適用できる(真または偽である).したがって,それらも最適化前の信念体系に含まれ得る.

さらに,$ \gamma\subseteq\gamma' $であり,\kagi{$ P $}に$p$自身を代入した結果がすべて真であるなら,$t$は引用解除的な真理述語である.この$t$を「真である」とし,また引用符が使えると仮定する.するとこの場合,任意の閉鎖文$p\in\gamma$について,型式
\[
   \text{\kagi{$ P $}は真である}\case{3}{0}{0}P
\]
の\kagi{$ P $}に$p$を代入した結果は真である.\fndpp{18}{18--19}の言語$\mathfrak{L}$の標準的モデルに相対的な真理は,対象領域の限定により引用解除的ではない\footnote{その領域を指示するクラス抽象体に相対化(クワイン~\cite[p.\,217]{クワインa})されていない文は引用解除できない.}.

\subsection{最適化}
\label{ssec:最適化}

\subsubsection{信念体系の関与}
\label{sssec:信念体系の関与}

以上で「不快な感情」と「信念体系」の本稿での用法が規定された.次に,メソッドは不快な感情の生起に信念体系の部分が関与しているとの仮定に基づいている\footnote{
    認知行動療法に分類されるような心理療法(論理療法~\cite{論理療法}や認知療法~\cite{認知療法}等)が準拠する仮説は,このような仮定を実質的に含むと考えられる.それらの心理療法には,多かれ少なかれ信念体系を修正するプロセスが取り込まれている.
}.そのような関与のパターンについては,以下のように考えられる.

パターン1:\ref{ssec:抑制条件}で述べたように,$ x\in\cty{\epsilon}y $について,既存の修正条件$ \tilde{x}\fap 0\in\tildel{\trgl{y}}\fap 0 $を因果的に惹起する条件もまた修正条件となり得る.この形成過程はN類型$ v $について,
\[
   \orp{u,\exe{\epsilon}v\cap\enf{\epsilon}v}\to_{\epsilon}\orp{w,\cty{\epsilon}z},
\]
であることとして再記述された.そして,実際には$ u\notin \exe{\epsilon}v\cap\enf{\epsilon}v $であっても,それの言語的代替,すなわち$ u\in \app{\epsilon}v\cap\enf{\epsilon}v $であることを肯定する主体$ \tilde{x}\fap 1 $の言語的傾向性が存在するなら,同一の制御可能性が形成され得ることも指摘された.
この言語的代替の条件を言い換えると,$ u\in \app{\epsilon}v\cap\enf{\epsilon}v $であることを記述する文$p$が,その時点における$ \tilde{x}\fap 1 $の信念体系に属するならば,$ w\in\cty{\epsilon}z $となり得る.
ここで$p$が上の条件を記述するということは,
\[
    (\exists l)(\orp{l,\orp{u,\app{\epsilon}v\cap\enf{\epsilon}v}}\in\mathcal{K}^{(\breve{\epsilon}\fap 0)}\con{1}
    \text{ $p$は$l$に解釈される}
    )
\]
ということを意味する(\fndp{16}{16}).

パターン2:これも\ref{ssec:抑制条件}で指摘されたが,修正条件またはその危険が実際にはなくても,それの言語的代替によって同様の情動が生じ得る.すなわち上記の例では,ある$n\subseteq\indx{\epsilon}$が存在して,
\[
   \tilde{x}\fap 0\in\prob{(\tildel{\trgl{y}}\fap 0)}{(n\uphl\epsilon)}
\]
であることを記述する文$q$が,その時点における$ \tilde{x}\fap 1 $の信念体系に属するならば,同様の情動が生じ得る.

パターン3:信念体系の要素間の論理的連関その他によって,上記のパターンにおける文$p$や$q$を肯定する傾向性は,信念体系に属する$ r_0,r_1,\dots,r_n $またはそれらの連言$r$を肯定する傾向性に依存することがあり得る.すると,$p$や$q$を肯定する傾向性によって生じる情動は,結局,$r$を肯定する傾向性に依存していることになる.この意味で,信念体系が情動の生起に関与する仕方は全体論的であり,個別の文を肯定する傾向性が1個の情動に対応するわけではない.

パターン4:修正条件(の危険)またはその言語的代替によって生じる情動は,関連する価値的・規範的判断に依存しているか,または,それらの判断によって不必要に増幅されていることが多いと考えられる.
例えば,上司か顧客の叱責によって生じた憂鬱は,「期日までにプロダクトを納品すべきだったのにしなかった」という義務違反に関する規範的判断と,そこから引き出される「自分は非難に値する」とか「自分はダメだ」等の価値的判断に依存している可能性がある.
関連して,上司か顧客の機嫌を損ねたことによって(仕事の依頼が減る等の)さらなる抑制条件が生じる危険を肯定する傾向性は,現在の状態に不安を追加し,憂鬱状態をさらに増幅するだろう.これ自体はパターン1であるが,この因果関係の信念は,還元されない反事実的条件法演算子によってその危険測度を曖昧化したまま表現されるものかもしれない.さらに,将来の状況の評価について「酷いことになる」等の漠然とした価値的判断を行なうことによって,意味もなく不安と憂鬱が増幅されているかもしれない.
以上とは逆に,本人がむしろ「自分に過失はなく,叱責は不当である」との信念を持っていたら,憂鬱ではなく「怒り」と呼ばれるような状態が生じると考えられる.この信念には,上司か顧客は「叱責行動をすべきでないのにした」という規範的判断が含まれている.

なお,本稿で言う価値的・規範的判断とは,典型的には,記述的な文$p$に対して,「〜は良い」,「〜は悪い」等の価値演算子または「〜であるべき」,「〜であってはならない」等の規範演算子を結合した文$p'$を肯定する傾向性だ.それに加えて,それ自体は内包演算子を持たないが,$p'$を肯定する傾向性と相互依存関係にあり,かつ,$p\con{1}p'$に解釈可能な$p''$を肯定する傾向性も,価値的・規範的判断に含めるとしよう.例えば,「彼は無能だ」,「今日の夕日は美しい」などの文だ.もっとも,このような分析は実はどうでもよい.さらには価値演算子と規範演算子の種類やそれらの区別,したがって価値判断と規範的判断の区別も重要ではない.いずれにしてもこれらは最適化によって排除されるからだ.

\subsubsection{メソッド}
\label{sssec:メソッド}

信念体系が不快な感情に関与するパターン1〜4にはいずれも,そのパターンが実現した場合,特定の不快な情動的状態が信念体系の部分集合$g$に依存するか,または,$g$によってその情動的状態が増幅されるようなケースである.メソッドは信念体系を部分的に最適化して$g$を取り除き,この依存関係を切断する.それによって,不快な感情を無害化するのだ.
この点に関して,人$x$と時間$t$について,ある$ e\subseteq x\cap t $を考える.また,$ \alpha = \classab{e:e\text{ は情動的状態}} $と置く.$t$と同時かそれに先行する信念体系の部分的最適化によって,抑制条件の実現にもかかわらず,$ e\in \prob{\alpha}{(1\uphl\epsilon)} = \alpha $であることが因果的に阻止されているとしよう.それでも依然としてあるレベル$n$について,当該修正条件が$e\in \prob{\alpha}{(n\uphl\epsilon)}$であることを惹起しているかもしれない\footnote{あるいは,$ e\neq e' $について,$ e'\in\alpha $もまた不快な情動的状態であり,当該修正条件は$ e'\in\alpha $を惹起しているかもしれない.}.そして,$e\in \prob{\alpha}{(n\uphl\epsilon)}$であること自体が別の不快な情動的状態と言えるケースもある.しかし,メソッドの継続的な適用の結果,残存する情動は無害なレベル$i$に止まるようになることが想定されている.$i\leq n$であるかどうかはメソッドの継続適用による最適化の進行具合による.

ところで,「無害化する」という言い方に既に表われているが,メソッドは不快な感情が何らかの意味で有害であることを前提している.このことを明確化するのに新しい概念は必要ない.不快な感情が有害であることは単に次のことを意味する.すなわち,抑制条件によって惹起される情動的状態もまた,当該情動がその内部で生起している人間の何らかの行動に対して,それ自体として抑制条件になり得る.そして,この情動的状態がさらに他の抑制条件を惹起することがあり得る.例えば,不安や憂鬱と呼ばれる状態は,その主体を不活性化して,他の有益な(促進条件を実現する)行動を阻害することに因果的に寄与している可能性がある.

もっとも,不快な感情の無害化それ自体がメソッドの目的なのではない.メソッドによって不快な感情を無害化するように信念体系を改訂する理由は,それに関与しているであろう(最適化によって除去される)信念体系の部分が,あなたが本来持っている創造の力を阻害しているからだ.メソッドは信念体系の最適化を通じてその力を解放するのだ.したがって,メソッドによって感情をどうにかすることが重要なわけではない.メソッドが感情に注目する唯一の理由は,それ自体が抑制条件になるような感情が生起しているとき,普通は,信念体系の最適化が不足しているということだ.そして,感情の生起態様やその強さ・持続性は,最適化を要する信念体系の部分を特定化するサインとして重要になる.

以上を要約すると,メソッドの適用の図式的な記述が得られる.それは極めて単純だ.すなわち,不快な感情が生起したとき,
\begin{enumerate}[label=(\arabic*)]
    \item 関連する信念体系の部分$g$を仮説的に特定化する.
    \item 信念体系を部分的に最適化して$g$を取り除く.
\end{enumerate}
以上が本稿のメソッドである.不快な感情が生起していないときでも,最適化可能な信念体系の部分$g$を発見したら,(2)を適用できる.それでも第一次的に不快な感情をトリガーに置く理由は,それを契機とする方が最適化可能な部分を見つけやすいからだ.もっとも,「最適化」の具体的な方法が明らかにならない限り,この図式自体はたいして意味はない.最適化の過程において一応次の二段階を区別することができる.
\begin{enumerate}[label=\roman*.]
    \item 論理的還元
    \item 存在論的還元
\end{enumerate}
iは,信念体系の部分集合を,量化論理によって統制された言語の文集合と入れ替える(規格化する)工程であるが,この言語は要するに\fndpp{18}{18--19}の全体的言語$\mathfrak{L}$のことだ.
他方,iiは,信念体系の存在論を集合論的に確定して,現実を再構築する工程だ.これ以上の詳細は第\ref{sec:論理的還元}節及び第\ref{sec:存在論的還元}節で展開される.
実際のところiの工程はiiの工程を含む仕方で行われることも多いから,最適化を二段階に分けるのはあくまで便宜的なものだ.

メソッドは繰り返し適用されることが想定されている.一度処理したように思えた情動的状態は,何度も繰り返しやってくる.その度にメソッドの適用を繰り返すのだ.メソッドの継続的な適用は,やがてあなたの魂をこの現実という楔から解き放ち,制限のない自由な空間へと飛翔させる.その経験がこれまでとは別の現実を創造することを可能にするのだ.
