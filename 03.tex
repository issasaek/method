% !TeX root = method.tex

\section{存在論的還元}
\label{sec:存在論的還元}

信念体系の最適化の第二段階は,第一段階である論理的還元(規格化)を通過した信念体系$\mathfrak{B}'$を,その存在論を確定する理論$\mathrm{ZFC}$を受容している信念体系$\mathfrak{B}''$へと改造する工程である.存在論は存在文のクラスであり,$\mathfrak{B}'$の存在論は$\mathfrak{B}''$のそれの部分クラスである.また,$\mathrm{ZFC}$は\fndp{9}{9}で提示した標準的な集合論の体系だ.

ところで,(人間の)信念体系$b$と理論$u$について,$u\subseteq b$であることは普通はない.公理の全ての帰結を人間が把握することはできないからだ.また,理論の公理は,公準として枚挙的に指定されるか,公理型式の代入例が公理であるという仕方で指定される.すると,$u$の公理集合$u'\subseteq u$と,$u$の公準の集合$u''\subseteq u'$について,$u' = u''$でない限り,$u'\subseteq b$であることも普通はない.公理型式の全ての代入例を人間が把握することはできないからだ.
そこで,$b$が$u$を受容している事態を次のように捉えることにしよう.すなわち,$ \orp{p,h} $に対して,文$p$が文型$h$の代入例であることを記述する文を与える関数を$\phi$とすると,任意の$p\in u$と,$u$の公準の集合$ u''$について,
\begin{align*}
    p\in u''\case{2}{1}{0}(\exists h)(h\text{ は$u$の公理型式}\con{1}\phi\fap\orp{p,h}\in b)\case{1}{1}{1}p\in b
\end{align*}
であるとき,かつそのときに限り,$b$は$u$を受容している.

次に,$\mathrm{ZFC}$が$\mathfrak{B}''$の存在論を確定している,と言うことの意味は\ref{ssec:存在論}で明らかになる.

\subsection{存在論}
\label{ssec:存在論}

一般に,標準言語$l$と$t\subseteq l$について,$t$のメンバーである存在文のクラスを$t$の存在論と言う.ここで存在文とは,\kagi{$ (\exists x)Fx $}の代入例となる標準言語の文だ.また,$t$は何らかの$l$の理論であるか,ある時点のある人の信念体系である.
以下$l = \mathfrak{L}$と想定しよう.$\mathfrak{L}$においては,$2$項述語記号の$0$番目が要素関係を表わす\kagi{$ \in $}に相当する原始的述語である(\ref{ssec:標準言語}).$u\subseteq l$が何らかのクラス理論であるとき,$u$の公理はこの\kagi{$ \in $}の用法を規定している.そして,$u$が$t$の存在論を確定しているというのは,以下の条件を充たすことを意味している.
\begin{enumerate}
    \item $u$は$t$が理論ならその部分クラスであり,$t$が信念体系ならそれに受容されている理論である.
    \item $t$の任意の原始的述語$f$について,$f$が$n$項述語記号の$i$番目であり,かつ,$\orp{n,i}\neq\orp{2,0}$であるとする.このとき,下記(1)(2)の型式に後述の代入を行った結果のいずれかが$t$のメンバーである.また,$t$が理論ならば当該代入結果のいずれかが$t$の公理である.さらに,(3)に後述の代入を行った結果が$u$の定理である.
    \setcounter{equation}{0}
    \begin{align}
        \classab{\orp{\alpha_1,\dots,\alpha_n}:R\alpha_1\dots\alpha_n} = \beta.\\
        \classab{\orp{\alpha_1,\dots,\alpha_n}:R\alpha_1\dots\alpha_n} \subseteq \beta.\\
        (\exists x)(x = \beta).
    \end{align}
    代入の仕方は次の通り.\kagi{$ R $}に$f$を,\kagi{$ \alpha_1\dots\alpha_n $}に$n$個の変項を並べて代入する.
    \kagi{$\orp{\alpha_1,\dots,\alpha_n}$}には,同じ$n$個の変項から作られる順序対表現を代入する.\kagi{$ \beta $}には$l$の文から作られるクラス抽象体を代入する.
\end{enumerate}

条件2の(1)(2)への代入結果を「還元規則」と言う.それは原始的述語の外延を規定することによって$t$の存在論を明確化する.
例えば,理論$t$のメンバーに\kagi{$ x\text{ は電子である} $}に相当する文が出現し,かつ,それが(原始的述語に変項を結合した)原子文であるとしよう.
すると,もしクラス理論$u\subseteq t$が$t$の存在論を確定しているならば,
\[
   \classab{x:x\text{ は電子である}}\subseteq \mathcal{P}(\timex{\mathbb{R}}{4})
\]
のような対応する還元規則が$t$のメンバーになっており,かつ,
\[
   (\exists x)(x = \mathcal{P}(\timex{\mathbb{R}}{4}))
\]
が$u$で証明可能である.この点,$\mathrm{ZFC}$から置換公理を除いた$\mathrm{ZC}$について,$\mathrm{ZC}\subseteq u$ならば実際に証明可能であり,さらに,上記の還元規則と分出公理(\ref{ssec:集合論の体系}のA \ref{axim:分出})によって,
\[
   (\exists y)(y = \classab{x:x\text{ は電子である}})
\]
が$t$で証明可能になる.他の原始的述語についても同様である\footnote{
    原始的でない複合的な述語に関する還元規則についても,これに出現する原始的述語の還元規則に基づいて,$t$で証明可能になり得る.例えば,$t$のメンバーに「$x$はマイナスの電荷を持つ素粒子である」に相当する複合文が出現する場合,これの還元規則と外延の存在が$t$で証明可能であり得る.
}.

なお,$t$の存在論が確定されているとき,それはクラス理論を($t$が信念体系なら少なくともその一部を)含んでいるから,量化の変項の値に制限はないとみなされる.この点,クラス理論に属しない文($t$が理論なら集合論以外の公理など)において,変項の値の範囲を実質的に制限したい場合は,\kagi{$ (x)(x\in\alpha\case{1}{1}{0}Fx) $}という明示的な形式を使うことができる.この形式では,束縛変項\kagi{$ x $}の値域が,\kagi{$ \alpha $}に代入される抽象体が指示するクラスへと実質的に制限される.

ところで,\fndpp{18}{18--20}において,真理集合$\mathrm{T}^{(\breve{\epsilon}\fap 0)}\cap\mathfrak{L}$の部分クラスとして,標準的な科学とテクノロジーに関連する理論を実質的に全て含むような包括的な$\mathfrak{L}$の理論$\mathfrak{K}$が想定されている.
この$\mathfrak{K}$の存在論は$\mathrm{ZC}$によって確定されると考えられる.つまり,$\mathrm{ZC}\subseteq\mathfrak{K}$であり\footnote{正確には,原初的言語ではなく$\mathfrak{L}$に合わせて$\mathrm{ZC}$を再解釈したものが$\mathfrak{K}$に含まれる.},また,言語$\mathfrak{L}$の構成上,同一性述語と数学的述語は原始的述語ではなく,定義によって導入される.したがって,同一性の法則や数学的法則は$\mathrm{ZC}$の定理として,$\mathfrak{K}$に取り込まれる.
この点について,第一段階を通過した信念体系$ \mathfrak{B}' $は,理想的には,$\mathfrak{B}'\subseteq\mathfrak{L}$であり,かつ,$\mathfrak{K}$を部分的に含んでいる.例えば,$ \orp{a,b}\in x\bkg{\epsilon}y $であることを記述する文が$ \mathfrak{B}' $のメンバーであるとき,
\[
    x\uphl(x\bkg{\epsilon}y)\subseteq \univ\times y
\]
であることを記述する$\mathfrak{K}$に属する一般化が,$ \mathfrak{B}' $のメンバーになるかもしれない.しかし,背景条件に言及する形ではなく,明示的に解釈空間に相対化された因果記述も$\mathfrak{B}'$に含まれ得る.つまり,
\[
    \orp{\alpha,\beta}\to_{\epsilon}\orp{\gamma,\delta}
\]
の型文字にそれぞれ$a,x,b,y$を指示する抽象体を代入した結果.\kagi{$ \epsilon $}は型文字のままか変項$v$に置換されると考えられる.そして,\kagi{$ \epsilon $}の位置に来る抽象体が指示するか,または,$v$の値となる解釈空間は,\fndp{18}{18}の条件1〜4を充たすものと($\mathfrak{B}'$において)想定されている.したがって,$\mathfrak{B}'$には,
\setcounter{equation}{0}
\begin{align}
    (\exists x)(\exists n)[\mathrm{sim}\,x\con{1}2\in n\in\mathbb{N}\con{1}\Lambda\neq \trgl{\arg x} = \mathcal{W}\fap n]
\end{align}
に相当する文や,それが含意する
\begin{align}
    (\exists z)(\exists n)(z = \mathcal{W}\fap n\neq\Lambda\con{1}2\in n \in\mathbb{N})
\end{align}
に相当する文($\mathrm{ZC}$のモデルの対象領域の存在を肯定する)が含まれ得る.また$\mathfrak{B}'$には,解釈空間に相対化された因果の概念だけでなく,それに基づく規制の概念も取り込まれている.すると規制類型の階層構造によって(\fndp{44}{44}),次の型式の\kagi{$ \gamma $}に$2$より大きい数を指示する抽象体を代入した結果も,$\mathfrak{B}'$に含まれ得る.
\begin{align}
    (\exists z)(z = \mathcal{W}\fap \gamma\neq\Lambda).
\end{align}
(1)(2)(3)は$\mathrm{T}^{(\breve{\epsilon}\fap 0)}\cap\mathfrak{L}$のメンバーではなく,それゆえに$\mathfrak{K}$や$\mathrm{ZC}$のメンバーでもない\footnote{
    もし$\mathfrak{K}$のメンバーならば,$\mathfrak{K}$のモデルの存在,つまり$\mathfrak{K}$の無矛盾性が$\mathfrak{K}$で証明可能になるだろう.しかしそれはあり得ない.$\mathrm{ZC}$についても同様である.清水~\cite[p.\,131,169]{清水}.
}.
これらに類する存在文を体系的に供給するには,$\mathrm{ZFC}$が必要である.$\mathfrak{B}'$から$\mathfrak{B}''$への移行は,還元規則と$\mathrm{ZFC}$の公理(の一部)を$\mathfrak{B}'$に追加することによって行われる.

\subsection{集合論の体系}
\label{ssec:集合論の体系}

\fndpp{4}{4--8}で導入された記法にによると,以下のA \ref{axim:外延性}〜A \ref{axim:選択}の(自由出現する変項をすべて普遍量化した)普遍閉鎖体が$\mathrm{ZFC}$の公理(型)である.A \ref{axim:分出}とA \ref{axim:置換}は公理の型であり,それぞれの\kagi{$ \alpha $}に$\mathfrak{L}$の開放文から作られるクラス抽象体を代入した結果が公理であることを表現している.他方,A \ref{axim:外延性}〜A \ref{axim:無限}及びA \ref{axim:正則性}とA \ref{axim:選択}は公準である.

\begin{axim}[外延性]
\label{axim:外延性}
$
    x=y\con{1}x\in z\case{1}{1}{1}y\in z.
$
\end{axim}

\begin{axim}[一対化,和,冪]
\label{axim:一対化,和,冪}
$
    \classab{x,y},\:\union{x},\:\mathcal{P}(x)\in\univ.
$
\end{axim}

\begin{axim}[無限]
\label{axim:無限}
$
    \mathbb{N}\in\univ.
$
\end{axim}

\begin{axim}[分出]
\label{axim:分出}
$
    x\cap\alpha\in\univ.
$
\end{axim}

\begin{axim}[置換]
\label{axim:置換}
$
    \func\alpha\case{1}{1}{1}\alpha\img x\in\univ.
$
\end{axim}

\begin{axim}[正則性]
\label{axim:正則性}
$
    x\neq \Lambda \ld{.}\supset (\exists y)(y\in x\md{.}x\cap y=\Lambda).
$
\end{axim}

\begin{axim}[選択]
\label{axim:選択}
$
    (\exists y)(
        y\subseteq\mathfrak{E}\con{1}x\cap\barl{\classab{\Lambda}}\subseteq\arg y
    ).
$
\end{axim}

\noindent 標準的な科学とテクノロジーが要求する数学は,\fndp{4}{4}の原初的言語に還元可能であり,その存在論は上記からA \ref{axim:置換}を除いた体系ZCによって供給できる.他方,物理理論その他の科学理論は原初的言語に還元可能ではないが,$\mathfrak{K}$の部分クラスであり,$\mathfrak{K}$の存在論はZCで供給可能である(\ref{sssec:物理的対象}).さらに因果と規制の概念は,解釈空間に相対化する形式ならば,原初的言語に還元可能である.しかし,それが要求する存在論(規制の概念の使用が想定される工学的理論の存在論)はZCでは供給できない.本稿で$\mathrm{ZFC}$が要請されたのはそのためだ.もっとも,$\mathrm{ZFC}$によって不可避的に,環境の予測と制御にとって必要な範囲を遥かに超えて,著しく存在論は拡張される.\ref{sssec:累積階層}でその全体像が確認される.

\subsubsection{物理的対象}
\label{sssec:物理的対象}

\fndp{6}{6}の定義によると,任意の$ x,y $について,$ x = y $であることは,
\[
    (z)(z\in x\case{3}{1}{1}z\in y)
\]
であることを意味する.すると,$ \neg(\exists w)(w\in x)\con{1}\neg(\exists w)(w\in y)\case{1}{1}{1}x = y $.すなわち,要素を持たない対象が存在するなら,それは1個しかない.他方A \ref{axim:分出}によって,$ \Lambda = x\cap\Lambda\in\univ $.すると,要素を持たない唯一の対象が存在して,それは空集合である.それゆえ,クラスではない個体が要素を持たないなら,そのようなものは存在しない.そこで,個体を非クラスではなく,$ x = \classab{x} $であるクラス$ x $として特徴づける流儀もある\footnote{
    クワイン~\cite[p.\,29]{クワインa}.
}.しかし上記の体系では,A \ref{axim:正則性}によって,$ (x)(x\notin x) $.したがって,この意味での個体も存在しない.
それで何も問題はない.個体を云々する唯一の動機は物理的対象の存在を供給することだが,そのために個体は必要ない.物理的対象は,非クラスでも自分自身を要素とするクラスでもない,純然たるクラスと結局のところ同一視できるからである.

すなわち,実数のクラス$\mathbb{R}$は$ \mathbb{R}\subseteq\mathcal{P}(\mathbb{N}) $となるように定義可能である\footnote{
    クワイン~\cite[pp.\,113--117]{クワインa}.
}.すると,個々の実数は$\mathbb{N}$の部分クラスとして,$\mathbb{R}$自体は$ \mathcal{P}(\mathrm{N}) $の部分クラスとして,A \ref{axim:分出}によって存在が供給される.また,\fndp{10}{10}のT 7により,$ (x)(y)(x\times y\in\univ) $.すると,$ \timex{\mathbb{R}}{4}\in\univ $.
$ \timex{\mathbb{R}}{4} $を四次元時空と同一視すれば,その要素を時空点,その任意の部分クラスを時空領域とみなすことができる.さらに進んで,物理的対象とそれが占有する時空領域との関係は1対1対応であるから,任意の時空領域を物理的対象と同一視する.すると,ある物理的対象が時空領域$y$を占有しているということは,単に$y$がある物理的状態にあることを意味するようになる.

このような同一視が言語$ \mathfrak{L} $で行われている場合,「物理的対象$x$は時空領域$y$を占有する」に相当する原始的述語は不要になる($\mathfrak{L}$の述語ではない).時空的占有関係は同一性に帰着するからである.
また,型式\kagi{$ (x)(Fx\case{1}{0}{1}x\subseteq \alpha) $}の\kagi{$ Fx $}に物理的対象(またはその有限クラス)を特徴づける文脈を,\kagi{$ \alpha $}に\kagi{$ \alpha\in\univ $}が$\mathrm{ZFC}$で証明可能なクラス抽象体を代入した結果のあるものが,$ \mathrm{T}^{(\breve{\epsilon}\fap 0)}\cap\mathfrak{L} $のメンバーであるとみなされる(架橋法則).
例えば,\kagi{$ Fx $}に\kagi{$ x\text{ は電子である} $}を,\kagi{$ \alpha $}に\kagi{$ \timex{\mathbb{R}}{4} $}を代入した結果.
他の例として,\kagi{$ Fx $}に
\[
    (\exists a)(\exists t)(x = \classab{a,t}\con{1}Gat)
\]
を,\kagi{$ \alpha $}に\kagi{$ \mathcal{P}(\timex{\mathbb{R}}{4}) $}を代入した結果.なお\kagi{$ Gat $}は,物理的対象$ a $が時間$ t $に一定の状態にあることを記述する文脈である.
ただし,後者の架橋法則については,次のようにして$ \mathfrak{L} $の原始的述語が調整されるかもしれない.\kagi{$ Gat $}が表わす文脈が2項の原始的述語$p$を持つ原子文だとすると,$p$を破棄して,代わりに$a$の$t$における時間的部分$a\cap t$について真である1項述語$q$を使用する.すなわち,
下記の\kagi{$ Gat $}に$p$を持つ原子文を,\kagi{$ Hx $}に$q$を持つ原子文を代入した結果が真であるなら,$p$を廃棄して,$q$を残す.
\[
    (a)(t)(Gat\case{3}{0}{0}(\exists x)(Hx\con{1}x = a\cap t)).
\]

なお,時間の概念については,$x,y\in\timex{\mathbb{R}}{4}$がある座標系$f$に相対的に同時であるということを,
\[
    \mathcal{L}\fap(f\fap x) = \mathcal{L}\fap(f\fap y)
\]
として,そこから適当に構成することができる(\fndp{38}{38}).座標系$f$は,$f\subseteq\timex{\mathbb{R}}{4}\times\timex{\mathbb{R}}{4}$なる1対1変換であり,物理理論に適合するものである.
他方,空間については,上記の\kagi{$ \mathcal{L} $}を\kagi{$ \mathcal{R} $}に交換すると,$f$に相対的な同位置性の概念が得られる.

\subsubsection{累積階層}
\label{sssec:累積階層}

置換公理(A \ref{axim:置換})によってクラスの累積的な階層構造が導入できるが,それは順序数の概念に基づいている\footnote{以下の順序数の定義は,クワイン~\cite[pp.\,127--144]{クワインa}による.}.順序数は,$ \union{x}\subseteq x $であり,かつ,$ \mathfrak{E}\uphr x $が整列順序であるような任意の$ x $だ.整列順序は基礎づけられた順序関係である.
この点,関係$\alpha$が基礎づけられているのは,$x$の任意のメンバーに対して,$x$のあるメンバーが再び$\alpha$を持つような$x\neq \Lambda$が存在しないならばである.すなわち,
\begin{df}
\label{df:基礎}
\kagi{$
    \mathrm{Fnd}\,\alpha
$}は\kagi{$
    (x)(x\subseteq\breve{\alpha}\img x\case{1}{1}{1}x = \Lambda)
$}を表わす.
\end{df}

\noindent 次に,順序関係は,連結的($\mathrm{Connex}\,\alpha$)であり,推移的($ \alpha\resl\alpha\subseteq\alpha $),かつ非反射的($\alpha\subseteq\bar{I}$)な$\alpha$である.

\begin{df}
\label{df:連結性}
\kagi{$
    \mathrm{Connex}\,\alpha
$}は\kagi{$
    (x)(y)(x,y\in(\alpha\cup\breve{\alpha})\img\univ\case{1}{1}{1}\orp{x,y}\in\alpha\cup\breve{\alpha}\cup I)
$}を表わす,
\end{df}

\begin{df}
\label{df:順序}
\kagi{$
    \mathrm{Ordg}\,\alpha
$}は\kagi{$
    \alpha\resl\alpha\subseteq\alpha\con{1}\alpha\subseteq\bar{I}\con{1}\mathrm{Connex}\,\alpha
$}を表わす.
\end{df}
\noindent 推移性と非反射性から,非対称性($\alpha\subseteq\barl{\breve{\alpha}}$)が結果する.つまり,
$ \alpha\resl\alpha\subseteq\alpha\subseteq\bar{I}\case{1}{1}{1}\alpha\subseteq\barl{\breve{\alpha}} $.そして,次の定義は順序数のクラスを導入する.
\begin{df}
\label{df:順序数}
\kagi{$
    \mathrm{NO}
$}は\kagi{$
    \classab{x:\union{x}\subseteq x\con{1}\mathrm{Fnd}\,\mathfrak{E}\uphr x\con{1}\mathrm{Ordg}\,\mathfrak{E}\uphr x}
$}を表わす\footnote{
    A \ref{axim:正則性}の正則性公理によると,そもそも$ \mathrm{Fnd}\,\mathfrak{E} $であり,$ \mathfrak{E}\subseteq\bar{I} $だから,順序数の定義は実はもっと簡略化できるが,正則性を前提しない定義の方が汎用的である.
}.
\end{df}

\noindent おのおのの順序数はそれ以前のすべての順序数のクラスであり,順序数の大小関係は$ \mathfrak{E}\uphr\mathrm{NO} $である.また,有限順序数は単に自然数であり($ \mathbb{N}\subseteq\mathrm{NO} $),$\mathbb{N}$自身は最初の超限順序数$\omega$である($\mathbb{N}=\omega$).
また,後続者関数を$ \mathrm{S} = \lambda_x(x\cup\classab{x}) $とすると(\fndp{8}{8}),$ x\in(\mathrm{S}\img\univ)\cap\mathrm{NO} $は先行者を持つ後続順序数であり,$ x\in\barl{(\mathrm{S}\img\univ)}\cap\mathrm{NO} $は直接の先行者を持たない極限順序数である.

次に,右域が順序数か$ \mathrm{NO} $である一般化された系列を導入して,そこから累積階層の超限系列$\mathrm{W}$を定義する.
\begin{df}
\label{df:超限系列}
\kagi{$
    \seq\alpha
$}は\kagi{$
    \func\alpha\con{1}\breve{\alpha}\img\univ\in\mathrm{NO}\case{2}{1}{1}\breve{\alpha}\img\univ = \mathrm{NO}
$}を表わす,
\end{df}

\begin{df}
\label{df:累積階層}
\kagi{$
    \mathrm{W}
$}は\kagi{$
\union{
    \classab{w:\seq w\con{1}
    w\fap\Lambda = \Lambda\con{1}
    (x)[x\in\breve{w}\img\univ\case{1}{1}{2}\\\hfill
        x\in\mathrm{S}\img\univ\case{1}{1}{1}w\fap x = \mathcal{P}(w\fap(\breve{\mathrm{S}}\fap x))\con{2}
        \Lambda\neq x \notin \mathrm{S}\img\univ\case{1}{1}{1}w\fap x = \union{(w\img x)}
    ]
}
}
$}を表わす.
\end{df}

\noindent A \ref{axim:正則性}の正則性公理は,$ \univ = \union{(\mathrm{W}\img\univ)} $と等値である\footnote{キューネン~\cite[p.\,134]{キューネン}}.
すべての存在者は,$ \Lambda $から始まる反復的な過程のどこかの段階,すなわち,ある水準$x\in\mathrm{NO}$の階層$ \mathrm{W}\fap x $の要素として現れる.
この点,\fndp{11}{11}の$ \mathcal{W} $は,$\mathrm{W}$と$n\in\mathbb{N}$に$\omega\cdot n$を与える関数(その左域が$ \omega^{2} $未満の極限順序数のクラスになる)との積である.つまり,
\[
    \mathcal{W} = \mathrm{W}\resl\classab{\orp{\omega\cdot n,n}:n\in\mathbb{N}}.
\]
そして,順序数の加法・乗法・累乗の一般的な定義をどう構成するにしても,
\begin{align*}
    \func z\con{1}\arg z = \mathbb{N}\con{1}z\fap\Lambda = \Lambda\con{1}z\resl\mathrm{S}\resl\breve{z}\subseteq \lambda_x[x\cup(\lambda_i(\iter{\mathrm{S}}{i}\fap x)\img\mathbb{N})]
\end{align*}
なる無限系列$ z $について,$ \omega\cdot n = z\fap n $.また,$ \omega^{2} = \union{(z\img\univ)} $.
なお関連して,$ \func \lambda_i(\iter{\mathrm{S}}{i}\fap x) $,かつ$ \mathbb{N}\in\univ $(A \ref{axim:無限}).それゆえA \ref{axim:置換}により,
\[
    \lambda_i(\iter{\mathrm{S}}{i}\fap x)\img\mathbb{N}\in\univ.
\]
また,$ (x)(y)(x\cup y\in\univ) $であるから(\fndp{10}{10}のT 6),
\[
    x\cup(\lambda_i(\iter{\mathrm{S}}{i}\fap x)\img\mathbb{N})\in\univ.
\]

①環境の制御と予知に直接使用する言語的装置においても,②またその延長上に想定される規制の概念を使用する工学的過程においても,必要とされる存在者の領域は,ある$n\in\mathbb{N}$について$ \mathrm{W}\fap(\omega\cdot n) $であるか,せいぜい$ \mathrm{W}\fap(\omega^{2}) $である.それは,どの水準の階層でもそうであるが,存在者の全領域の無限に小さな部分でしかない.特に①の存在論はZCで供給可能である.そして,$ \omega\in x\notin \mathrm{S}\img\univ $なる任意の$x\in\mathrm{NO}$について,$\mathrm{W}\fap x$はZCのモデルになる\footnote{
    キューネン~\cite[p.\,192]{キューネン}.
}.すなわち,$ u = \mathrm{W}\fap x\con{1}r\fap\orp{2,0} = \mathfrak{E}\uphr(\mathrm{W}\fap x) $なる任意のモデル$ \orp{u,r}\in\mathrm{MD} $において,ZCの公理は全て真である.したがって,既に$ \mathrm{W}\fap(\omega\cdot 2) $を対象領域とするモデルが,(ZCを含む)世界に関する包括的理論$ \mathfrak{K} $のモデルになると想定できるのだ.

\subsection{最終工程}
\label{ssec:最終工程}

信念体系の最適化の第二段階は,$\mathrm{ZFC}$を受容して自覚的に存在論を確定することだ.メソッドの適用という実践としては,それは,我々を苦しめたり喜ばせたりする物理的環境を,その都度存在論的に再記述する工程として現れる.「その都度」と言うのは,不快な感情が生起した場面か,または,信念体系の最適化可能な部分を発見した場面において,と言うことだ.不快な感情それ自体も含めて,関連する物理的システムは何らかのクラスであること,それは$ \Lambda $から始まる反復的過程の中にあること,その事実を信念体系に浸透させるように事態を再記述するのだ.
しかし,第二段階はこれで終わりではない.最後のステップがある.

\subsubsection{代替モデル}
\label{sssec:代替モデル}

本稿のメソッドの中核は,第一段階及び第二段階を通じて,信念体系の要素を標準言語の文,要するに$\mathfrak{L}$の文へと規格化・形式化することだった.そして\ref{ssec:外延性}では,標準言語に共通する性質として外延性が言われた.共通する重大な特徴がもう1つある.モデルの代替性だ.すなわち,任意の論理式のクラス$ l\subseteq\mathrm{L} $について,$l$のモデル$m\in\mathrm{MD}$が存在するなら,$m$以外の無限に多くの$l$のモデルが存在する.ここで$m$が$l$のモデルであると言うのは,$l$の要素が$m$で全て真であるということ,つまり$ l\subseteq\mathrm{T}^{m} $ということを意味する.モデルを固定できないということは$l$は特定の存在論的基盤を持たないということだ\footnote{
    クワイン~\cite[pp.\,44-46]{クワインe}.
}.$l$は科学とテクノロジーに関連する全ての理論を包括した理論$\mathfrak{K}$であり得る(\ref{ssec:存在論}).

モデルの代替性は(ZC及び$\mathrm{ZFC}$の)3つの定理(T \ref{thm:代替モデル充足},T \ref{thm:代替モデル真理},T \ref{thm:LST})にまとめることができる.それを示すためにいくつかの概念を準備しよう.
\paragraph*{論理式の複雑度}任意の$p\in\mathrm{L}$にその複雑度(真理関数及び量化子の出現回数)を与える関数$\mathfrak{d}$を再帰的に規定する.すなわち,$p$が原子式のとき,$\mathfrak{d}\fap p = 0$.また,任意の$q,q'\in\mathrm{L}$について,
\begin{gather*}
p = \orp{1,q} \case{1}{1}{1} \mathfrak{d}\fap p = \mathfrak{d}\fap q+1,\\
p = \orp{2,q,q'} \case{1}{1}{2} \mathfrak{d}\fap p = \mathfrak{d}\fap q + \mathfrak{d}\fap q' +1,\\
(\exists i)(p = \orp{3,i,q}) \case{1}{0}{1} \mathfrak{d}\fap p = \mathfrak{d}\fap q +1.
\end{gather*}
\paragraph*{代替関数と代替モデル}$ m,m'\in\mathrm{MD}\con{1}m = \orp{u,r}\con{1}m'=\orp{u',r'} $とする.このとき,
\begin{gather*}
    \func f\con{1}\func\breve{f}\con{1}f\img\univ = u'\con{1}\breve{f}\img\univ = u,\\
    r' = \classab{\orp{v,\orp{n,i}}:
    v = \classab{\mathcal{O}\fap x:
        (\exists z)(
            \mathcal{O}\fap z\in r\fap \orp{n,i}\con{1}x = f\resl z
        )
    }\con{1}\orp{n,i}\in\arg r
    }
\end{gather*}
である$f$を,$ m $から$ m' $への代替関数と言う.そして,$ m $から$ m' $への代替関数が存在するとき,$m'$は$m$の代替モデルと言う.

\begin{thm}
\label{thm:代替モデル充足}
$f$が$ m = \orp{u,r} $から$ m' = \orp{u',r'} $への代替関数ならば,任意の$p\in\mathrm{L}$について,
\[
    (s)(s\in \mathrm{seq}^u\case{1}{1}{2}\orp{s,p}\in\mathrm{SR}^m\case{3}{1}{1}\orp{f\resl s,p}\in\mathrm{SR}^{m'}).
\]
\end{thm}
\begin{pfx}
\setcounter{equation}{0}
論理式の複雑度に関する帰納法による.
\[
    \alpha = \classab{n:
        (p)[p\in\mathrm{L}\con{1}\mathfrak{d}\fap p = n\case{1}{1}{0}
            (s)(s\in \mathrm{seq}^u\case{1}{1}{2}\orp{s,p}\in\mathrm{SR}^m\case{3}{1}{1}\orp{f\resl s,p}\in\mathrm{SR}^{m'})
        ]
    }
\]
と置く.そして,任意の$ n\in\mathbb{N} $について,$ n\subseteq \alpha\case{1}{1}{1}n\in\alpha $であること,つまり,
\begin{equation}
    n\in\mathbb{N}\con{1}n\subseteq\alpha \tag*{[1]}
\end{equation}
であると仮定して,$ n\in\alpha $であることを示す.そこで,
\begin{equation}
    p\in\mathrm{L}\con{1}\mathfrak{d}\fap p = n
\end{equation}
であると仮定する.

$n = 0$のケース:$p$は原子式であり,ある$n,i,k$が存在して,$ p = \orp{0,\orp{n,i},k} $.すると任意の$s\in \mathrm{seq}^u$について,
\begin{align*}
    \orp{s,p}\in \mathrm{SR}^m & \case{3}{1}{1} \mathcal{O}\fap (s \resl k)\in r\fap\orp{n,i}\\
    &\:\, \case{3}{0}{1} \mathcal{O}\fap (s \resl k)\in r\fap\orp{n,i}\con{1}(\exists x)(x = f\resl(s \resl k))\\
    &\:\, \case{3}{0}{1} \mathcal{O}\fap (f\resl s \resl k)\in r'\fap\orp{n,i}\\
    &\:\, \case{3}{0}{1} \orp{f\resl s,p}\in \mathrm{SR}^{m'}.
\end{align*}
したがって,
\begin{equation}
    0 = n\case{1}{1}{0}(s)(s\in \mathrm{seq}^u\case{1}{1}{2}\orp{s,p}\in\mathrm{SR}^m\case{3}{1}{1}\orp{f\resl s,p}\in\mathrm{SR}^{m'}).
\end{equation}

$n \neq 0$のケース:この場合,$p$は否定,条件法,普遍量化のいずれか.つまり,
\begin{equation*}
    (\exists q)(p = \orp{1,q})\case{2}{0}{0}(\exists q)(\exists q')(p = \orp{2,q,q'})\case{2}{0}{0}(\exists q)(\exists i)(p = \orp{3,i,q}).
\end{equation*}

ある$q$が存在して$p = \orp{1,q}$であるとき:$\mathfrak{d}\fap q\in n$.すると任意の$s\in\mathrm{seq}^{m}$について,
\begin{align*}
    \orp{s,p}\in\mathrm{SR}^m & \case{3}{1}{1}\orp{s,q}\notin\mathrm{SR}^m \\
     &\:\, \case{3}{0}{1}\orp{f\resl s,q}\notin\mathrm{SR}^{m'} \tag{[1]により}\\
     &\:\, \case{3}{0}{1}\orp{f\resl s,p}\in\mathrm{SR}^{m'}.
\end{align*}

ある$q,q'$が存在して$p = \orp{2,q,q'}$であるとき:$\mathfrak{d}\fap q,\mathfrak{d}\fap q'\in n$.すると任意の$s\in\mathrm{seq}^{m}$について,
\begin{align*}
    \orp{s,p}\in\mathrm{SR}^m & \case{3}{1}{2}\orp{s,q}\in\mathrm{SR}^m\case{1}{1}{1}\orp{s,q'}\in\mathrm{SR}^m\\
    &\:\, \case{3}{0}{2} \orp{f\resl s,q}\in\mathrm{SR}^{m'}\case{1}{1}{1}\orp{f\resl s,q'}\in\mathrm{SR}^{m'} \tag{[1]により}\\
    &\:\, \case{3}{0}{1} \orp{f\resl s,p}\in\mathrm{SR}^{m'}.
\end{align*}

ある$q$が存在して$p = \orp{3,i,q}$であるとき:$\mathfrak{d}\fap q\in n$.任意の$s\in\mathrm{seq}^{m}$について,$\orp{s,p}\in \mathrm{SR}^m$であると仮定する.すると,任意の$a\in u$について,
\begin{align*}
    a\in u & \case{1}{1}{1}\orp{s\tbinom{i}{a},q}\in\mathrm{SR}^m\\
    & \:\,\case{1}{0}{1}\orp{f \resl (s\tbinom{i}{a}),q}\in\mathrm{SR}^{m'} \tag{[1]により}\\
    & \:\,\case{1}{0}{1}\orp{(f\resl s)\tbinom{i}{f\fap a},q}\in\mathrm{SR}^{m'}.
\end{align*}
他方,任意の$a'\in u'$について,ある$a\in u$が存在して,$ a' = f\fap a $.すると,
\begin{equation*}
    \orp{(f\resl s)\tbinom{i}{a'},q}\in\mathrm{SR}^{m'}.
\end{equation*}
それゆえ,$ \orp{f\resl s,p}\in \mathrm{SR}^{m'} $.他方,$ \orp{f\resl s,p}\in \mathrm{SR}^{m'}$と仮定して,証明を逆に辿ると,$\orp{s,p}\in \mathrm{SR}^m $.

以上から,
\begin{equation}
    0 \neq n\case{1}{1}{0}(s)(s\in \mathrm{seq}^u\case{1}{1}{2}\orp{s,p}\in\mathrm{SR}^m\case{3}{1}{1}\orp{f\resl s,p}\in\mathrm{SR}^{m'}).
\end{equation}

(1)(2)(3)により,$ n\in\alpha $.[1]の仮定と合わせると,$ (n)(n\in\mathbb{N}\con{1}n\subseteq\alpha\case{1}{1}{1}n\in\alpha) $.したがって,任意の複雑度$ j\in\mathbb{N} $について,$ j\in\alpha $.つまり,任意の$p\in\mathrm{L}$について,
\[
    (s)(s\in \mathrm{seq}^u\case{1}{1}{2}\orp{s,p}\in\mathrm{SR}^m\case{3}{1}{1}\orp{f\resl s,p}\in\mathrm{SR}^{m'}).
\]

\end{pfx}

\begin{thm}
\label{thm:代替モデル真理}
$ m' = \orp{u',r'} $が$ m = \orp{u,r} $の代替モデルならば,$\mathrm{T}^{m} = \mathrm{T}^{m'}$.
\end{thm}
\begin{pfx}
\setcounter{equation}{0}
T \ref{thm:代替モデル真理}の仮定により,$m$から$m'$への代替関数$ f $が存在する.

次に,$p\in\mathrm{T}^{m}$と仮定すると,
\begin{equation}
    (s)(s\in\mathrm{seq}^{u}\case{1}{1}{1}\orp{s,p}\in\mathrm{SR}^{m}).
\end{equation}
任意の$s'\in \mathrm{seq}^{u'}$について,$ \breve{f}\resl s'\in \mathrm{seq}^{u} $.すると,(1)により,$ \orp{\breve{f}\resl s',p}\in\mathrm{SR}^{m} $.また,T \ref{thm:代替モデル充足}により,
\[
   \orp{\breve{f}\resl s',p}\in\mathrm{SR}^{m}\case{3}{1}{1}\orp{f\resl(\breve{f}\resl s'),p}\in\mathrm{SR}^{m'}.
\]
したがって,$ \orp{f\resl(\breve{f}\resl s'),p}\in\mathrm{SR}^{m'} $.$ f\resl(\breve{f}\resl s') = s' $だから,
\begin{equation*}
    (s')(s'\in\mathrm{seq}^{u'}\case{1}{1}{1}\orp{s',p}\in\mathrm{SR}^{m'}).
\end{equation*}
すなわち,$ p\in\mathrm{T}^{m'} $.$ p\in\mathrm{T}^{m'}\case{1}{1}{1}p\in\mathrm{T}^{m} $も同様にして証明できるから,$ \mathrm{T}^{m'}=\mathrm{T}^{m'} $.
\end{pfx}

\begin{thm}
\label{thm:LST}
任意の$ l\subseteq\mathrm{L} $について,$l$がモデルを持つならば,$l$は可算モデル(対象領域が$\mathbb{N}$と同じサイズを持つ)を持つ.論理的記法で言い換えると,任意の$ l\subseteq\mathrm{L} $について,
\begin{multline*}
    (\exists m)(m\in\mathrm{MD}\con{1}l\subseteq \mathrm{T}^{m})\case{1}{0}{0}\\
    (\exists m')[
        m'\in\mathrm{MD}\con{1}l\subseteq \mathrm{T}^{m'}\con{1}
        (\exists f)(\func f\con{1}\func\breve{f}\con{1}f\img\univ = \mathbb{N}\con{1}\breve{f}\fap\univ = \mathcal{L}\fap(m'))
    ].
\end{multline*}
\end{thm}
\begin{pf}
    省略する\footnote{
        清水~\cite[p.\,131]{清水},クワイン~\cite[pp.\,190--193]{クワインb}等を参照.
    }.
\end{pf}

\noindent T \ref{thm:LST}はレーヴェンハイム・スコーレムの定理(LST)と呼ばれる.LSTによって,モデルを持つ論理式の集合は,$\mathbb{N}$を領域とするモデルを持つことが帰結する.すなわち,ある$m\in\mathrm{MD}$について,$ l\subseteq\mathrm{T}^{m} $であるとき,T \ref{thm:LST}により,$ \mathbb{N} $と$\mathcal{L}\fap m$との間の1対1変換$f$が存在する.すると,$ \mathcal{L}\fap (m') = \mathbb{N} $である$m$の代替モデル$m'$が存在する.したがって,T \ref{thm:代替モデル真理}によって,$l\subseteq\mathrm{T}^{m'} $.

\subsubsection{相対化}
\label{sssec:相対化}

1つのモデルが特定されている場合,そこから代替モデルを作るのは極めて容易だ.解釈空間の開始点$\breve{\epsilon}\fap 0 $について考えよう.それは$\mathfrak{L}$における標準的モデルであり(\fndp{19}{19}),包括理論$\mathfrak{K}$のモデルであると想定されている($ \mathfrak{K}\subseteq \mathrm{T}^{(\breve{\epsilon}\fap 0)} $).
今,$\breve{\epsilon}\fap 0 = \orp{u,r}\con{1}f = \lambda_x\classab{x}\uphr u$と置く.そして,
\begin{gather*}
    u' = (\lambda_x\classab{x}\uphr u)\img\univ,\\
    r' = \classab{\orp{v,\orp{n,i}}:
        v = \classab{\mathcal{O}\fap x:
            (\exists z)(
                \mathcal{O}\fap z\in r\fap \orp{n,i}\con{1}x = f\resl z
            )
        }\con{1}\orp{n,i}\in\arg r
    }
\end{gather*}
とすると,$f$は,$\orp{u,r}$から$\orp{u',r'}$への代替関数であり,$\orp{u',r'}$は$\orp{u,r}$の代替モデルである.この場合の代替関数は対象領域$ u $のメンバーをその単一クラスに変換する.他方,$ f = \lambda_x(\bar{x}\cap u)\uphr u $とすると,$u$上の補クラスに変換する.また,$ f = \lambda_x(\orp{\Lambda,x})\uphr u $とすると,$u$のメンバーを$\Lambda$とそれの順序対に変換する.
このような代替関数は際限なく作れるが,代替モデルの対象領域は元のモデルの対象領域と同じサイズに制限されている.しかし,どのような$\mathfrak{K}$のモデルが存在するにしても,LSTによって,$\mathbb{N}$を領域とする$\mathfrak{K}$のモデルが存在することになる.また,$\mathrm{T}^{(\breve{\epsilon}\fap 0)}\subseteq \mathrm{T}^{(\breve{\epsilon}\fap 0)}$であるから,$ \breve{\epsilon}\fap 0 $はトリヴィアルに$\mathrm{T}^{(\breve{\epsilon}\fap 0)}$のモデルだ.しかし,このことから,$\mathrm{T}^{(\breve{\epsilon}\fap 0)}$のメンバーが全て真となるような$\mathbb{N}$を領域とする別のモデルが存在することが帰結する.

T \ref{thm:代替モデル充足}〜T \ref{thm:LST}による帰結が意味することは,標準言語の理論あるいは文の集合が前提とする存在論は,モデルあるいはモデルを特定化する言語に相対的であるということだ.したがって,何が存在するかは事実の問題ではない.もちろん本稿では,定理の証明も$\mathfrak{K}$とそのモデルを対象化して考察することも,$\mathrm{ZFC}$を使用して行われている.したがって,$\mathrm{ZFC}$が認める存在者が存在することは予め前提されている.$\mathrm{V} = \union{(\mathrm{W}\img\univ)}$であることは前提されているのだ.問題は,そうだとしても,包括的な世界記述$\mathrm{T}^{(\breve{\epsilon}\fap 0)}$や,標準的な科学を包括する理論$\mathfrak{K}$がモデルを持つ限り,それらが肯定する存在が$\union{(\mathrm{W}\img\univ)}$のどのメンバーであるのかは,事実の問題ではないということである.要するに,我々が馴染んでいる物理的世界はないということだ.累積階層のどの部分についても,本性として現実性を帯びているような部分はない.

T \ref{thm:代替モデル充足}〜T \ref{thm:LST}は$\mathrm{ZFC}$(のモデル)に対しても適用される.ただし,それらが意味を持つのは$\mathrm{ZFC}$がモデルを持つ場合である.矛盾する理論が前提する存在者を云々しても無意味だからだ.しかし,$\mathrm{ZFC}$のモデルを$\mathrm{ZFC}$の中で構成することはできない.したがって,$\mathrm{ZFC}$の存在論を有意味に相対化するには,$\mathrm{ZFC}$より強いクラス理論を使用する必要がある.体系を強める典型的な方法は,いわゆる巨大基数の存在を認める公理を追加することだ\footnote{
    薄葉・藤田~\cite{薄葉}等を参照.
}.しかし,そうすると今度は,その強められた体系$\mathrm{ZFC}^{+}$が認める存在者が予め前提されるだけだ.そこには$\mathrm{ZFC}$が認める存在者も含まれている.そして$\mathrm{ZFC}^{+}$の中で同様にして,$\mathrm{T}^{(\breve{\epsilon}\fap 0)}$及び$\mathfrak{K}$の存在論は相対化される.さらに$\mathrm{ZFC}^{+}$に対しても,それが認める以上の巨大基数を要請してそれを強めることができるが,後の展開は同じことだ.要するに,クラスの存在論は相対化できないということだ.これに対して,物理的世界の存在論は$ \mathrm{ZFC},\mathrm{ZFC}^{+},\mathrm{ZFC}^{++},\dots $のどの段階でも相対化される.

物理的世界の記述から存在論的基盤が剥ぎ取られると,残るのは言語の使用が喚起するビジョンだけだ.物理的世界を支えている固有の対象や領域はない.したがって,あなたの周囲の現実が強固に感じられる理由は,それが実在的だからではなく,あなたが特定のモデルあるいはそれを特定化する記述を強固に受け入れていることの反映でしかない.
因果と規制の概念が解釈空間に相対化されていることは,この事実を暗示している.1つの解釈空間を選択することは,1つの現実(開始点$ \breve{\epsilon}\fap 0 $)と可能性の範囲を固定することだ.
さらに,あなたが普段それに自己同一化している人間,あるいは,あなたの思考・感情・行動もまた物理的システムであり,それゆえに,実体のない幻影にすぎない.全ての存在は無限の全体性$\union{(\mathrm{W}\img\univ)}$の要素であり,$\union{(\mathrm{W}\img\univ)}$の中には,それらの物理的システムであるような固有の対象はないからだ.
こうして物理的自我は消え去り,無限の全体性だけが残される.他方で,いわゆる形而上学的主体(哲学的自我)\footnote{ここでは,世界の全てが自らの思考・表象の内容であるような主体を意味している.}が露呈するが,それは世界そのものとして無限の全体性に同一化される.物理的世界とは,この拡張された自我が観照する,特定のモデルに基づくビジョンなのだ\footnote{このとき当該モデルは「現実性を持つ」とも言える.}.
メソッドの要点は,言語を規格化して存在をクラスに還元し,物理的世界を相対化することだ.信念体系の最適化が進行するにつれて,物理的システムとしてのあなたはより適応的になるだろう.しかし最も重要なことは,最適化プロセスの終了後に,あなたが本来のあり方へと,全てを内在する全体性へと回帰するということだ.

\section{結論}
\label{sec:結論}

本稿で構築されたメソッドの適用をまとめよう.まず,不快な感情が生起したとき,
\begin{enumerate}[label=(\arabic*)]
    \item 関連する信念体系の部分$g$を仮説的に特定化する,
    \item 信念体系を部分的に最適化して$g$を取り除く.
\end{enumerate}
不快な感情が生起していないときでも,最適化可能な信念体系の部分$g$を発見したら,(2)を適用する.

次に,(2)の適用の仕方はこうだ.
\begin{enumerate}[label=(\alph*)]
    \item $g$の内包性を除去して論理的表記法に規格化した$g'$について,$g'$を信念体系に取り込み,$g$を廃棄する.
    \item $g$に関連して存在論的に不明瞭な部分があれば,$\mathrm{ZFC}$に基づいて存在論を確定する.
    \item 代替モデルに関する定理に基づいて,物理的世界の存在論を相対化する.
\end{enumerate}

