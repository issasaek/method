\documentclass[leqno]{ltjsarticle}
\renewcommand{\headfont}{\bfseries} % 見出しフォント
\usepackage{luatexja}
\usepackage{amsmath,amssymb,amsthm}
\usepackage{newpxtext,newpxmath}
\usepackage{enumitem} % 箇条書きのカスタマイズ
\usepackage[absolute,overlay]{textpos}
\usepackage{url}
\newcommand{\version}{v1.2.3}
\usepackage[luatex,unicode,pdfencoding=auto]{hyperref}
\hypersetup{
  bookmarksnumbered=true,
  colorlinks=true,
  allcolors=blue,
  pdftitle={Loredux-Method},
  pdfauthor={佐伯 一冴},
  pdfsubject={\version},
  pdfkeywords={}
}
% 定理環境
\newtheoremstyle{mystyle}%   % スタイル名
    {5pt}%                   % 上部スペース
    {5pt}%                   % 下部スペース
    {}%                      % 本文フォント
    {}%                      % 1行目のインデント量
    {\bfseries}%             % 見出しフォント
    {}%                      % 見出し後の句読点
    {9.247pt}% 全角の幅        % 見出し後のスペース
    {\thmname{#1}\thmnumber{\hspace{2pt}#2}\thmnote{\hspace{7pt}[\gtfamily#3]}}%                      % 見出しの書式
\theoremstyle{mystyle}
\newtheorem{df}{D}[section]% 定理環境(定義)
\newtheorem{axim}{A}[]% 定理環境(公理)
\newtheorem{thm}{T}[]% 定理環境(定理)
\newtheorem*{pf}{証明.}% 定理環境(証明) \textbf{照明.}
\newtheorem*{pfx}{証明の概略.}% 定理環境(証明)
\newtheorem*{nom}{正規性論証.}% 定理環境(証明)
\newtheorem*{dem}{論証.}% 定理環境(証明)
% log型関数
\DeclareMathOperator{\func}{Func}
\DeclareMathOperator{\cor}{Cor}
\DeclareMathOperator{\seq}{SEQ}
\DeclareMathOperator{\trcl}{trcl} % 下部構造
% 0項演算子
\newcommand{\univ}{\mathrm{V}} % 普遍クラス
\newcommand{\img}{{\mspace{1mu}{``}\mspace{1mu}}} % 関係の像
\newcommand{\fap}{{\mspace{1mu}{`}\mspace{1mu}}} % 関数適用
\newcommand{\resl}{\mathbin{\mid}} % 関係の積
\newcommand{\uphl}{\mathbin{\upharpoonleft}} % 左域の制限
\newcommand{\uphr}{\mathbin{\upharpoonright}} % 右域の制限
\newcommand{\exten}{\mathbin{\between}} % モデルに相対的な論理式の外延
% 1項演算子
\newcommand{\union}[1]{{\textstyle\bigcup}\mspace{3mu}#1} % クラス合併
\newcommand{\intersect}[1]{{\textstyle\bigcap}\mspace{3mu}#1} % クラス共通部分
\newcommand{\classab}[1]{\{\mspace{2mu}#1\mspace{2mu}\}} % クラス抽象
\newcommand{\barl}[1]{\bar{\:\:}#1} % 補クラス
\newcommand{\dotl}[1]{\dot{\:\:}#1} % 関係部分
\newcommand{\brevel}[1]{\breve{\:\:}#1} % 関係の逆
\newcommand{\tildel}[1]{\tilde{\:\:}#1} % 系列の反転
\newcommand{\ance}[1]{{^*}#1} % 祖先関係
\newcommand{\orp}[1]{\langle #1 \rangle}% 順序対
\newcommand{\kagi}[1]{「~#1~」}% 半角引用
\newcommand{\trgl}[1]{{^\triangleleft}#1} % 左域の唯一の要素
\newcommand{\indx}[1]{\mathfrak{h}#1} % 有意指標
\newcommand{\app}[1]{\mathrm{app}_{#1}} % 適用条件
\newcommand{\exe}[1]{\mathrm{exe}_{#1}} % 構成要件
\newcommand{\enf}[1]{\mathrm{enf}_{#1}} % 執行可能性
\newcommand{\cs}[1]{\mathrm{cs}_{#1}} % 制御構造
\newcommand{\cty}[1]{\mathrm{cty}_{#1}} % 制御可能性
\newcommand{\bkg}[1]{\mathbin{\vartriangleright_{#1}}} % 背景条件
% 2項演算子
\newcommand{\iter}[2]{#1^{\mid #2}} % 関係の反復
\newcommand{\timex}[2]{#1^{\times #2}} % 直積の反復
\newcommand{\prob}[2]{#1^{:#2}} % 蓋然性
\newcommand{\msec}[2]{#1^{\circledcirc #2}} % メンバーの系列要素
\newcommand{\mser}[2]{#1^{\otimes #2}} % メンバーの反転系列要素
% 真理関数の点
\newcommand{\md}[1]{{\;#1\;}}
\newcommand{\ld}[1]{{\;#1\!}}
\newcommand{\rd}[1]{{\!#1\;}}
\newcommand{\tdot}{{:\mspace{-3.5mu}.}}
\newcommand{\tdos}{{.\mspace{-3.5mu}:}}
\newcommand{\sdot}{{:\mspace{2mu}:}}
% 真理関数
\newcommand{\con}[1]{%
	\ifcase #1%
    \or \md{.} \or \md{:} \or \md{\tdot} \or \md{\sdot}
	\fi%
}%
\newcommand{\case}[3]{%
	\ifcase #2%
    \or \ld{.} \or \ld{:} \or \ld{\tdos} \or \ld{\sdot}
	\fi%
  \ifcase #1%
    \or \supset \or \lor \or \equiv \or 
  \fi%
  \ifcase #3%
    \or \rd{.} \or \rd{:} \or \rd{\tdot} \or \rd{\sdot}
  \fi%
}%

\newcommand{\fndp}[2]{foundation~\cite[\href{https://issasaek.github.io/foundation/foundation.pdf\#page=#1}{p.\,#2}]{foundation}}
\newcommand{\fndpp}[2]{foundation~\cite[\href{https://issasaek.github.io/foundation/foundation.pdf\#page=#1}{pp.\,#2}]{foundation}}

\begin{document}

% !TeX root = method.tex

\begin{textblock*}{0.1\linewidth}(510pt, 60pt)
    \small \version
\end{textblock*}

\title{Loredux-Method}
\author{佐伯 一冴\\\small\url{https://github.com/issasaek}}
\date{}
\maketitle

\begin{abstract}
    信念体系を最適化して,抑制条件が惹起する不快な感情を無害化するメソッドを構築する.メソッドの基本的な適用パターンは,不快な感情が生じたときに,①関連する信念体系の部分を特定化して,②信念体系を部分的に最適化して問題の部分を取り除く,というステップから成り立つ.信念体系の最適化は,それを論理的表記法で統制された言語に規格化する論理的還元の段階と,集合論への存在論的還元を通じて現実を再構築する段階に分けることができる.メソッドの継続的な適用は使用者を日常的意識から解放するだろう.
\end{abstract}

\section{序論}
\label{sec:序論}

本稿では,個人の信念体系を最適化するメソッドが扱われる.メソッドの目的はその使用者の幸福である.ただし,ここで言う幸福は特定の心理状態ではなく,メソッドの継続的な適用の結果到達できると仮定される全体的状態のことだ.このメソッドの構築と適用の両面において,foundation~\cite{foundation}で開発された理論的概念を使用する.そのため本稿はfoundation~\cite{foundation}の内容と記法を前提としている.参照は\kagi{\fndp{1}{1}}のようにページリンク付で行われる.

類似した機能を持つメソッドは,心理療法や自助テクニックから宗教的な教義に至るまで,既に古来から夥しい数が存在する.本稿が別のメソッドをわざわざ考案する理由は,単に,既存のメソッドの中に,
\begin{enumerate}[label=(\arabic*)]
    \item 環境依存性が最低限度に止まる.
    \item 日常的意識の根本的な改変を含む.
    \item 科学的世界観と衝突しない.
\end{enumerate}
という条件を全て充たすものを見つけられなかったからだ.(1)について,メソッドの機能を阻害するような使用者を取り巻く(変更可能性の低い)環境的条件の種類が多いほど,そのメソッドは環境依存的である.この環境依存性を低く保つハードルは高い.それでも本稿では,特定の人間関係や資源の状態を前提としないだけでなく,(牢獄に囚われたり,不治の病に伏せているといった)もっと制限された状況においても可及的に使用できるメソッドを志向している.
また(2)は,神経症的状態を解消してより適応的になるという通常の心理療法が持つ機能を含むとしても,さらにその先へと運んでくれるメソッドが欲しいということだ.信念体系の最適化には,日常的な言語的装置をそれとは異なるビジョンを喚起するように改造する工程が含まれている.宗教的なメソッドも結局はそのような志向を持っているだろう.しかし,それらの多くは(3)を充たさないことで除外される.我々の信念体系には,科学者共同体に受容された科学理論か,または常識化されたその劣化バージョンが部分的に含まれている.それらと矛盾するような信念を新たに組み入れることによって,体系を大きく変更したり,矛盾による混乱を放置するとしたら,それは端的にコストが高すぎるのだ.それを実行するには既にテクノロジーが浸透しすぎており,それゆえに宗教的メソッドが機能を果たせる時期は過ぎ去ってしまった.

以上が本稿のメソッドの基本的なコンセプトだ.メソッドの継続的な適用は「幸福」の語が無意味になるほど日常的意識を改変するかもしれない.その段階ではその語でメソッドの目的を表現することはいよいよ困難になるだろう.日常的意識は多数の言語的傾向性から成り立っている.それらはおそらく,石器時代を通じて当時の環境に対処するよう進化論的に獲得されてきた傾向性に根を持っており,我々は未だにそれに隷属しているのだ.メソッドは新しい言語的装置により信念体系を更新する.そして古い神々を殺すだろう.

\section{基本構造}
\label{sec:基本構造}

メソッドは主に不快な感情をトリガーとして使用する.そしてその感情の生起に関与している信念体系の部分を修正する.しかし,このことが何を意味するのかは説明を要する.それと同時に,メソッドの細部を構築する前にその全体像をここで提示しておこう.

\subsection{抑制条件}
\label{ssec:抑制条件}

ここで「不快な感情」と言われているのは,負の修正条件すなわち抑制条件(の危険)の実現によって引き起こされるような内的状態(情動)のことだ.抑制条件とは,ある条件(適用条件)下での行動によってそれが実現された場合に,それによって,同一行動主体のその条件下の同一行動が抑制されるような条件である.「抑制される」というのは,適用条件によってその行動を遮断するような制御回路(負の制御構造)が,その行動主体に構築される(蓋然性がある),ということを意味している\footnote{修正条件の概念は,学習理論における弁別学習に関する概念を改造したものである.例えば,レイノルズ~\cite[pp.\,8--12]{レイノルズ}を参照.なお,蘇った記憶のような内的な状態も修正条件となり得る.}.
例えば,ある人が受注を受けたプロダクトを期日までに納品できなかった結果,顧客か上司に叱責されるという場面を考えよう.この場合,統制に関与する適用条件としては,納品することによって顧客の重要な利益が侵害される等の特殊事情が存在しない,といった条件が考えられる.そのような条件下において,作為/不作為により期日において顧客がプロダクトを占有していない状態を惹起することが,顧客/上司の叱責を因果的に決定している.
このような因果連鎖の実現によって,以下の因果関係(制御構造)の蓋然性が惹起される,という因果関係が成立しているならば,顧客か上司の叱責は,当該適用条件と行動に対して抑制条件になっている.すなわち,別の顧客に関するものでよいが,
\[
   \text{上記の特殊事情が存在しない → 顧客のプロダクト占有を阻止する因果系列を遮断}
\]
という制御構造.
なお,抑制条件とは逆に,行動を促進する(適用条件によって行動が起動されるという正の制御構造を惹起する)条件は促進条件(促進条件)であるが,その実現が阻止されることも,ある状況と行動に対して抑制条件になり得る.

foundation~\cite{foundation}の記法によると,修正条件が特定の領域$x$と規制類型$y$,それゆえ適用条件$\app{\epsilon}y$や構成要件$\exe{\epsilon}y$に相対的であることが明確になる.すなわち,$ x\in\cty{\epsilon}y $について,$y\in\mathrm{Reg}$の制御方向が,
\[
    \tildel{\trgl{y}}\fap 1 = \barl{(\trgl{y}\fap 0)}\cap\msec{(\arg y)}{0}
\]
ならば($y$はS類型),修正条件$ \orp{\tilde{x}\fap 0,\tildel{\trgl{y}}\fap 0} $は抑制条件である(\fndp{23}{23}).
逆に,$ y $の制御方向が$\tildel{\trgl{y}}\fap 1 = \trgl{y}\fap 0 $ならば($y$はP類型),促進条件である.
いずれにせよ,$ \tilde{x}\fap 0\in\tildel{\trgl{y}}\fap 0 $であるとき修正条件は実現される.そして,行動主体は単に$\tilde{x}\fap 1$で特定される.

ところで,修正条件を因果的に惹起する条件も同じ状況と行動に対して修正条件になり得る\footnote{レイノルズ~\cite[p.\,12]{レイノルズ}.この因果連鎖は最終的に,他の修正条件によって因果的に決定されていない生得的な修正条件に行き着く.}.例えば,上司または顧客による状況把握が別の報告者の行動に依存しているならば,その報告者の報告や認知的状態もまた,同様の修正条件となり得る.
厳密化して言うと,$ a\in b $によって$ \tilde{x}\fap 0\in\tildel{\trgl{y}}\fap 0 $の蓋然性が因果的に惹起された場合,つまり,
\begin{gather}
    a\in b,\\
    (\exists n)[n\subseteq\indx{\epsilon}\con{1}
        \orp{a,b}\to_{\epsilon}\orp{\tilde{x}\fap0,\prob{(\tildel{\trgl{y}}\fap 0)}{(n\uphl\epsilon)}}
    ],
\end{gather}
である場合,$ \tilde{w}\fap 1 = \tilde{x}\fap 1 $である$w$と$z$について,$w\in\cty{\epsilon}z$になり得る.ただし,$y$と$z$は,それぞれの右域から修正領域に関する条件を除外した条件が同一であり,かつ,$ \trgl{z} $は$ \trgl{y} $の修正条件を$b$に交換した系列である.すなわち,
\[
    \mathrm{pur}\,y = \mathrm{pur}\,z\con{1}\tildel{(\tildel{\trgl{y}}\tbinom{0}{b})} = \trgl{z}.
\]
このようにして,$ b = \tildel{\trgl{z}}\fap 0 $について,$ \orp{\tilde{w}\fap 0,b} $が新たに修正条件となる.
(2)の\kagi{$ \indx{\epsilon} $}は有意な蓋然性レベルを表わしている.$ \beta\subseteq\trgl{\arg\epsilon} $である限り,$ \prob{\beta}{(1\uphl\epsilon)}=\beta $である.蓋然性の定式化は\fndp{17}{17}にある.

この修正条件の形成過程は,以下のN類型$v$の執行可能性によって$ w\in \cty{\epsilon}z $が構築される過程として再記述できる.つまり,
\[
    \arg\trgl{v} = 3\con{1}\tildel{\trgl{v}}\fap 1 = \Lambda\con{1}\trgl{v}\fap 0 = b\con{1}
    \tildel{\trgl{v}}\fap 0 = \tildel{\trgl{y}}\fap 0
\]
である規制類型$v$と,$ \tilde{u}\fap 0 = \tilde{x}\fap 0 $である$ u\in\app{\epsilon}v $について,
\[
   \orp{u,\exe{\epsilon}v\cap\enf{\epsilon}v}\to_{\epsilon}\orp{w,\cty{\epsilon}z}.
\]
この点,実際には$ u\notin \exe{\epsilon}v\cap\enf{\epsilon}v $であっても,それの言語的代替(実質的には(2)の言語的代替),すなわち$ u\in \app{\epsilon}v\cap\enf{\epsilon}v $であることを肯定する主体$ \tilde{x}\fap 1 $の言語的傾向性が存在するなら,同一の制御可能性が形成され得る.上記の例で言えば,行動主体が,上司または顧客による状況把握が報告者の行動に依存していると単に思い込んでいるに過ぎない場合であっても,報告者の行動は同様の抑制条件となり得る.

さて,本稿が「不快な感情」と言う語で表示しようとするのは,以上の意味での抑制条件か,またはその危険(蓋然性)の実現によって生じる行動主体の情動である.例によって,抑制条件またはその危険が実際にはなくても,それの言語的代替つまりそれを肯定する行動主体の言語的傾向性があれば,同様の情動が生じる.再び先述の例では,上司か顧客の叱責によって生じる情動的状態は,「恐怖」と呼ばれる状態か「怒り」と呼ばれる状態かもしれない.あるいは「落ち込み」や「憂鬱」と呼ばれる状態でもあり得る.そして,そのような叱責より時間的に前の段階において,叱責がそれに依存する現在の状況自体(単に行動主体が現在の時間的・空間的位置を占めることをも含む)が抑制条件として,同様の情動か「不安」と呼ばれるような別のタイプの情動も生じるだろう.どのタイプの情動が生じるかは,この例では行動主体の認知的状態と信念体系に依存していると思われる.生じる情動のタイプを厳密に区別することは必要ではないが,それに関与している信念体系の部分が異なることを通じて,メソッドの適用において違いが生じる可能性はある.

\subsection{信念体系}
\label{ssec:信念体系}

人$x$の時間$t$における信念体系は,$x$が$t$において肯定する文の集合である.文を肯定する制御構造について,\fndp{37}{37}では次のような規制類型$\eta$を想定した.
$\app{\epsilon}\eta$は文タイプの真偽が質問される状況を特定化し,$ \exe{\epsilon}\eta $はそこにおいてその文を肯定する因果系列を実現する.そして,$\eta$の修正条件はコミュニケーションの成立を示すような質問者の反応を特定化する.
型文字\kagi{$ \tau $}で,$e$が言語的タイプ$p$のトークンであるような$ \orp{e,p} $のクラス(を指示するクラス抽象体の位置)を表わすなら\footnote{
    $ \tau\img\univ\subseteq \mathcal{P}(\timex{\mathbb{R}}{4}) $,そして,任意の$ p\in \breve{\tau} $はある同一の言語に属する.すなわち,$p$は当該言語の原始記号の有限系列であり,原始記号自体はそのトークンの集合である.\fndp{36}{36}を参照.},
\begin{gather*}
    \arg \eta \subseteq\classab{e:(\exists k)(\exists p)(\tilde{e}\fap 2 = \orp{k,p}\con{1}k\subseteq \timex{\mathbb{R}}{4}\con{1}p\in\breve{\tau}\img\univ)},\\
    \tildel{\trgl{\eta}}\fap 2 = \classab{\orp{k,p}:p\in\breve{\tau}\img\univ\con{1}(\exists q)(\orp{k,q}\in\tau\con{1}\text{ $q$は$p$の肯定 })}.
\end{gather*}
$ e\in\arg \eta\con{1}\tilde{e}\fap 2 = \orp{k,p} $とすると,$ \tilde{e}\fap 2\in\tildel{\trgl{\eta}}\fap 2 $であるとき,$r$の肯定であるような言語的タイプのトークン$k$が生成される.例えば,質問「$\phi$ですか」に対する「$\phi$です」.ただし,\kagi{$ \phi $}に$p$を代入する.

すると,文$p$が時区間$t$において$x$の信念体系に属するのは,$p,x$を内部構造に含む$t$内の任意の準拠領域$e$について,$e\in\cs{\epsilon}\eta$であるとき,かつそのときに限られる\footnote{多少考えれば肯定するような文も信念体系に取り込むために,制御構造の(高度の)蓋然性$ \prob{(\cs{\epsilon}\eta)}{n\uphl\epsilon} $に拡張してもよい.ただし,$n\subseteq\indx{\epsilon}$である.}.実際に質問された状況には限られないから$ e\in\app{\epsilon}\eta $である必要はない.しかし,$\app{\epsilon}\eta$から質問に関する条件$ f $を除外した条件を充たさない$e$については,$\cs{\epsilon}\eta$が成立しない可能性がある.この点を踏まえると,
\[
    \tilde{e}\fap 1 = x\con{1}\mathcal{R}\fap(\tilde{e}\fap 2)= p\con{1}
    e\fap 0,\mathcal{L}\fap(\tilde{e}\fap 2)\subseteq t\con{1}e\in\intersect{\classab{z:\arg\eta\subseteq z\neq f}}
\]
である任意の$e$について$ e\in\cs{\epsilon}\eta $であるとき,かつそのときに限り,$p$は$t$において$x$の信念体系に属する.
なお,本心で肯定しているかどうか等の厄介事が当然生じるが,その手の変速事例は,$\eta$の適用条件において,どうにかして処理されていると想像しよう.
関連して,最適化前の信念体系は矛盾している可能性が否定できないが,$p$とその否定を文字通り両方肯定するような言語的傾向性は通常成立しない.

差し当たって,信念体系に属する文$p$は,その真理値が$p$のトークンが生成される状況に左右されない永久文\footnote{クワイン~\cite[p.\,325]{クワインd}.}である必要はない.
この点,「今」「ここ」「私」のような指標的表現は,それが出現する文の使用状況に応じて指示対象が変化する.したがって,指標的表現を含む文は,その真理値がトークン生成状況に相対的であるから永久文ではない.また,1個の名詞や形容詞,例えば「ウサギ」や「寒い」が文として使用される場合,その文は「それはウサギだ」,「今ここは寒い」等の代用とみなすことができる\footnote{クワイン~\cite[p.\,5]{クワインe}.}.
そして,非永久文は,その文のトークンを生成する領域と近接する領域について真または偽であるなら場面文である.他方,トークン生成領域を含むより時間的に長い領域について真または偽であるなら持続文である.例えば持続文「今日の朝刊が届いた」が真であれば,その真理値はその日の終わりまで持続する.

日常言語にはこのような非永久文が頻繁に出現するが,それの真理条件や含意関係を明確化する場合は,指標的表現の指示対象を補って永久化する必要が生じる.それに加えて,真理関数と量化の構成方法が明示化された\fndp{4}{4}の原初的言語やその拡張(\fndp{18}{18})の文へと規格化しなければならない.本稿のメソッドにもこの規格化の工程が含まれている.そこで,文の永久化の手続を一般的な形で述べよう.すなわち,
非永久文$p$について,それに出現する指標的表現を変項の系列$ v\fap 0,v\fap 1,\dots,v\fap n $に置き換えた結果を$q$とする.また,$p$のあるトークン生成において,変項の値(指標的表現の指示対象)がそれぞれ$ k\fap 0,k\fap 1,\dots,k\fap n $であるとする.すると,系列$k$に相対的な$p$の永久化は,形式
\begin{align*}
    (\exists \nu_0)(\exists \nu_1)\dots(\exists \nu_n)(\phi\con{1}\orp{\nu_0,\nu_1,\dots,\nu_n}=\alpha)
\end{align*}
への代入によって得ることができる.すなわち,\kagi{$ (\exists \nu_0)(\exists \nu_1)\dots(\exists \nu_n) $}には,変項$ v\fap 0,v\fap 1,\dots,v\fap n $の存在量化子を順番に代入し,\kagi{$ \nu_0,\nu_1,\dots,\nu_n $}には,$ v\fap 0,v\fap 1,\dots,v\fap n $を\kagi{$ , $}で区切って順次代入する.そして,\kagi{$ \phi $}に$q$を,\kagi{$ \alpha $}に$ k\fap 0,k\fap 1,\dots,k\fap n $からなる順序対を指示するクラス抽象体を代入する.

ところで,最適化前の信念体系に属する文$p$は,価値的・規範的判断のような非記述的な文であり得る.環境を予測し制御するテクノロジーを構成するかその基盤となる言語的装置に寄与しない文は,実在の構造を記述するものではない.一方,信念体系は文の真偽に関する質問刺激に対してそれを肯定する傾向性に基づいている.すると,非記述的な文が信念体系に含まれるのかという問題が生じるように見える.しかし,文が記述的かどうかは,以下のような真理述語の適用可能性とは関係がない.

すなわち,ある言語(の文集合)$\gamma$と,任意の$q\in\gamma$を指示する表現と$q$の翻訳を構成できる言語$\gamma'$を考える.
そして,型文字\kagi{$ T $}で,$\gamma$のメンバーに適用される原始的または複合的な$\gamma'$の述語$t$の位置を表わすものとしよう.すると,任意の閉鎖文$p\in \gamma$について,型式
\[
   (\exists y)(\alpha = y\con{1}Ty)\case{3}{0}{0}P
\]
の\kagi{$ \alpha $}に$p$を指示する表現を,\kagi{$ P $}に$p$の翻訳である$\gamma'$の文を代入した結果が($\gamma'$において)真であるなら,$t$は$\gamma$の真理述語である.
$\gamma$を内包的な文を含む日常言語の文の集合とするなら,価値的・規範的判断にもこのような真理述語を適用できる(真または偽である).したがって,それらも最適化前の信念体系に含まれ得る.

さらに,$ \gamma\subseteq\gamma' $であり,\kagi{$ P $}に$p$自身を代入した結果がすべて真であるなら,$t$は引用解除的な真理述語である.この$t$を「真である」とし,また引用符が使えると仮定する.するとこの場合,任意の閉鎖文$p\in\gamma$について,型式
\[
   \text{\kagi{$ P $}は真である}\case{3}{0}{0}P
\]
の\kagi{$ P $}に$p$を代入した結果は真である.\fndpp{18}{18--19}の言語$\mathfrak{L}$の標準的モデルに相対的な真理は,対象領域の限定により引用解除的ではない\footnote{その領域を指示するクラス抽象体に相対化(クワイン~\cite[p.\,217]{クワインa})されていない文は引用解除できない.}.

\subsection{最適化}
\label{ssec:最適化}

\subsubsection{信念体系の関与}
\label{sssec:信念体系の関与}

以上で「不快な感情」と「信念体系」の本稿での用法が規定された.次に,メソッドは不快な感情の生起に信念体系の部分が関与しているとの仮定に基づいている\footnote{
    認知行動療法に分類されるような心理療法(論理療法~\cite{論理療法}や認知療法~\cite{認知療法}等)が準拠する仮説は,このような仮定を実質的に含むと考えられる.それらの心理療法には,多かれ少なかれ信念体系を修正するプロセスが取り込まれている.
}.そのような関与のパターンについては,以下のように考えられる.

パターン1:\ref{ssec:抑制条件}で述べたように,$ x\in\cty{\epsilon}y $について,既存の修正条件$ \tilde{x}\fap 0\in\tildel{\trgl{y}}\fap 0 $を因果的に惹起する条件もまた修正条件となり得る.この形成過程はN類型$ v $について,
\[
   \orp{u,\exe{\epsilon}v\cap\enf{\epsilon}v}\to_{\epsilon}\orp{w,\cty{\epsilon}z},
\]
であることとして再記述された.そして,実際には$ u\notin \exe{\epsilon}v\cap\enf{\epsilon}v $であっても,それの言語的代替,すなわち$ u\in \app{\epsilon}v\cap\enf{\epsilon}v $であることを肯定する主体$ \tilde{x}\fap 1 $の言語的傾向性が存在するなら,同一の制御可能性が形成され得ることも指摘された.
この言語的代替の条件を言い換えると,$ u\in \app{\epsilon}v\cap\enf{\epsilon}v $であることを記述する文$p$が,その時点における$ \tilde{x}\fap 1 $の信念体系に属するならば,$ w\in\cty{\epsilon}z $となり得る.
ここで$p$が上の条件を記述するということは,
\[
    (\exists l)(\orp{l,\orp{u,\app{\epsilon}v\cap\enf{\epsilon}v}}\in\mathcal{K}^{(\breve{\epsilon}\fap 0)}\con{1}
    \text{ $p$は$l$に解釈される}
    )
\]
ということを意味する(\fndp{16}{16}).

パターン2:これも\ref{ssec:抑制条件}で指摘されたが,修正条件またはその危険が実際にはなくても,それの言語的代替によって同様の情動が生じ得る.すなわち上記の例では,ある$n\subseteq\indx{\epsilon}$が存在して,
\[
   \tilde{x}\fap 0\in\prob{(\tildel{\trgl{y}}\fap 0)}{(n\uphl\epsilon)}
\]
であることを記述する文$q$が,その時点における$ \tilde{x}\fap 1 $の信念体系に属するならば,同様の情動が生じ得る.

パターン3:信念体系の要素間の論理的連関その他によって,上記のパターンにおける文$p$や$q$を肯定する傾向性は,信念体系に属する$ r_0,r_1,\dots,r_n $またはそれらの連言$r$を肯定する傾向性に依存することがあり得る.すると,$p$や$q$を肯定する傾向性によって生じる情動は,結局,$r$を肯定する傾向性に依存していることになる.この意味で,信念体系が情動の生起に関与する仕方は全体論的であり,個別の文を肯定する傾向性が1個の情動に対応するわけではない.

パターン4:修正条件(の危険)またはその言語的代替によって生じる情動は,関連する価値的・規範的判断に依存しているか,または,それらの判断によって不必要に増幅されていることが多いと考えられる.
例えば,上司か顧客の叱責によって生じた憂鬱は,「期日までにプロダクトを納品すべきだったのにしなかった」という義務違反に関する規範的判断と,そこから引き出される「自分は非難に値する」とか「自分はダメだ」等の価値的判断に依存している可能性がある.
関連して,上司か顧客の機嫌を損ねたことによって(仕事の依頼が減る等の)さらなる抑制条件が生じる危険を肯定する傾向性は,現在の状態に不安を追加し,憂鬱状態をさらに増幅するだろう.これ自体はパターン1であるが,この因果関係の信念は,還元されない反事実的条件法演算子によってその危険測度を曖昧化したまま表現されるものかもしれない.さらに,将来の状況の評価について「酷いことになる」等の漠然とした価値的判断を行なうことによって,意味もなく不安と憂鬱が増幅されているかもしれない.
以上とは逆に,本人がむしろ「自分に過失はなく,叱責は不当である」との信念を持っていたら,憂鬱ではなく「怒り」と呼ばれるような状態が生じると考えられる.この信念には,上司か顧客は「叱責行動をすべきでないのにした」という規範的判断が含まれている.

なお,本稿で言う価値的・規範的判断とは,典型的には,記述的な文$p$に対して,「〜は良い」,「〜は悪い」等の価値演算子または「〜であるべき」,「〜であってはならない」等の規範演算子を結合した文$p'$を肯定する傾向性だ.それに加えて,それ自体は内包演算子を持たないが,$p'$を肯定する傾向性と相互依存関係にあり,かつ,$p\con{1}p'$に解釈可能な$p''$を肯定する傾向性も,価値的・規範的判断に含めるとしよう.例えば,「彼は無能だ」,「今日の夕日は美しい」などの文だ.もっとも,このような分析は実はどうでもよい.さらには価値演算子と規範演算子の種類やそれらの区別,したがって価値判断と規範的判断の区別も重要ではない.いずれにしてもこれらは最適化によって排除されるからだ.

\subsubsection{メソッド}
\label{sssec:メソッド}

信念体系が不快な感情に関与するパターン1〜4にはいずれも,そのパターンが実現した場合,特定の不快な情動的状態が信念体系の部分集合$g$に依存するか,または,$g$によってその情動的状態が増幅されるようなケースである.メソッドは信念体系を部分的に最適化して$g$を取り除き,この依存関係を切断する.それによって,不快な感情を無害化するのだ.
この点に関して,人$x$と時間$t$について,ある$ e\subseteq x\cap t $を考える.また,$ \alpha = \classab{e:e\text{ は情動的状態}} $と置く.$t$と同時かそれに先行する信念体系の部分的最適化によって,抑制条件の実現にもかかわらず,$ e\in \prob{\alpha}{(1\uphl\epsilon)} = \alpha $であることが因果的に阻止されているとしよう.それでも依然としてあるレベル$n$について,当該修正条件が$e\in \prob{\alpha}{(n\uphl\epsilon)}$であることを惹起しているかもしれない\footnote{あるいは,$ e\neq e' $について,$ e'\in\alpha $もまた不快な情動的状態であり,当該修正条件は$ e'\in\alpha $を惹起しているかもしれない.}.そして,$e\in \prob{\alpha}{(n\uphl\epsilon)}$であること自体が別の不快な情動的状態と言えるケースもある.しかし,メソッドの継続的な適用の結果,残存する情動は無害なレベル$i$に止まるようになることが想定されている.$i\leq n$であるかどうかはメソッドの継続適用による最適化の進行具合による.

ところで,「無害化する」という言い方に既に表われているが,メソッドは不快な感情が何らかの意味で有害であることを前提している.このことを明確化するのに新しい概念は必要ない.不快な感情が有害であることは単に次のことを意味する.すなわち,抑制条件によって惹起される情動的状態もまた,当該情動がその内部で生起している人間の何らかの行動に対して,それ自体として抑制条件になり得る.そして,この情動的状態がさらに他の抑制条件を惹起することがあり得る.例えば,不安や憂鬱と呼ばれる状態は,その主体を不活性化して,他の有益な(促進条件を実現する)行動を阻害することに因果的に寄与している可能性がある.

もっとも,不快な感情の無害化それ自体がメソッドの目的なのではない.メソッドによって不快な感情を無害化するように信念体系を改訂する理由は,それに関与しているであろう(最適化によって除去される)信念体系の部分が,あなたが本来持っている創造の力を阻害しているからだ.メソッドは信念体系の最適化を通じてその力を解放するのだ.したがって,メソッドによって感情をどうにかすることが重要なわけではない.メソッドが感情に注目する唯一の理由は,それ自体が抑制条件になるような感情が生起しているとき,普通は,信念体系の最適化が不足しているということだ.そして,感情の生起態様やその強さ・持続性は,最適化を要する信念体系の部分を特定化するサインとして重要になる.

以上を要約すると,メソッドの適用の図式的な記述が得られる.それは極めて単純だ.すなわち,不快な感情が生起したとき,
\begin{enumerate}[label=(\arabic*)]
    \item 関連する信念体系の部分$g$を仮説的に特定化する.
    \item 信念体系を部分的に最適化して$g$を取り除く.
\end{enumerate}
以上が本稿のメソッドである.不快な感情が生起していないときでも,最適化可能な信念体系の部分$g$を発見したら,(2)を適用できる.それでも第一次的に不快な感情をトリガーに置く理由は,それを契機とする方が最適化可能な部分を見つけやすいからだ.もっとも,「最適化」の具体的な方法が明らかにならない限り,この図式自体はたいして意味はない.最適化の過程において一応次の二段階を区別することができる.
\begin{enumerate}[label=\roman*.]
    \item 論理的還元
    \item 存在論的還元
\end{enumerate}
iは,信念体系の部分集合を,量化論理によって統制された言語の文集合と入れ替える(規格化する)工程であるが,この言語は要するに\fndpp{18}{18--19}の全体的言語$\mathfrak{L}$のことだ.
他方,iiは,信念体系の存在論を集合論的に確定して,現実を再構築する工程だ.これ以上の詳細は第\ref{sec:論理的還元}節及び第\ref{sec:存在論的還元}節で展開される.
実際のところiの工程はiiの工程を含む仕方で行われることも多いから,最適化を二段階に分けるのはあくまで便宜的なものだ.

メソッドは繰り返し適用されることが想定されている.一度処理したように思えた情動的状態は,何度も繰り返しやってくる.その度にメソッドの適用を繰り返すのだ.メソッドの継続的な適用は,やがてあなたの魂をこの現実という楔から解き放ち,制限のない自由な空間へと飛翔させる.その経験がこれまでとは別の現実を創造することを可能にするのだ.

% !TeX root = method.tex

\section{論理的還元}
\label{sec:論理的還元}

信念体系の最適化の第一段階は,量化論理の表記法によって統制された言語(標準言語)への規格化ないし論理的還元だ.
ここで想定されている作業は,間主観性を持たないものの,環境を予測し制御するテクノロジーを構成するかその基盤となる言語的装置を開発することと,本質的に類似する.
つまり,科学と工学における理論構築の少々劣化した主観的なバージョンだ.
したがって,それらの理論構築の際に追求される事柄がここでも追求される.それは予測と制御の効果を最大化するために,概念枠を単純化し明確化することだ.
この概念枠には,全ての科学理論に共有されるような一般的な言語的装置(表記法)も含まれる.それを開発するのが量化論理学だ\footnote{
    したがって,量化論理学は実在の最も一般的な特徴を描写する科学と言うことができる.クワイン~\cite[pp.\,268--269]{クワインd}.
}.

\subsection{標準言語}
\label{ssec:標準言語}

1個の標準言語は,形式化された文またはその図式である論理式のクラス$\mathrm{L}$(量化言語)の部分クラスだ.
量化言語の文法は,原子式の構成要素として,無限個の変項\kagi{$x$},\kagi{$x'$},\kagi{$x''$}などと,無限個の述語記号
\[
   \text{ 
    \kagi{$ (F,) $},\kagi{$ (F,)' $},\kagi{$ (F,)'' $},$\dots$,\kagi{$ (F,\!,) $},\kagi{$ (F,\!,)' $},\kagi{$ (F,\!,)'' $},$\dots$
    }
\]
を持つ.\kagi{$,$}の数が$n$,\kagi{$'$}の数が$i$であるとき,$n$項述語記号の$i$番目を意味する.実際上は,変項は\kagi{$x$},\kagi{$y$},\kagi{$z$}等々で,原始的述語以外の述語記号は\kagi{$F$},\kagi{$G$},\kagi{$H$}等々で代用される.
次に,量化言語の文法は,原子式から複合式を構成する手段として,否定と条件法の真理関数\kagi{$\neg$}及び\kagi{$\supset$}と,普遍量化子\kagi{$(x)$},\kagi{$(x')$},\kagi{$(x'')$}などを持つ.
そして論理式は,次の3つの規則により再帰的に記述される.
\begin{enumerate}[label=(\arabic*)]
    \item \kagi{$ R\alpha_1\dots\alpha_n $}の\kagi{$ R $}に$0\neq n$項述語記号の$i$番目を,\kagi{$ \alpha_1\dots\alpha_n $}に$n$個の変項を並べて代入した結果は,論理式である.
    \item \kagi{$\neg P$}と\kagi{$(P\supset Q)$}において,\kagi{$P$}と\kagi{$Q$}に任意の論理式を代入した結果は,論理式である.
    \item \kagi{$(\alpha)P$}において,\kagi{$\alpha$}に任意の変項を,\kagi{$P$}に任意の論理式を代入した結果は,論理式である.
\end{enumerate}
連言\kagi{$ \con{1} $},選言\kagi{$ \lor $},双条件法\kagi{$ \equiv $}等の他の真理関数は否定と条件法から,存在量化子\kagi{$ (\exists x) $}は普遍量化子から定義できる.否定と条件法及び普遍量化以外の組み合わせを原始的として,他をそこから定義することもできる.

次に,1個の標準言語は,当該言語の述語として使用する述語記号を指定することによって規定される.例えば$\mathfrak{L}$では,$2$項述語記号の$0$番目\kagi{$ (F,\!,) $}が,要素関係を表わす\kagi{$ \in $}の形式的な表現であるとみなされる.すなわち,
\begin{itemize}
    \item \kagi{$(\alpha\in\beta)$}の\kagi{$\alpha$}と\kagi{$\beta$}に任意の変項を代入した結果は,\\\hfill
    \kagi{$ (F,\!,)\alpha\beta $}に同一の代入をした結果を表わす.
\end{itemize}
この文脈的定義により,\kagi{$ (F,\!,) $}は,$\mathfrak{L}$における実質的な原始的述語として(実際上は\kagi{$ \in $}で代用して),使用される.
\fndp{4}{4}の原初的言語は,\kagi{$ \in $}を唯一の原始的述語として持つ標準言語である.他方,$\mathfrak{L}$は原初的言語に多数の述語を追加したものであり,科学とテクノロジーに寄与する文が実質的に全て含まれる程度に包括的な標準言語である.
そして,$\mathfrak{L}$の原始的述語以外の述語記号が出現しない論理式が$\mathfrak{L}$の文である.他方,文でない論理式は文の論理構造を表わす図式(量化図式)となる($\mathfrak{L}$の原始的述語でない述語記号は$\mathfrak{L}$の開放文の位置を表わす).

なお,標準言語は変項以外の単称名辞を持たない.ある対象$z$の名前は,記述型\kagi{$ (\imath x)Ax $}の\kagi{$ Ax $}に$ z $についてのみ真である述語を代入した結果によって代用できる.そして,
\kagi{$ F(\imath x)Ax $}という文脈は,同一性を表わす\kagi{$ = $}がその言語で定義可能であれば,
\[
    (\exists y)[Fy\con{1}(x)(Ax\case{3}{0}{1}x = y)]
\]
の省略形とみなすことができる\footnote{クワイン~\cite[p.\,250]{クワインb}}.

文であれ図式であれ論理式はモデルによって解釈できる.1個のモデルは,空でない集合$u$(対象領域)と述語記号に$u$上の$n$項関係を割り当てる解釈関数$r$の組$\orp{u,r}$だ.
そして,論理式のクラス$x$が論理式$y$を含意するのは,
\[
   (m)(s)[m\text{ はモデル}\con{1}s\text{ は$\mathcal{L}\fap m$上の対象列}\case{1}{1}{1}
        (l)(l\in x\case{1}{1}{1}\orp{s,m}\text{ は$l$を充足})\case{1}{0}{1}\orp{s,m}\text{ は$y$を充足}
   ]
\]
であるとき,かつそのときに限られる.このような$x,y$のクラス(論理的含意関係)を\kagi{$ \mathrm{imp} $}で表わすと,1個の論理式が含意する関係は$ \brevel{(\lambda_x\classab{x})}\resl \mathrm{imp} $である.また,妥当式(任意のモデルにおいて真となる論理式)は$ \brevel{\mathrm{imp}}\img\classab{\Lambda} $である(\fndp{15}{15}).論理的真理つまり論理的に真である文は,妥当式である標準言語の文である.この点に関連して,標準言語$l\subseteq\mathrm{L}$と$ \alpha\subseteq l $について,クラス$\classab{p:\orp{\alpha,p}\in\mathrm{imp}}\cap l$を$\alpha$を公理集合とする$l$の理論と言う.量化理論は$\Lambda$を公理集合とする$\mathrm{L}$の理論である.
妥当式の全てが定理となるという意味で完全である,さまざまな量化理論の演繹体系が知られている\footnote{
    クワイン~\cite[pp.\,171--225]{クワインb}.
}.

さて,最適化対象である信念体系$\mathfrak{B}$と,それが規格化される先の標準言語(の文の集合)$\mathfrak{L}$を考える.規格化は,信念体系の問題の部分$ b\subseteq\mathfrak{B} $を,ある$l\subseteq\mathfrak{L}$と入れ替えることだ.つまり$\mathfrak{B}$を
\[
    \mathfrak{B}' = (\mathfrak{B}\cap\bar{b})\cup l
\]
に更新するのだ\footnote{メソッドの適用において,$b$と$l$を完全に特定化する必要はない.}.ただし,$l$のメンバーは$ \mathfrak{B}\cap\bar{b} $のどのメンバーの解釈でもない.
他方,$l$のメンバーは$b$のメンバーの解釈ではあり得る.しかし,ここでの解釈関係は同義性の関係ではないし,可及的に用法を保存するような関係ですらない.理論構築としての規格化は概念分析ではないのだ\footnote{
    理論構築としての規格化は,真理条件的な意味の理論を適用するために日常言語の文の論理形式を与えるような規格化(デイヴィドソン~\cite[pp.\,18--19]{デイヴィドソン})とも異なる.この意味での規格化は概念分析でもなく,文法的な規格化である.
}.さらに,仮に$\mathfrak{B}$のメンバーを永久文に限ったとしても,通常,
\[
    k\subseteq \mathfrak{L}\times\mathfrak{B}
\]
なる全ての(適切な)解釈関係$ k $について,$ l\nsubseteq k\img\univ\con{1}b\nsubseteq \breve{k}\img\univ $.つまり,$b$のどのメンバーの解釈でもない$l$のメンバー$q$が存在するし,逆に,$l$のどのメンバーにも解釈されないような$b$のメンバー$p$も存在する.
最適化において捨てられる信念$p$の代わりに,その信念の解釈であるような文を信念体系に取り込む必要はないということだ.仮に$p$の適切な解釈が$\mathfrak{L}$の中にあったとしてもである.
逆に最適化によって,これまでの信念体系に対応物のない新規の信念$q$が追加されてもよい.

\subsection{外延性}
\label{ssec:外延性}

標準言語に共通する特徴であって,メソッドに関連する重要な性質が外延性だ.信念体系を$\mathfrak{L}$に規格化することは,それに外延性を付与することを意味する.
$ \mathfrak{L} $の外延性は,任意の$ p,q,r\in \mathfrak{L} $と,$ r $における$ p $の出現のいくつかを$ q $に置換した式$ r' $について,
\[
    p\text{ の外延} = q\text{ の外延}\case{1}{1}{1}r\text{ の真理値} = r'\text{ の真理値}
\]
であることだ.しかし,この定式化に残る不明瞭さを除去する必要がある.

この点,\fndp{15}{15}におけるクラス$ \delta\exten\beta $は,要はモデル$\delta$に相対的な論理式$\beta$の外延である.これを$\beta$が閉鎖論理式であるケースにも拡張して,
\[
    \delta\parallel \beta = (\delta\exten\beta)\cup
    \classab{s\fap 0:
    % \classab{\mathcal{O}\fap(s\resl v):
    \mathrm{var}\,\beta = \Lambda\con{1}
    % v = \classab{\orp{0,0}}\con{1}
    \orp{s,\beta}\in\mathrm{SR}^\delta
    }
\]
と置く.$ \beta $が閉鎖論理式であるときは($ \mathrm{var}\,\beta = \Lambda $),
\begin{gather*}
    \beta\in\mathrm{T}^\delta\case{1}{1}{1}\delta\parallel\beta = \mathcal{L}\fap \delta,\\
    \beta\notin\mathrm{T}^\delta\case{1}{1}{1}\delta\parallel\beta = \Lambda.
\end{gather*}
ところで,\fndp{19}{19}において,$\breve{\epsilon}\fap 0$は$\mathfrak{L}$の標準的モデルとして規定された.これを前提すれば,任意の$ p,q,r\in \mathfrak{L} $と,$ r $における$ p $の出現のいくつかを$ q $に置換した式$ r' $について,
\[
   (\breve{\epsilon}\fap 0)\parallel p = (\breve{\epsilon}\fap 0)\parallel q \case{1}{1}{2}r\in \mathrm{T}^{(\breve{\epsilon}\fap 0)}\case{3}{1}{1}r'\in\mathrm{T}^{(\breve{\epsilon}\fap 0)}.
\]
であることとして,$\mathfrak{L}$の外延性を説明できる.

しかし実際には,外延性を規定するのに外延を実体化する必要はない.さらに特定の言語やモデルにも依存せず,標準言語一般に適用可能な外延性の定式化が存在する.すなわち,型式
\[
   (x_1)\dots(x_n)(A\case{3}{0}{0}B)\case{1}{0}{1}C_A\case{3}{0}{0}C_B.
\]
の\kagi{$ A $},\kagi{$ B $}及び\kagi{$ C_A $}にそれぞれ任意の論理式$ p,q,r $を代入し,\kagi{$ C_B $}には,$ r $における$ p $の$0\leq n$箇所の出現を$ q $で置き換えた論理式$r'$を代入する.\kagi{$ (x_1)\dots(x_n) $}には,$ p,q $で自由出現し,かつ,$ r,r' $で束縛出現する変項のすべてを順次代入し,そのような変項がなければ削除する.以上の操作の結果は妥当式である\footnote{
    清水~\cite[p.\,84]{清水}.
}.この事実が標準言語の外延性である.

日常言語においては,したがって最適化前の信念体系においても,外延性が成り立たない文脈は広範囲に存在する.
例えば,\kagi{$ OP $}を,\kagi{$ P $}の位置に来る文に演算子「ということは事実であるべきだ」を結合して得られる文の型とする.今,$x$は人を殺したが,$y$は誰も殺していないとしよう.また,\kagi{$ \alpha $}と\kagi{$ \beta $}で,それぞれ$ x $と$ y $を指示する単称記述の位置を表わす.すると次の(1)(2)は共に真となる.
\setcounter{equation}{0}
\begin{gather}
    (\exists z)(z\text{ は人}\con{1}\text{$\alpha$は$z$を殺す}),\\
    \neg(\exists z)(z\text{ は人}\con{1}\text{$\beta$は$z$を殺す}).
\end{gather}
しかし,\kagi{$ OP $}の\kagi{$ P $}に(1)を代入した結果は偽であり,(2)を代入した結果は真である.

次の例として,\kagi{$ \text{□}P $}を,\kagi{$ P $}の位置に来る文に演算子「ということは必然的である」を結合して得られる文の型とする.\kagi{$ \neg(\text{□}\neg P) $}は「$P$は可能である」に相当する.そして,次の(3)(4)は共に真であるが,\kagi{$ \text{□}P $}の\kagi{$ P $}に(3)を代入した結果は真であり,(4)を代入した結果は偽である.
\begin{gather}
    1+2 = 3,\\
    (\exists x)(x\text{ はマッコウクジラ}\con{1}x\text{ はアルビノである}).
\end{gather}

次に,\kagi{$ P\text{□→}Q $}の\kagi{$ P $}と\kagi{$ Q $}の位置に来る文に反事実的条件法の演算子「もし仮に〜ならば…だろう」を結合して得られる文の型とする.
$x$は水中にも冷蔵庫内にもない砂糖の塊であり,\kagi{$ \alpha $}でそれを指示する単称記述の位置を表わす.すると,(1)(2)は共に偽である.
\begin{gather}
    \alpha\text{ は水に入れられる},\\
    \alpha\text{ は冷蔵庫に入れられる}.
\end{gather}
しかし,\kagi{$ P\text{□→}(\alpha\text{ は溶ける}) $}の\kagi{$ P $}に(5)を代入した結果は真であるが,(6)を代入した結果は偽である.

\ref{sssec:信念体系の関与}で信念体系の部分が不快な感情に関与する仕方をパターン化した.これらのパターンの事例において最も致命的なケースは,以上で述べたような内包演算子が作り出す非外延的な文によるものである.内包性こそが悪の根なのだ.\ref{ssec:因果と規範}以降で内包的な文の集合の理論的代替物が扱われる.しかし,内包演算子の分析がこれ以上に行われることはない.内包性の理論的代替物は元の文集合に対して互換性がないからだ.理論構築としての規格化は,概念分析や文法的な規格化とは異なり後方互換性の確保を要請しない.つまり,古い文集合の機能や関連する傾向性が,理論的代替物においても実現される必要はない.内包的な文は単に打ち捨てられる.もっとも,メソッドの適用が最大限に効果を発揮した場合でさえ,おそらく内包演算子の日常的な使用を止めることはできないだろう.それらはコミュニケーションにおける技術的な有効性を持つからだ.
メソッドが目指すものは,内包的な文の集合の理論的代替物を構築することよって,それが信念体系に与える影響を最小化することだ.

\subsection{因果と規範}
\label{ssec:因果と規範}

反事実的条件法の理論的代替物は,解釈空間に相対化された因果の概念である(\fndpp{11}{11--20}).因果関係$\mathcal{C}^\epsilon$は,$ a\in x $であることが$ b\in y $であることを($\epsilon$に相対的に)因果的に決定するような$\orp{x,y}$のクラスだ.\kagi{$ \epsilon $}は解釈空間を指示するクラス抽象体の位置を表わす型文字であり,解釈空間は文脈によって異なり得る.なお,\kagi{$ \orp{\orp{a,x},\orp{b,y}}\in\mathcal{C}^\epsilon $}は\kagi{$ \orp{a,x}\to_{\epsilon}\orp{b,y} $}とも書かれる.
他方,価値と規範に関する理論的代替物は,この因果の概念に基づく規制の概念だ(\fndpp{21}{21--29}).
準拠領域$x$について,規制類型$y$に関する規制が成立するのは,$y$の適用条件$ \app{\epsilon}y $を充たすような$x$について,$y$の執行可能性$\enf{\epsilon}y$と制御可能性$\cty{\epsilon}y$という2個の因果的構造が成立する場合だ.ただし,制御方向$\tildel{\trgl{y}}\fap 1 = \Lambda$のときは制御可能性は不要である.つまり,
\[
    x\in\app{\epsilon}y\cap \enf{\epsilon}y\con{2}\tildel{\trgl{y}}\fap 1 \neq \Lambda\case{1}{1}{1}x\in\cty{\epsilon}y
\]
であるとき,かつそのときに限り,$\orp{x,y}$は規制である.以上の説明は大雑把な要約だ.因果と規制の概念の完全な定式化は上記の参照先にある.

メソッドの適用は,ケースごとにこれらの理論的概念を使用して事態を再記述する\footnote{
    因果と規制の概念は解釈空間に相対化されているから,再記述に使用される文は,特定化されない解釈空間を表示する変項の自由出現を持つと考えられる.
}.そして,古い信念体系を更新するのだ.物理的システムに還元されない価値と規範という混乱した観念は捨て去られ,規制の因果的構造の記述によって代替される.因果関係と蓋然性は,解釈空間に明示的に相対化されることによって,それらの記述が何に依存しているのかが明確化される.因果と規制の概念の開発から始まり長い道程を経て,ようやく情報処理の擾乱は一掃され,信念体系は浄化されるのだ.
以下で適用例を示そう.

\subsubsection{未来を心配すること}
\label{sssec:未来を心配する}

未来の抑制条件の実現が現在の状況に依存している場合,その現在の状況も修正条件になる.それによって生じるのが「不安」と呼ばれるような感情だ.
この修正条件間の因果関係を直接経験しなくても,それの言語的代替によって現在の状況が修正条件になる.言語能力を持つ人間が不安を感じるには,未来を思い描くだけでよいのだ.
例えば,「今のこの収入だから,近い将来金欠になるだろう.」,「今一人だから今後も一人だ.」,「今の健康状態だと,まもなく生活に支障が出るだろう.」.
このような因果関係の記述を,解釈空間に相対化された因果関係の記述,つまり,
\[
   \orp{a,x}\to_{\epsilon}\orp{b,y}
\]
という形式に置き換える意義はいろいろとある.解釈空間が定めるモデル間の比較類似性という因果の判定基準が一応存在すること自体がそうだ.また,背景条件$ x\bkg{\epsilon}y $を考察して,$\orp{a,b}$がそれを充たすかどうか検討することもできる(\fndp{17}{17}).さらに,蓋然性が因果的に決定される事態として,$ y = \prob{z}{(n\uphl\epsilon)} $等と置く場合は,危険測度の値$n$に自覚的になれる.将来の漠然とした因果記述によって強い不安を覚えるケースでは,危険測度が曖昧化されることで,実質的にはそれが過大に見積もられていることが少なくないと思われる.

しかし,このようなケースでより致命的な要素は,将来の状況またはそれが依存していると記述される現在の状況の重要度の評価の方だ.最適化前はこの評価は漠然とした価値判断,「大変だ」,「とんでもないことになる」,「ヤバいことになる」等々になっているはずだ.この種の価値判断では,状況の記述とそれに対する自分の反応の仕方の記述が混淆されており,実際に起きていること以上の被害を錯覚させる.それが不安の持続性と耐性を高めてしまうのだ.この部分の最適化は因果関係の記述を精錬するだけではできない.\fndp{31}{31}の評価構造を使う.評価構造は規制集合の工学的妥当性その他の条件の重要度を判定する道具である.
ここでの問題は,S類型$y$と$ x\in\cty{\epsilon}y $について,$ e = \tilde{x}\fap 0\con{1}d = \tildel{\trgl{y}}\fap 0 $である場合における,抑制条件$\orp{e,d}$の(時区間$t$における)重要度の判定だ.ユーザー集合$\kappa$は信念体系の主体$ \tilde{x}\fap 1 $の単一クラスである.他方,抑制条件のクラスを扱うならば,
\[
    e \subseteq \classab{\orp{\tilde{x}\fap 0,\tildel{\trgl{y}}\fap 0}:x\in\cty{\epsilon}y\con{1}\tilde{x}\fap 1 \in\kappa}\con{1}
    d \subseteq \mathcal{P}(\mathfrak{E}),
\]
としてもよい.いずれにせよ,$ e\in d $であることが因果的に決定する$ \kappa $メンバーの促進条件のクラス$\zeta_1$を次のように規定する.
\begin{multline*}
    \zeta_1 = \classab{\orp{\tilde{a}\fap 0,\tildel{\trgl{b}}\fap 0}:
    \orp{e,d}\to_{\epsilon}\orp{\tilde{a}\fap 0,\tildel{\trgl{b}}\fap 0}\con{1}a\in\cty{\epsilon}b\con{1}
    a\fap 0\subseteq t\con{1}\tilde{a}\fap 1 \in \kappa\con{1}\\
    b\in\mathrm{Reg}\con{1}
    (\exists x)(\tildel{\trgl{b}}\fap 0 = x^{::\epsilon}\cap\mser{(\arg b)}{0})\con{1}
    \tildel{\trgl{b}}\fap 1 = \trgl{b}\fap 0
    }
\end{multline*}
同じく抑制条件のクラス$\zeta_2$は,上の等式の\kagi{$ \tildel{\trgl{b}}\fap 1 = \trgl{b}\fap 0 $}を\kagi{$ \tildel{\trgl{b}}\fap 1 = \barl{(\trgl{b}\fap 0)}\cap\msec{(\arg b)}{0} $}に変更すれば得られる.
なお,$\zeta_2$のメンバーは,それを惹起することによって$x\in\cty{\epsilon}y$となるような基礎抑制条件とは限らない.つまり,次のような$b$の抑制条件とは限らない.
\[
    \mathrm{pur}\,b = \mathrm{pur}\,y\con{1}\tildel{(\tildel{\trgl{b}}\tbinom{0}{(\tildel{\trgl{y}}\fap 0)})} = \trgl{y}.
\]
次に,評価関数$\gamma$の独立変項は,手続上包括的に評価される$\zeta_1$の部分クラス,または,同じく$\zeta_2$の部分クラスである\footnote{
    修正条件$ \orp{e,d} $の単純な汎用性は,それが促進条件か抑制条件によって,$\gamma\img\mathcal{P}(\zeta_1)$または$\gamma\img\mathcal{P}(\zeta_2)$で測ることができる.一見すると,$\orp{e,d}$が修正条件となる他の規制類型のクラスも関係するように思われるが,他の修正条件を惹起するということは,その修正条件を持つ規制類型と類似する規制類型に対して修正条件になるということを意味するから,当該クラスを独立に考慮する必要はない.
}.つまり,
\[
    (w)(w\in\breve{\gamma}\img\univ\case{1}{1}{2}w\subseteq\zeta_1\case{2}{1}{1}w\subseteq\zeta_2).
\]
さらにここでの評価関数に固有の要素として,$ \gamma $の独立変項は,機能的に同一の生得的な促進条件のクラスであるか,機能的に同一の生得的な抑制条件のクラスである.生得的修正条件は,通常,他の非生得的な修正条件の危険惹起を媒介して危険惹起される.他方,$\gamma$の値は,当該規格で許容される媒介関数(\fndp{32}{32})が独立変項のメンバーに与える最大値と考えられる.そして,
\[
    (\gamma\img\mathcal{P}(\zeta_1)\text{ のメンバーの総和}) - (\gamma\img\mathcal{P}(\zeta_2)\text{ のメンバーの総和})
\]
の値が小さい程,当該抑制条件の重要度(要するに有害度)は高い(評価対象が促進条件の場合は逆である).
こうして,現在または未来の抑制条件に関する価値判断は,評価構造の記述に置き換えられる.結局,実際に起きていることは,様々な抑制条件の実現とそれの因果的関係でしかない.後は評価構造によるそれらの分類の仕方があるだけだ.それを超えて状況を価値的に表現することは,端的に誤った情報処理なのだ.

\subsubsection{過去に囚われること}
\label{sssec:過去に囚われること}

\ref{sssec:未来を心配する}とは逆に,現在の状況が過去に実現された抑制条件に依存しているパターンもある.現在の状況はそれ自体として抑制条件であるが,それを基礎づける制御可能性が過去の抑制条件に依存するわけではない.単に過去の抑制条件の実現が,現在の他の抑制条件の実現を因果的に決定しているということだ.
例えば,家族と死別するとか恋人と別れる等の喪失によって,それがなければ現在実現していたであろう促進条件が阻止されている.そして,促進条件の阻止が抑制条件となって落ち込みや憂鬱等の情動を生じさせるのだ.また,同じ状況が将来も継続すること(現在の状況によって今後も類似の状況が生じるという因果関係)を肯定する言語的傾向性によって,\ref{sssec:未来を心配する}のパターンにより,不安などが抱き合わせで生じることもよくある.

いずれにしても,過去・現在・未来の状況との間の因果関係の記述を最適化することが必要だ.しかし,\ref{sssec:未来を心配する}のケースと同様に,このケースにおいて最も致命的なのは,過去の喪失やそれに依存する現在の状況に対して,「もううんざりだ」,「もうやってられない」,「辛すぎる」などと心の内外で叫ぶことだ.この部分の最適化すなわち漠然とした価値的判断を評価構造に置き換えることが,結局最も重要になる.事実起きていることは,評価構造が分類するような仕方で抑制条件の因果的ネットワークが周囲に渦巻いているということ,単にそれだけのことなのだ.そこに価値的判断を上乗せすることは情報処理を混乱させる以外の意味を持たない.つまり全く無意味ということだ.

なお関連する事例として,過去の出来事によって生じるジレンマ状況について考察しよう.S類型$ y,y' $について,$y$の制御構造の実現によって$y'$の構成要件の実現が因果的に決定されているケースだ.つまり,
\setcounter{equation}{0}
\begin{gather}
    \orp{x,y},\orp{x',y'}\in\mathrm{REG}^\epsilon\con{1}x\in\cs{\epsilon}y\con{1}x'\in \exe{\epsilon}y',\\
    \orp{x,\app{\epsilon}y\cap\cs{\epsilon}y}\to_{\epsilon}\orp{x',\exe{\epsilon}y'}.
\end{gather}
典型的には,
\[
    (\exists z)(\exists w)[
        \orp{\orp{z,w},\orp{x,y}}\in\mathrm{IS}\con{1}
        \tildel{(x')}\uphr\bar{1}=\tilde{z}\uphr\bar{1}\con{1}\tildel{\trgl{(y')}}\uphr\bar{1}=\tildel{\trgl{w}}\uphr\bar{1}
    ].
\]
であり,かつ,$ x\in \exe{\epsilon}y $であることが作為行動であるケース.すると上記の遮断関係(\fndp{27}{27})により,$ x'\in \exe{\epsilon}y' $であることはそれに対応する不作為行動になる.
(1)(2)を充たす$ \orp{x,y},\orp{x',y'} $が仮に同一の法体系に含まれる法規制であるなら,一方が他方の機能を阻害しているから工学的妥当性に問題を生じるだろう.しかし,個人レベルの規制ではジレンマは頻発する.例えば,別れた恋人に連絡すれば復縁できるだろう状況において,あえて連絡しない不作為により分離状態が維持されているケース.この分離状態それ自体は抑制条件であり,それによって悲しみ等の情動も生じる.しかし,既に別の恋人がいるとかその他の事情を適用条件として,別れた恋人に連絡すると別のトラブルが発生する執行可能性が成り立つ.その結果,連絡する作為が遮断される.

このようなジレンマ状況においては自ら抑制条件を維持していると言えるが,それは抑制条件を阻止する行動が別の制御構造により遮断される結果にすぎない.当該抑制条件が抑制的であることに変わりはない.すなわち,
$ x'\notin\cs{\epsilon}y' $であるが,執行随伴性$ x'\in\exe{\epsilon}y'\cap\enf{\epsilon}y' $によって,ある$ x'' $について$ \cs{\epsilon}y' $の蓋然性が実現する限り,$ x'\in\cty{\epsilon}y' $.もちろん$ x' = x'' $でもよい.それゆえ,最適化前の信念体系の状態によっては,状況全体に対する悲観的な価値判断等によって,不必要に強い情動が生じ得る.メソッドによる最適化の必要性と可能性については,ジレンマ状況でもそれ以外の状況でも同じことだ.

\subsubsection{自己非難と他者非難}
\label{sssec:自己非難と他者非難}

\ref{sssec:未来を心配する}と\ref{sssec:過去に囚われること}で処理されたのは,特に限定のない何らかの状況が抑制条件となって情動を生成するケースだった.他方,ここで扱われるのは,他人や自分の行動またはその結果が抑制条件となって情動を生成するケースである.
他人や自分の行動それ自体は物理的な因果的構造だ.しかし,それに対する規範的判断が同期することによって,他人の行動の場合は「怒り」と呼ばれる情動が,自分の行動の場合は「落ち込み」とか呼ばれる情動が,(不適応な結果を招く程に)大きく増幅させられる.
規範的判断は主として,行動記述$p$に内包的演算子「〜であるべき」や「〜であってはならない」等を結合して作られる規範文を肯定する傾向性だ.$p$が行動記述でない拡張的な規範的判断も存在する.「世界はこうあるべきでない」,「世の中不公平だ」等々と類似のセリフがそれだ.規範的判断のクラスは最適化前の信念体系に深く食い入り我々を蝕んでいる.このような規範的判断のクラスを最適化することは,規制の概念を使用して規範的システムを再記述することを意味する(\ref{ssec:因果と規範}).言い換えれば,規範的判断を,規制の構成要件該当性と執行可能性及び制御可能性の記述に置き換えることだ.

この点について,既存の規範的判断の理論的代替物は,主として,倫理的または道徳的と分類される規制(倫理規制)の記述である.倫理規制は以下のような特徴を持つ.
まず,倫理規制の修正条件は,その構成要件が帰属される主体が属するコミュニティのメンバーによる非難/賞賛行動の結果,またはその集積と考えられる.
すなわち,コミュニティ$\kappa$の倫理的なS規制$ \orp{x,y} $について,
\begin{align*}
    \tilde{x}\fap 0 \subseteq\classab{\orp{\tilde{z}\fap 2,\tildel{\trgl{w}}\fap 2}:
        \tilde{z}\fap 1\in\kappa\con{1}z\in\app{\epsilon}w\con{1}w\text{ は非難行動のP類型}
    }\con{1}\tildel{\trgl{y}}\fap 0 \subseteq\mathcal{P}(\mathfrak{E}).
\end{align*}
他方,倫理的なP規制の場合は,上記の「非難」を「賞賛」に変更する.
次に,倫理規制では認定手続と執行手続は未分化であり,それらを合体させたプロセスが各実行規制(\fndp{35}{35})の結節点の内部で進行する.
この点,$\orp{x,y}$の修正条件の実行規制は,
\[
   v \subseteq \classab{\orp{z,w}:\orp{\tilde{z}\fap 2,\tildel{\trgl{w}}\fap 2}\in \tilde{x}\fap 0\con{1}
   \tilde{z}\fap 1\in\kappa\con{1}z\in\app{\epsilon}w\con{1}w\text{ は非難行動のP類型}}
\]
なる$ v $のメンバーであり\footnote{
    $v$のメンバーは,既に述べた非難または賞賛する行動のP規制であるが,その適用条件自体は希薄である.
},$ v\subseteq \classab{\orp{z,w}:z\in\exe{\epsilon}w} $であることによって,$ \orp{x,y} $の修正条件が直接実現される.
そして,倫理規制の構成要件実現に対して,それを非難/賞賛する人($z\in v\img\univ$の結節点$\tilde{z}\fap 1$)の内部で,構成要件該当性$x\in \exe{\epsilon}y$や制御可能性$ x\in\cty{\epsilon}y $の認定が行われる.そのプロセス(倫理手続)がなければ実行規制の執行可能性(例えば,違反者を攻撃→自己満足)は生じない.

また,倫理手続には正規性の認定も含まれ,執行可能性$ x\in\enf{\epsilon}y $の認定はこれで代替される.倫理規制の正規集合は単純であり制定階層は普通1段階しかない.すなわち,倫理規制は,その執行可能性がある評価構造による工学的妥当性を充たすという事実によって制定される.そして,その執行可能性が同じ評価構造による工学的妥当性を充たさないという事実によって廃止される.制定規制も廃止規制もN規制であり,この2種類のN規制を始祖として,あるコミュニティの倫理的な正規集合または規制体系が作られる.なお,ここでの評価構造は,既に述べたコミュニティ$\kappa$をユーザー集合とする.この評価構造による工学的妥当性が実際に成立していることによって,(証拠が生成され)倫理手続でそれを認定する制御構造が構築される.その結果,倫理規制の執行可能性が実現される\footnote{倫理学は評価構造を開発してそれを宣伝することによって,このプロセスに介入しようとする.}.
この点,他者や自己を義務や禁止に違反することで非難する場合,そのことを「正当化」する価値判断が同時に存在することが多い.倫理規制の正規性つまり評価構造による工学的妥当性の記述は,この規範的正当化の理論的代替物である.

さて,規範的判断に関連する信念体系の最適化によって気付かされることの1つは,我々が倫理規制の執行機関として自分や他人を非難する場合,その殆どの場合に正規性の認定を誤っているということだ.つまり,恣意的な評価構造を用いる場合は別として,殆どの場合に非難の執行可能性は工学的妥当性を持たない.誰かを咎める意味はないのだ.
我々は既存の意味での規範と価値を喪失する.規範は物理的システムの中に消え去り,また,あらゆるものが(その値がゼロであれ何であれ)等しい価値を持つようになる.それゆえ,賞賛に値するものも非難に値するものもない.
% !TeX root = method.tex

\section{存在論的還元}
\label{sec:存在論的還元}

信念体系の最適化の第二段階は,第一段階である論理的還元(規格化)を通過した信念体系$\mathfrak{B}'$を,その存在論を確定する理論$\mathrm{ZFC}$を受容している信念体系$\mathfrak{B}''$へと改造する工程である.存在論は存在文のクラスであり,$\mathfrak{B}'$の存在論は$\mathfrak{B}''$のそれの部分クラスである.また,$\mathrm{ZFC}$は\fndp{9}{9}で提示した標準的な集合論の体系だ.

ところで,(人間の)信念体系$b$と理論$u$について,$u\subseteq b$であることは普通はない.公理の全ての帰結を人間が把握することはできないからだ.また,理論の公理は,公準として枚挙的に指定されるか,公理型式の代入例が公理であるという仕方で指定される.すると,$u$の公理集合$u'\subseteq u$と,$u$の公準の集合$u''\subseteq u'$について,$u' = u''$でない限り,$u'\subseteq b$であることも普通はない.公理型式の全ての代入例を人間が把握することはできないからだ.
そこで,$b$が$u$を受容している事態を次のように捉えることにしよう.すなわち,$ \orp{p,h} $に対して,文$p$が文型$h$の代入例であることを記述する文を与える関数を$\phi$とすると,任意の$p\in u$と,$u$の公準の集合$ u''$について,
\begin{align*}
    p\in u''\case{2}{1}{0}(\exists h)(h\text{ は$u$の公理型式}\con{1}\phi\fap\orp{p,h}\in b)\case{1}{1}{1}p\in b
\end{align*}
であるとき,かつそのときに限り,$b$は$u$を受容している.

次に,$\mathrm{ZFC}$が$\mathfrak{B}''$の存在論を確定している,と言うことの意味は\ref{ssec:存在論}で明らかになる.

\subsection{存在論}
\label{ssec:存在論}

一般に,標準言語$l$と$t\subseteq l$について,$t$のメンバーである存在文のクラスを$t$の存在論と言う.ここで存在文とは,\kagi{$ (\exists x)Fx $}の代入例となる標準言語の文だ.また,$t$は何らかの$l$の理論であるか,ある時点のある人の信念体系である.
以下$l = \mathfrak{L}$と想定しよう.$\mathfrak{L}$においては,$2$項述語記号の$0$番目が要素関係を表わす\kagi{$ \in $}に相当する原始的述語である(\ref{ssec:標準言語}).$u\subseteq l$が何らかのクラス理論であるとき,$u$の公理はこの\kagi{$ \in $}の用法を規定している.そして,$u$が$t$の存在論を確定しているというのは,以下の条件を充たすことを意味している.
\begin{enumerate}
    \item $u$は$t$が理論ならその部分クラスであり,$t$が信念体系ならそれに受容されている理論である.
    \item $t$の任意の原始的述語$f$について,$f$が$n$項述語記号の$i$番目であり,かつ,$\orp{n,i}\neq\orp{2,0}$であるとする.このとき,下記(1)(2)の型式に後述の代入を行った結果のいずれかが$t$のメンバーである.また,$t$が理論ならば当該代入結果のいずれかが$t$の公理である.さらに,(3)に後述の代入を行った結果が$u$の定理である.
    \setcounter{equation}{0}
    \begin{align}
        \classab{\orp{\alpha_1,\dots,\alpha_n}:R\alpha_1\dots\alpha_n} = \beta.\\
        \classab{\orp{\alpha_1,\dots,\alpha_n}:R\alpha_1\dots\alpha_n} \subseteq \beta.\\
        (\exists x)(x = \beta).
    \end{align}
    代入の仕方は次の通り.\kagi{$ R $}に$f$を,\kagi{$ \alpha_1\dots\alpha_n $}に$n$個の変項を並べて代入する.
    \kagi{$\orp{\alpha_1,\dots,\alpha_n}$}には,同じ$n$個の変項から作られる順序対表現を代入する.\kagi{$ \beta $}には$l$の文から作られるクラス抽象体を代入する.
\end{enumerate}

条件2の(1)(2)への代入結果を「還元規則」と言う.それは原始的述語の外延を規定することによって$t$の存在論を明確化する.
例えば,理論$t$のメンバーに\kagi{$ x\text{ は電子である} $}に相当する文が出現し,かつ,それが(原始的述語に変項を結合した)原子文であるとしよう.
すると,もしクラス理論$u\subseteq t$が$t$の存在論を確定しているならば,
\[
   \classab{x:x\text{ は電子である}}\subseteq \mathcal{P}(\timex{\mathbb{R}}{4})
\]
のような対応する還元規則が$t$のメンバーになっており,かつ,
\[
   (\exists x)(x = \mathcal{P}(\timex{\mathbb{R}}{4}))
\]
が$u$で証明可能である.この点,$\mathrm{ZFC}$から置換公理を除いた$\mathrm{ZC}$について,$\mathrm{ZC}\subseteq u$ならば実際に証明可能であり,さらに,上記の還元規則と分出公理(\ref{ssec:集合論の体系}のA \ref{axim:分出})によって,
\[
   (\exists y)(y = \classab{x:x\text{ は電子である}})
\]
が$t$で証明可能になる.他の原始的述語についても同様である\footnote{
    原始的でない複合的な述語に関する還元規則についても,これに出現する原始的述語の還元規則に基づいて,$t$で証明可能になり得る.例えば,$t$のメンバーに「$x$はマイナスの電荷を持つ素粒子である」に相当する複合文が出現する場合,これの還元規則と外延の存在が$t$で証明可能であり得る.
}.

なお,$t$の存在論が確定されているとき,それはクラス理論を($t$が信念体系なら少なくともその一部を)含んでいるから,量化の変項の値に制限はないとみなされる.この点,クラス理論に属しない文($t$が理論なら集合論以外の公理など)において,変項の値の範囲を実質的に制限したい場合は,\kagi{$ (x)(x\in\alpha\case{1}{1}{0}Fx) $}という明示的な形式を使うことができる.この形式では,束縛変項\kagi{$ x $}の値域が,\kagi{$ \alpha $}に代入される抽象体が指示するクラスへと実質的に制限される.

ところで,\fndpp{18}{18--20}において,真理集合$\mathrm{T}^{(\breve{\epsilon}\fap 0)}\cap\mathfrak{L}$の部分クラスとして,標準的な科学とテクノロジーに関連する理論を実質的に全て含むような包括的な$\mathfrak{L}$の理論$\mathfrak{K}$が想定されている.
この$\mathfrak{K}$の存在論は$\mathrm{ZC}$によって確定されると考えられる.つまり,$\mathrm{ZC}\subseteq\mathfrak{K}$であり\footnote{正確には,原初的言語ではなく$\mathfrak{L}$に合わせて$\mathrm{ZC}$を再解釈したものが$\mathfrak{K}$に含まれる.},また,言語$\mathfrak{L}$の構成上,同一性述語と数学的述語は原始的述語ではなく,定義によって導入される.したがって,同一性の法則や数学的法則は$\mathrm{ZC}$の定理として,$\mathfrak{K}$に取り込まれる.
この点について,第一段階を通過した信念体系$ \mathfrak{B}' $は,理想的には,$\mathfrak{B}'\subseteq\mathfrak{L}$であり,かつ,$\mathfrak{K}$を部分的に含んでいる.例えば,$ \orp{a,b}\in x\bkg{\epsilon}y $であることを記述する文が$ \mathfrak{B}' $のメンバーであるとき,
\[
    x\uphl(x\bkg{\epsilon}y)\subseteq \univ\times y
\]
であることを記述する$\mathfrak{K}$に属する一般化が,$ \mathfrak{B}' $のメンバーになるかもしれない.しかし,背景条件に言及する形ではなく,明示的に解釈空間に相対化された因果記述も$\mathfrak{B}'$に含まれ得る.つまり,
\[
    \orp{\alpha,\beta}\to_{\epsilon}\orp{\gamma,\delta}
\]
の型文字にそれぞれ$a,x,b,y$を指示する抽象体を代入した結果.\kagi{$ \epsilon $}は型文字のままか変項$v$に置換されると考えられる.そして,\kagi{$ \epsilon $}の位置に来る抽象体が指示するか,または,$v$の値となる解釈空間は,\fndp{18}{18}の条件1〜4を充たすものと($\mathfrak{B}'$において)想定されている.したがって,$\mathfrak{B}'$には,
\setcounter{equation}{0}
\begin{align}
    (\exists x)(\exists n)[\mathrm{sim}\,x\con{1}2\in n\in\mathbb{N}\con{1}\Lambda\neq \trgl{\arg x} = \mathcal{W}\fap n]
\end{align}
に相当する文や,それが含意する
\begin{align}
    (\exists z)(\exists n)(z = \mathcal{W}\fap n\neq\Lambda\con{1}2\in n \in\mathbb{N})
\end{align}
に相当する文($\mathrm{ZC}$のモデルの対象領域の存在を肯定する)が含まれ得る.また$\mathfrak{B}'$には,解釈空間に相対化された因果の概念だけでなく,それに基づく規制の概念も取り込まれている.すると規制類型の階層構造によって(\fndp{44}{44}),次の型式の\kagi{$ \gamma $}に$2$より大きい数を指示する抽象体を代入した結果も,$\mathfrak{B}'$に含まれ得る.
\begin{align}
    (\exists z)(z = \mathcal{W}\fap \gamma\neq\Lambda).
\end{align}
(1)(2)(3)は$\mathrm{T}^{(\breve{\epsilon}\fap 0)}\cap\mathfrak{L}$のメンバーではなく,それゆえに$\mathfrak{K}$や$\mathrm{ZC}$のメンバーでもない\footnote{
    もし$\mathfrak{K}$のメンバーならば,$\mathfrak{K}$のモデルの存在,つまり$\mathfrak{K}$の無矛盾性が$\mathfrak{K}$で証明可能になるだろう.しかしそれはあり得ない.$\mathrm{ZC}$についても同様である.清水~\cite[p.\,131,169]{清水}.
}.
これらに類する存在文を体系的に供給するには,$\mathrm{ZFC}$が必要である.$\mathfrak{B}'$から$\mathfrak{B}''$への移行は,還元規則と$\mathrm{ZFC}$の公理(の一部)を$\mathfrak{B}'$に追加することによって行われる.

\subsection{集合論の体系}
\label{ssec:集合論の体系}

\fndpp{4}{4--8}で導入された記法にによると,以下のA \ref{axim:外延性}〜A \ref{axim:選択}の(自由出現する変項をすべて普遍量化した)普遍閉鎖体が$\mathrm{ZFC}$の公理(型)である.A \ref{axim:分出}とA \ref{axim:置換}は公理の型であり,それぞれの\kagi{$ \alpha $}に$\mathfrak{L}$の開放文から作られるクラス抽象体を代入した結果が公理であることを表現している.他方,A \ref{axim:外延性}〜A \ref{axim:無限}及びA \ref{axim:正則性}とA \ref{axim:選択}は公準である.

\begin{axim}[外延性]
\label{axim:外延性}
$
    x=y\con{1}x\in z\case{1}{1}{1}y\in z.
$
\end{axim}

\begin{axim}[一対化,和,冪]
\label{axim:一対化,和,冪}
$
    \classab{x,y},\:\union{x},\:\mathcal{P}(x)\in\univ.
$
\end{axim}

\begin{axim}[無限]
\label{axim:無限}
$
    \mathbb{N}\in\univ.
$
\end{axim}

\begin{axim}[分出]
\label{axim:分出}
$
    x\cap\alpha\in\univ.
$
\end{axim}

\begin{axim}[置換]
\label{axim:置換}
$
    \func\alpha\case{1}{1}{1}\alpha\img x\in\univ.
$
\end{axim}

\begin{axim}[正則性]
\label{axim:正則性}
$
    x\neq \Lambda \ld{.}\supset (\exists y)(y\in x\md{.}x\cap y=\Lambda).
$
\end{axim}

\begin{axim}[選択]
\label{axim:選択}
$
    (\exists y)(
        y\subseteq\mathfrak{E}\con{1}x\cap\barl{\classab{\Lambda}}\subseteq\arg y
    ).
$
\end{axim}

\noindent 標準的な科学とテクノロジーが要求する数学は,\fndp{4}{4}の原初的言語に還元可能であり,その存在論は上記からA \ref{axim:置換}を除いた体系ZCによって供給できる.他方,物理理論その他の科学理論は原初的言語に還元可能ではないが,$\mathfrak{K}$の部分クラスであり,$\mathfrak{K}$の存在論はZCで供給可能である(\ref{sssec:物理的対象}).さらに因果と規制の概念は,解釈空間に相対化する形式ならば,原初的言語に還元可能である.しかし,それが要求する存在論(規制の概念の使用が想定される工学的理論の存在論)はZCでは供給できない.本稿で$\mathrm{ZFC}$が要請されたのはそのためだ.もっとも,$\mathrm{ZFC}$によって不可避的に,環境の予測と制御にとって必要な範囲を遥かに超えて,著しく存在論は拡張される.\ref{sssec:累積階層}でその全体像が確認される.

\subsubsection{物理的対象}
\label{sssec:物理的対象}

\fndp{6}{6}の定義によると,任意の$ x,y $について,$ x = y $であることは,
\[
    (z)(z\in x\case{3}{1}{1}z\in y)
\]
であることを意味する.すると,$ \neg(\exists w)(w\in x)\con{1}\neg(\exists w)(w\in y)\case{1}{1}{1}x = y $.すなわち,要素を持たない対象が存在するなら,それは1個しかない.他方A \ref{axim:分出}によって,$ \Lambda = x\cap\Lambda\in\univ $.すると,要素を持たない唯一の対象が存在して,それは空集合である.それゆえ,クラスではない個体が要素を持たないなら,そのようなものは存在しない.そこで,個体を非クラスではなく,$ x = \classab{x} $であるクラス$ x $として特徴づける流儀もある\footnote{
    クワイン~\cite[p.\,29]{クワインa}.
}.しかし上記の体系では,A \ref{axim:正則性}によって,$ (x)(x\notin x) $.したがって,この意味での個体も存在しない.
それで何も問題はない.個体を云々する唯一の動機は物理的対象の存在を供給することだが,そのために個体は必要ない.物理的対象は,非クラスでも自分自身を要素とするクラスでもない,純然たるクラスと結局のところ同一視できるからである.

すなわち,実数のクラス$\mathbb{R}$は$ \mathbb{R}\subseteq\mathcal{P}(\mathbb{N}) $となるように定義可能である\footnote{
    クワイン~\cite[pp.\,113--117]{クワインa}.
}.すると,個々の実数は$\mathbb{N}$の部分クラスとして,$\mathbb{R}$自体は$ \mathcal{P}(\mathrm{N}) $の部分クラスとして,A \ref{axim:分出}によって存在が供給される.また,\fndp{10}{10}のT 7により,$ (x)(y)(x\times y\in\univ) $.すると,$ \timex{\mathbb{R}}{4}\in\univ $.
$ \timex{\mathbb{R}}{4} $を四次元時空と同一視すれば,その要素を時空点,その任意の部分クラスを時空領域とみなすことができる.さらに進んで,物理的対象とそれが占有する時空領域との関係は1対1対応であるから,任意の時空領域を物理的対象と同一視する.すると,ある物理的対象が時空領域$y$を占有しているということは,単に$y$がある物理的状態にあることを意味するようになる.

このような同一視が言語$ \mathfrak{L} $で行われている場合,「物理的対象$x$は時空領域$y$を占有する」に相当する原始的述語は不要になる($\mathfrak{L}$の述語ではない).時空的占有関係は同一性に帰着するからである.
また,型式\kagi{$ (x)(Fx\case{1}{0}{1}x\subseteq \alpha) $}の\kagi{$ Fx $}に物理的対象(またはその有限クラス)を特徴づける文脈を,\kagi{$ \alpha $}に\kagi{$ \alpha\in\univ $}が$\mathrm{ZFC}$で証明可能なクラス抽象体を代入した結果のあるものが,$ \mathrm{T}^{(\breve{\epsilon}\fap 0)}\cap\mathfrak{L} $のメンバーであるとみなされる(架橋法則).
例えば,\kagi{$ Fx $}に\kagi{$ x\text{ は電子である} $}を,\kagi{$ \alpha $}に\kagi{$ \timex{\mathbb{R}}{4} $}を代入した結果.
他の例として,\kagi{$ Fx $}に
\[
    (\exists a)(\exists t)(x = \classab{a,t}\con{1}Gat)
\]
を,\kagi{$ \alpha $}に\kagi{$ \mathcal{P}(\timex{\mathbb{R}}{4}) $}を代入した結果.なお\kagi{$ Gat $}は,物理的対象$ a $が時間$ t $に一定の状態にあることを記述する文脈である.
ただし,後者の架橋法則については,次のようにして$ \mathfrak{L} $の原始的述語が調整されるかもしれない.\kagi{$ Gat $}が表わす文脈が2項の原始的述語$p$を持つ原子文だとすると,$p$を破棄して,代わりに$a$の$t$における時間的部分$a\cap t$について真である1項述語$q$を使用する.すなわち,
下記の\kagi{$ Gat $}に$p$を持つ原子文を,\kagi{$ Hx $}に$q$を持つ原子文を代入した結果が真であるなら,$p$を廃棄して,$q$を残す.
\[
    (a)(t)(Gat\case{3}{0}{0}(\exists x)(Hx\con{1}x = a\cap t)).
\]

なお,時間の概念については,$x,y\in\timex{\mathbb{R}}{4}$がある座標系$f$に相対的に同時であるということを,
\[
    \mathcal{L}\fap(f\fap x) = \mathcal{L}\fap(f\fap y)
\]
として,そこから適当に構成することができる(\fndp{38}{38}).座標系$f$は,$f\subseteq\timex{\mathbb{R}}{4}\times\timex{\mathbb{R}}{4}$なる1対1変換であり,物理理論に適合するものである.
他方,空間については,上記の\kagi{$ \mathcal{L} $}を\kagi{$ \mathcal{R} $}に交換すると,$f$に相対的な同位置性の概念が得られる.

\subsubsection{累積階層}
\label{sssec:累積階層}

置換公理(A \ref{axim:置換})によってクラスの累積的な階層構造が導入できるが,それは順序数の概念に基づいている\footnote{以下の順序数の定義は,クワイン~\cite[pp.\,127--144]{クワインa}による.}.順序数は,$ \union{x}\subseteq x $であり,かつ,$ \mathfrak{E}\uphr x $が整列順序であるような任意の$ x $だ.整列順序は基礎づけられた順序関係である.
この点,関係$\alpha$が基礎づけられているのは,$x$の任意のメンバーに対して,$x$のあるメンバーが再び$\alpha$を持つような$x\neq \Lambda$が存在しないならばである.すなわち,
\begin{df}
\label{df:基礎}
\kagi{$
    \mathrm{Fnd}\,\alpha
$}は\kagi{$
    (x)(x\subseteq\breve{\alpha}\img x\case{1}{1}{1}x = \Lambda)
$}を表わす.
\end{df}

\noindent 次に,順序関係は,連結的($\mathrm{Connex}\,\alpha$)であり,推移的($ \alpha\resl\alpha\subseteq\alpha $),かつ非反射的($\alpha\subseteq\bar{I}$)な$\alpha$である.

\begin{df}
\label{df:連結性}
\kagi{$
    \mathrm{Connex}\,\alpha
$}は\kagi{$
    (x)(y)(x,y\in(\alpha\cup\breve{\alpha})\img\univ\case{1}{1}{1}\orp{x,y}\in\alpha\cup\breve{\alpha}\cup I)
$}を表わす,
\end{df}

\begin{df}
\label{df:順序}
\kagi{$
    \mathrm{Ordg}\,\alpha
$}は\kagi{$
    \alpha\resl\alpha\subseteq\alpha\con{1}\alpha\subseteq\bar{I}\con{1}\mathrm{Connex}\,\alpha
$}を表わす.
\end{df}
\noindent 推移性と非反射性から,非対称性($\alpha\subseteq\barl{\breve{\alpha}}$)が結果する.つまり,
$ \alpha\resl\alpha\subseteq\alpha\subseteq\bar{I}\case{1}{1}{1}\alpha\subseteq\barl{\breve{\alpha}} $.そして,次の定義は順序数のクラスを導入する.
\begin{df}
\label{df:順序数}
\kagi{$
    \mathrm{NO}
$}は\kagi{$
    \classab{x:\union{x}\subseteq x\con{1}\mathrm{Fnd}\,\mathfrak{E}\uphr x\con{1}\mathrm{Ordg}\,\mathfrak{E}\uphr x}
$}を表わす\footnote{
    A \ref{axim:正則性}の正則性公理によると,そもそも$ \mathrm{Fnd}\,\mathfrak{E} $であり,$ \mathfrak{E}\subseteq\bar{I} $だから,順序数の定義は実はもっと簡略化できるが,正則性を前提しない定義の方が汎用的である.
}.
\end{df}

\noindent おのおのの順序数はそれ以前のすべての順序数のクラスであり,順序数の大小関係は$ \mathfrak{E}\uphr\mathrm{NO} $である.また,有限順序数は単に自然数であり($ \mathbb{N}\subseteq\mathrm{NO} $),$\mathbb{N}$自身は最初の超限順序数$\omega$である($\mathbb{N}=\omega$).
また,後続者関数を$ \mathrm{S} = \lambda_x(x\cup\classab{x}) $とすると(\fndp{8}{8}),$ x\in(\mathrm{S}\img\univ)\cap\mathrm{NO} $は先行者を持つ後続順序数であり,$ x\in\barl{(\mathrm{S}\img\univ)}\cap\mathrm{NO} $は直接の先行者を持たない極限順序数である.

次に,右域が順序数か$ \mathrm{NO} $である一般化された系列を導入して,そこから累積階層の超限系列$\mathrm{W}$を定義する.
\begin{df}
\label{df:超限系列}
\kagi{$
    \seq\alpha
$}は\kagi{$
    \func\alpha\con{1}\breve{\alpha}\img\univ\in\mathrm{NO}\case{2}{1}{1}\breve{\alpha}\img\univ = \mathrm{NO}
$}を表わす,
\end{df}

\begin{df}
\label{df:累積階層}
\kagi{$
    \mathrm{W}
$}は\kagi{$
\union{
    \classab{w:\seq w\con{1}
    w\fap\Lambda = \Lambda\con{1}
    (x)[x\in\breve{w}\img\univ\case{1}{1}{2}\\\hfill
        x\in\mathrm{S}\img\univ\case{1}{1}{1}w\fap x = \mathcal{P}(w\fap(\breve{\mathrm{S}}\fap x))\con{2}
        \Lambda\neq x \notin \mathrm{S}\img\univ\case{1}{1}{1}w\fap x = \union{(w\img x)}
    ]
}
}
$}を表わす.
\end{df}

\noindent A \ref{axim:正則性}の正則性公理は,$ \univ = \union{(\mathrm{W}\img\univ)} $と等値である\footnote{キューネン~\cite[p.\,134]{キューネン}}.
すべての存在者は,$ \Lambda $から始まる反復的な過程のどこかの段階,すなわち,ある水準$x\in\mathrm{NO}$の階層$ \mathrm{W}\fap x $の要素として現れる.
この点,\fndp{11}{11}の$ \mathcal{W} $は,$\mathrm{W}$と$n\in\mathbb{N}$に$\omega\cdot n$を与える関数(その左域が$ \omega^{2} $未満の極限順序数のクラスになる)との積である.つまり,
\[
    \mathcal{W} = \mathrm{W}\resl\classab{\orp{\omega\cdot n,n}:n\in\mathbb{N}}.
\]
そして,順序数の加法・乗法・累乗の一般的な定義をどう構成するにしても,
\begin{align*}
    \func z\con{1}\arg z = \mathbb{N}\con{1}z\fap\Lambda = \Lambda\con{1}z\resl\mathrm{S}\resl\breve{z}\subseteq \lambda_x[x\cup(\lambda_i(\iter{\mathrm{S}}{i}\fap x)\img\mathbb{N})]
\end{align*}
なる無限系列$ z $について,$ \omega\cdot n = z\fap n $.また,$ \omega^{2} = \union{(z\img\univ)} $.
なお関連して,$ \func \lambda_i(\iter{\mathrm{S}}{i}\fap x) $,かつ$ \mathbb{N}\in\univ $(A \ref{axim:無限}).それゆえA \ref{axim:置換}により,
\[
    \lambda_i(\iter{\mathrm{S}}{i}\fap x)\img\mathbb{N}\in\univ.
\]
また,$ (x)(y)(x\cup y\in\univ) $であるから(\fndp{10}{10}のT 6),
\[
    x\cup(\lambda_i(\iter{\mathrm{S}}{i}\fap x)\img\mathbb{N})\in\univ.
\]

①環境の制御と予知に直接使用する言語的装置においても,②またその延長上に想定される規制の概念を使用する工学的過程においても,必要とされる存在者の領域は,ある$n\in\mathbb{N}$について$ \mathrm{W}\fap(\omega\cdot n) $であるか,せいぜい$ \mathrm{W}\fap(\omega^{2}) $である.それは,どの水準の階層でもそうであるが,存在者の全領域の無限に小さな部分でしかない.特に①の存在論はZCで供給可能である.そして,$ \omega\in x\notin \mathrm{S}\img\univ $なる任意の$x\in\mathrm{NO}$について,$\mathrm{W}\fap x$はZCのモデルになる\footnote{
    キューネン~\cite[p.\,192]{キューネン}.
}.すなわち,$ u = \mathrm{W}\fap x\con{1}r\fap\orp{2,0} = \mathfrak{E}\uphr(\mathrm{W}\fap x) $なる任意のモデル$ \orp{u,r}\in\mathrm{MD} $において,ZCの公理は全て真である.したがって,既に$ \mathrm{W}\fap(\omega\cdot 2) $を対象領域とするモデルが,(ZCを含む)世界に関する包括的理論$ \mathfrak{K} $のモデルになると想定できるのだ.

\subsection{最終工程}
\label{ssec:最終工程}

信念体系の最適化の第二段階は,$\mathrm{ZFC}$を受容して自覚的に存在論を確定することだ.メソッドの適用という実践としては,それは,我々を苦しめたり喜ばせたりする物理的環境を,その都度存在論的に再記述する工程として現れる.「その都度」と言うのは,不快な感情が生起した場面か,または,信念体系の最適化可能な部分を発見した場面において,と言うことだ.不快な感情それ自体も含めて,関連する物理的システムは何らかのクラスであること,それは$ \Lambda $から始まる反復的過程の中にあること,その事実を信念体系に浸透させるように事態を再記述するのだ.
しかし,第二段階はこれで終わりではない.最後のステップがある.

\subsubsection{代替モデル}
\label{sssec:代替モデル}

本稿のメソッドの中核は,第一段階及び第二段階を通じて,信念体系の要素を標準言語の文,要するに$\mathfrak{L}$の文へと規格化・形式化することだった.そして\ref{ssec:外延性}では,標準言語に共通する性質として外延性が言われた.共通する重大な特徴がもう1つある.モデルの代替性だ.すなわち,任意の論理式のクラス$ l\subseteq\mathrm{L} $について,$l$のモデル$m\in\mathrm{MD}$が存在するなら,$m$以外の無限に多くの$l$のモデルが存在する.ここで$m$が$l$のモデルであると言うのは,$l$の要素が$m$で全て真であるということ,つまり$ l\subseteq\mathrm{T}^{m} $ということを意味する.モデルを固定できないということは$l$は特定の存在論的基盤を持たないということだ\footnote{
    クワイン~\cite[pp.\,44-46]{クワインe}.
}.$l$は科学とテクノロジーに関連する全ての理論を包括した理論$\mathfrak{K}$であり得る(\ref{ssec:存在論}).

モデルの代替性は(ZC及び$\mathrm{ZFC}$の)3つの定理(T \ref{thm:代替モデル充足},T \ref{thm:代替モデル真理},T \ref{thm:LST})にまとめることができる.それを示すためにいくつかの概念を準備しよう.
\paragraph*{論理式の複雑度}任意の$p\in\mathrm{L}$にその複雑度(真理関数及び量化子の出現回数)を与える関数$\mathfrak{d}$を再帰的に規定する.すなわち,$p$が原子式のとき,$\mathfrak{d}\fap p = 0$.また,任意の$q,q'\in\mathrm{L}$について,
\begin{gather*}
p = \orp{1,q} \case{1}{1}{1} \mathfrak{d}\fap p = \mathfrak{d}\fap q+1,\\
p = \orp{2,q,q'} \case{1}{1}{2} \mathfrak{d}\fap p = \mathfrak{d}\fap q + \mathfrak{d}\fap q' +1,\\
(\exists i)(p = \orp{3,i,q}) \case{1}{0}{1} \mathfrak{d}\fap p = \mathfrak{d}\fap q +1.
\end{gather*}
\paragraph*{代替関数と代替モデル}$ m,m'\in\mathrm{MD}\con{1}m = \orp{u,r}\con{1}m'=\orp{u',r'} $とする.このとき,
\begin{gather*}
    \func f\con{1}\func\breve{f}\con{1}f\img\univ = u'\con{1}\breve{f}\img\univ = u,\\
    r' = \classab{\orp{v,\orp{n,i}}:
    v = \classab{\mathcal{O}\fap x:
        (\exists z)(
            \mathcal{O}\fap z\in r\fap \orp{n,i}\con{1}x = f\resl z
        )
    }\con{1}\orp{n,i}\in\arg r
    }
\end{gather*}
である$f$を,$ m $から$ m' $への代替関数と言う.そして,$ m $から$ m' $への代替関数が存在するとき,$m'$は$m$の代替モデルと言う.

\begin{thm}
\label{thm:代替モデル充足}
$f$が$ m = \orp{u,r} $から$ m' = \orp{u',r'} $への代替関数ならば,任意の$p\in\mathrm{L}$について,
\[
    (s)(s\in \mathrm{seq}^u\case{1}{1}{2}\orp{s,p}\in\mathrm{SR}^m\case{3}{1}{1}\orp{f\resl s,p}\in\mathrm{SR}^{m'}).
\]
\end{thm}
\begin{pfx}
\setcounter{equation}{0}
論理式の複雑度に関する帰納法による.
\[
    \alpha = \classab{n:
        (p)[p\in\mathrm{L}\con{1}\mathfrak{d}\fap p = n\case{1}{1}{0}
            (s)(s\in \mathrm{seq}^u\case{1}{1}{2}\orp{s,p}\in\mathrm{SR}^m\case{3}{1}{1}\orp{f\resl s,p}\in\mathrm{SR}^{m'})
        ]
    }
\]
と置く.そして,任意の$ n\in\mathbb{N} $について,$ n\subseteq \alpha\case{1}{1}{1}n\in\alpha $であること,つまり,
\begin{equation}
    n\in\mathbb{N}\con{1}n\subseteq\alpha \tag*{[1]}
\end{equation}
であると仮定して,$ n\in\alpha $であることを示す.そこで,
\begin{equation}
    p\in\mathrm{L}\con{1}\mathfrak{d}\fap p = n
\end{equation}
であると仮定する.

$n = 0$のケース:$p$は原子式であり,ある$n,i,k$が存在して,$ p = \orp{0,\orp{n,i},k} $.すると任意の$s\in \mathrm{seq}^u$について,
\begin{align*}
    \orp{s,p}\in \mathrm{SR}^m & \case{3}{1}{1} \mathcal{O}\fap (s \resl k)\in r\fap\orp{n,i}\\
    &\:\, \case{3}{0}{1} \mathcal{O}\fap (s \resl k)\in r\fap\orp{n,i}\con{1}(\exists x)(x = f\resl(s \resl k))\\
    &\:\, \case{3}{0}{1} \mathcal{O}\fap (f\resl s \resl k)\in r'\fap\orp{n,i}\\
    &\:\, \case{3}{0}{1} \orp{f\resl s,p}\in \mathrm{SR}^{m'}.
\end{align*}
したがって,
\begin{equation}
    0 = n\case{1}{1}{0}(s)(s\in \mathrm{seq}^u\case{1}{1}{2}\orp{s,p}\in\mathrm{SR}^m\case{3}{1}{1}\orp{f\resl s,p}\in\mathrm{SR}^{m'}).
\end{equation}

$n \neq 0$のケース:この場合,$p$は否定,条件法,普遍量化のいずれか.つまり,
\begin{equation*}
    (\exists q)(p = \orp{1,q})\case{2}{0}{0}(\exists q)(\exists q')(p = \orp{2,q,q'})\case{2}{0}{0}(\exists q)(\exists i)(p = \orp{3,i,q}).
\end{equation*}

ある$q$が存在して$p = \orp{1,q}$であるとき:$\mathfrak{d}\fap q\in n$.すると任意の$s\in\mathrm{seq}^{m}$について,
\begin{align*}
    \orp{s,p}\in\mathrm{SR}^m & \case{3}{1}{1}\orp{s,q}\notin\mathrm{SR}^m \\
     &\:\, \case{3}{0}{1}\orp{f\resl s,q}\notin\mathrm{SR}^{m'} \tag{[1]により}\\
     &\:\, \case{3}{0}{1}\orp{f\resl s,p}\in\mathrm{SR}^{m'}.
\end{align*}

ある$q,q'$が存在して$p = \orp{2,q,q'}$であるとき:$\mathfrak{d}\fap q,\mathfrak{d}\fap q'\in n$.すると任意の$s\in\mathrm{seq}^{m}$について,
\begin{align*}
    \orp{s,p}\in\mathrm{SR}^m & \case{3}{1}{2}\orp{s,q}\in\mathrm{SR}^m\case{1}{1}{1}\orp{s,q'}\in\mathrm{SR}^m\\
    &\:\, \case{3}{0}{2} \orp{f\resl s,q}\in\mathrm{SR}^{m'}\case{1}{1}{1}\orp{f\resl s,q'}\in\mathrm{SR}^{m'} \tag{[1]により}\\
    &\:\, \case{3}{0}{1} \orp{f\resl s,p}\in\mathrm{SR}^{m'}.
\end{align*}

ある$q$が存在して$p = \orp{3,i,q}$であるとき:$\mathfrak{d}\fap q\in n$.任意の$s\in\mathrm{seq}^{m}$について,$\orp{s,p}\in \mathrm{SR}^m$であると仮定する.すると,任意の$a\in u$について,
\begin{align*}
    a\in u & \case{1}{1}{1}\orp{s\tbinom{i}{a},q}\in\mathrm{SR}^m\\
    & \:\,\case{1}{0}{1}\orp{f \resl (s\tbinom{i}{a}),q}\in\mathrm{SR}^{m'} \tag{[1]により}\\
    & \:\,\case{1}{0}{1}\orp{(f\resl s)\tbinom{i}{f\fap a},q}\in\mathrm{SR}^{m'}.
\end{align*}
他方,任意の$a'\in u'$について,ある$a\in u$が存在して,$ a' = f\fap a $.すると,
\begin{equation*}
    \orp{(f\resl s)\tbinom{i}{a'},q}\in\mathrm{SR}^{m'}.
\end{equation*}
それゆえ,$ \orp{f\resl s,p}\in \mathrm{SR}^{m'} $.他方,$ \orp{f\resl s,p}\in \mathrm{SR}^{m'}$と仮定して,証明を逆に辿ると,$\orp{s,p}\in \mathrm{SR}^m $.

以上から,
\begin{equation}
    0 \neq n\case{1}{1}{0}(s)(s\in \mathrm{seq}^u\case{1}{1}{2}\orp{s,p}\in\mathrm{SR}^m\case{3}{1}{1}\orp{f\resl s,p}\in\mathrm{SR}^{m'}).
\end{equation}

(1)(2)(3)により,$ n\in\alpha $.[1]の仮定と合わせると,$ (n)(n\in\mathbb{N}\con{1}n\subseteq\alpha\case{1}{1}{1}n\in\alpha) $.したがって,任意の複雑度$ j\in\mathbb{N} $について,$ j\in\alpha $.つまり,任意の$p\in\mathrm{L}$について,
\[
    (s)(s\in \mathrm{seq}^u\case{1}{1}{2}\orp{s,p}\in\mathrm{SR}^m\case{3}{1}{1}\orp{f\resl s,p}\in\mathrm{SR}^{m'}).
\]

\end{pfx}

\begin{thm}
\label{thm:代替モデル真理}
$ m' = \orp{u',r'} $が$ m = \orp{u,r} $の代替モデルならば,$\mathrm{T}^{m} = \mathrm{T}^{m'}$.
\end{thm}
\begin{pfx}
\setcounter{equation}{0}
T \ref{thm:代替モデル真理}の仮定により,$m$から$m'$への代替関数$ f $が存在する.

次に,$p\in\mathrm{T}^{m}$と仮定すると,
\begin{equation}
    (s)(s\in\mathrm{seq}^{u}\case{1}{1}{1}\orp{s,p}\in\mathrm{SR}^{m}).
\end{equation}
任意の$s'\in \mathrm{seq}^{u'}$について,$ \breve{f}\resl s'\in \mathrm{seq}^{u} $.すると,(1)により,$ \orp{\breve{f}\resl s',p}\in\mathrm{SR}^{m} $.また,T \ref{thm:代替モデル充足}により,
\[
   \orp{\breve{f}\resl s',p}\in\mathrm{SR}^{m}\case{3}{1}{1}\orp{f\resl(\breve{f}\resl s'),p}\in\mathrm{SR}^{m'}.
\]
したがって,$ \orp{f\resl(\breve{f}\resl s'),p}\in\mathrm{SR}^{m'} $.$ f\resl(\breve{f}\resl s') = s' $だから,
\begin{equation*}
    (s')(s'\in\mathrm{seq}^{u'}\case{1}{1}{1}\orp{s',p}\in\mathrm{SR}^{m'}).
\end{equation*}
すなわち,$ p\in\mathrm{T}^{m'} $.$ p\in\mathrm{T}^{m'}\case{1}{1}{1}p\in\mathrm{T}^{m} $も同様にして証明できるから,$ \mathrm{T}^{m'}=\mathrm{T}^{m'} $.
\end{pfx}

\begin{thm}
\label{thm:LST}
任意の$ l\subseteq\mathrm{L} $について,$l$がモデルを持つならば,$l$は可算モデル(対象領域が$\mathbb{N}$と同じサイズを持つ)を持つ.論理的記法で言い換えると,任意の$ l\subseteq\mathrm{L} $について,
\begin{multline*}
    (\exists m)(m\in\mathrm{MD}\con{1}l\subseteq \mathrm{T}^{m})\case{1}{0}{0}\\
    (\exists m')[
        m'\in\mathrm{MD}\con{1}l\subseteq \mathrm{T}^{m'}\con{1}
        (\exists f)(\func f\con{1}\func\breve{f}\con{1}f\img\univ = \mathbb{N}\con{1}\breve{f}\fap\univ = \mathcal{L}\fap(m'))
    ].
\end{multline*}
\end{thm}
\begin{pf}
    省略する\footnote{
        清水~\cite[p.\,131]{清水},クワイン~\cite[pp.\,190--193]{クワインb}等を参照.
    }.
\end{pf}

\noindent T \ref{thm:LST}はレーヴェンハイム・スコーレムの定理(LST)と呼ばれる.LSTによって,モデルを持つ論理式の集合は,$\mathbb{N}$を領域とするモデルを持つことが帰結する.すなわち,ある$m\in\mathrm{MD}$について,$ l\subseteq\mathrm{T}^{m} $であるとき,T \ref{thm:LST}により,$ \mathbb{N} $と$\mathcal{L}\fap m$との間の1対1変換$f$が存在する.すると,$ \mathcal{L}\fap (m') = \mathbb{N} $である$m$の代替モデル$m'$が存在する.したがって,T \ref{thm:代替モデル真理}によって,$l\subseteq\mathrm{T}^{m'} $.

\subsubsection{相対化}
\label{sssec:相対化}

1つのモデルが特定されている場合,そこから代替モデルを作るのは極めて容易だ.解釈空間の開始点$\breve{\epsilon}\fap 0 $について考えよう.それは$\mathfrak{L}$における標準的モデルであり(\fndp{19}{19}),包括理論$\mathfrak{K}$のモデルであると想定されている($ \mathfrak{K}\subseteq \mathrm{T}^{(\breve{\epsilon}\fap 0)} $).
今,$\breve{\epsilon}\fap 0 = \orp{u,r}\con{1}f = \lambda_x\classab{x}\uphr u$と置く.そして,
\begin{gather*}
    u' = (\lambda_x\classab{x}\uphr u)\img\univ,\\
    r' = \classab{\orp{v,\orp{n,i}}:
        v = \classab{\mathcal{O}\fap x:
            (\exists z)(
                \mathcal{O}\fap z\in r\fap \orp{n,i}\con{1}x = f\resl z
            )
        }\con{1}\orp{n,i}\in\arg r
    }
\end{gather*}
とすると,$f$は,$\orp{u,r}$から$\orp{u',r'}$への代替関数であり,$\orp{u',r'}$は$\orp{u,r}$の代替モデルである.この場合の代替関数は対象領域$ u $のメンバーをその単一クラスに変換する.他方,$ f = \lambda_x(\bar{x}\cap u)\uphr u $とすると,$u$上の補クラスに変換する.また,$ f = \lambda_x(\orp{\Lambda,x})\uphr u $とすると,$u$のメンバーを$\Lambda$とそれの順序対に変換する.
このような代替関数は際限なく作れるが,代替モデルの対象領域は元のモデルの対象領域と同じサイズに制限されている.しかし,どのような$\mathfrak{K}$のモデルが存在するにしても,LSTによって,$\mathbb{N}$を領域とする$\mathfrak{K}$のモデルが存在することになる.また,$\mathrm{T}^{(\breve{\epsilon}\fap 0)}\subseteq \mathrm{T}^{(\breve{\epsilon}\fap 0)}$であるから,$ \breve{\epsilon}\fap 0 $はトリヴィアルに$\mathrm{T}^{(\breve{\epsilon}\fap 0)}$のモデルだ.しかし,このことから,$\mathrm{T}^{(\breve{\epsilon}\fap 0)}$のメンバーが全て真となるような$\mathbb{N}$を領域とする別のモデルが存在することが帰結する.

T \ref{thm:代替モデル充足}〜T \ref{thm:LST}による帰結が意味することは,標準言語の理論あるいは文の集合が前提とする存在論は,モデルあるいはモデルを特定化する言語に相対的であるということだ.したがって,何が存在するかは事実の問題ではない.もちろん本稿では,定理の証明も$\mathfrak{K}$とそのモデルを対象化して考察することも,$\mathrm{ZFC}$を使用して行われている.したがって,$\mathrm{ZFC}$が認める存在者が存在することは予め前提されている.$\mathrm{V} = \union{(\mathrm{W}\img\univ)}$であることは前提されているのだ.問題は,そうだとしても,包括的な世界記述$\mathrm{T}^{(\breve{\epsilon}\fap 0)}$や,標準的な科学を包括する理論$\mathfrak{K}$がモデルを持つ限り,それらが肯定する存在が$\union{(\mathrm{W}\img\univ)}$のどのメンバーであるのかは,事実の問題ではないということである.要するに,我々が馴染んでいる物理的世界はないということだ.累積階層のどの部分についても,本性として現実性を帯びているような部分はない.

T \ref{thm:代替モデル充足}〜T \ref{thm:LST}は$\mathrm{ZFC}$(のモデル)に対しても適用される.ただし,それらが意味を持つのは$\mathrm{ZFC}$がモデルを持つ場合である.矛盾する理論が前提する存在者を云々しても無意味だからだ.しかし,$\mathrm{ZFC}$のモデルを$\mathrm{ZFC}$の中で構成することはできない.したがって,$\mathrm{ZFC}$の存在論を有意味に相対化するには,$\mathrm{ZFC}$より強いクラス理論を使用する必要がある.体系を強める典型的な方法は,いわゆる巨大基数の存在を認める公理を追加することだ\footnote{
    薄葉・藤田~\cite{薄葉}等を参照.
}.しかし,そうすると今度は,その強められた体系$\mathrm{ZFC}^{+}$が認める存在者が予め前提されるだけだ.そこには$\mathrm{ZFC}$が認める存在者も含まれている.そして$\mathrm{ZFC}^{+}$の中で同様にして,$\mathrm{T}^{(\breve{\epsilon}\fap 0)}$及び$\mathfrak{K}$の存在論は相対化される.さらに$\mathrm{ZFC}^{+}$に対しても,それが認める以上の巨大基数を要請してそれを強めることができるが,後の展開は同じことだ.要するに,クラスの存在論は相対化できないということだ.これに対して,物理的世界の存在論は$ \mathrm{ZFC},\mathrm{ZFC}^{+},\mathrm{ZFC}^{++},\dots $のどの段階でも相対化される.

物理的世界の記述から存在論的基盤が剥ぎ取られると,残るのは言語の使用が喚起するビジョンだけだ.物理的世界を支えている固有の対象や領域はない.したがって,あなたの周囲の現実が強固に感じられる理由は,それが実在的だからではなく,あなたが特定のモデルあるいはそれを特定化する記述を強固に受け入れていることの反映でしかない.
因果と規制の概念が解釈空間に相対化されていることは,この事実を暗示している.1つの解釈空間を選択することは,1つの現実(開始点$ \breve{\epsilon}\fap 0 $)と可能性の範囲を固定することだ.
さらに,あなたが普段それに自己同一化している人間,あるいは,あなたの思考・感情・行動もまた物理的システムであり,それゆえに,実体のない幻影にすぎない.全ての存在は無限の全体性$\union{(\mathrm{W}\img\univ)}$の要素であり,$\union{(\mathrm{W}\img\univ)}$の中には,それらの物理的システムであるような固有の対象はないからだ.
こうして物理的自我は消え去り,無限の全体性だけが残される.他方で,いわゆる形而上学的主体(哲学的自我)\footnote{ここでは,世界の全てが自らの思考・表象の内容であるような主体を意味している.}が露呈するが,それは世界そのものとして無限の全体性に同一化される.物理的世界とは,この拡張された自我が観照する,特定のモデルに基づくビジョンなのだ\footnote{このとき当該モデルは「現実性を持つ」とも言える.}.
メソッドの要点は,言語を規格化して存在をクラスに還元し,物理的世界を相対化することだ.信念体系の最適化が進行するにつれて,物理的システムとしてのあなたはより適応的になるだろう.しかし最も重要なことは,最適化プロセスの終了後に,あなたが本来のあり方へと,全てを内在する全体性へと回帰するということだ.

\section{結論}
\label{sec:結論}

本稿で構築されたメソッドの適用をまとめよう.まず,不快な感情が生起したとき,
\begin{enumerate}[label=(\arabic*)]
    \item 関連する信念体系の部分$g$を仮説的に特定化する,
    \item 信念体系を部分的に最適化して$g$を取り除く.
\end{enumerate}
不快な感情が生起していないときでも,最適化可能な信念体系の部分$g$を発見したら,(2)を適用する.

次に,(2)の適用の仕方はこうだ.
\begin{enumerate}[label=(\alph*)]
    \item $g$の内包性を除去して論理的表記法に規格化した$g'$について,$g'$を信念体系に取り込み,$g$を廃棄する.
    \item $g$に関連して存在論的に不明瞭な部分があれば,$\mathrm{ZFC}$に基づいて存在論を確定する.
    \item 代替モデルに関する定理に基づいて,物理的世界の存在論を相対化する.
\end{enumerate}



\bibliographystyle{jplain}
\bibliography{refs}

\end{document}