% !TeX root = method.tex

\section{論理的還元}
\label{sec:論理的還元}

信念体系の最適化の第一段階は,量化論理の表記法によって統制された言語(標準言語)への規格化ないし論理的還元だ.
ここで想定されている作業は,間主観性を持たないものの,環境を予測し制御するテクノロジーを構成するかその基盤となる言語的装置を開発することと,本質的に類似する.
つまり,科学と工学における理論構築の少々劣化した主観的なバージョンだ.
したがって,それらの理論構築の際に追求される事柄がここでも追求される.それは予測と制御の効果を最大化するために,概念枠を単純化し明確化することだ.
この概念枠には,全ての科学理論に共有されるような一般的な言語的装置(表記法)も含まれる.それを開発するのが量化論理学だ\footnote{
    したがって,量化論理学は実在の最も一般的な特徴を描写する科学と言うことができる.クワイン~\cite[pp.\,268--269]{クワインd}.
}.

\subsection{標準言語}
\label{ssec:標準言語}

1個の標準言語は,形式化された文またはその図式である論理式のクラス$\mathrm{L}$(量化言語)の部分クラスだ.
量化言語の文法は,原子式の構成要素として,無限個の変項\kagi{$x$},\kagi{$x'$},\kagi{$x''$}などと,無限個の述語記号
\[
   \text{ 
    \kagi{$ (F,) $},\kagi{$ (F,)' $},\kagi{$ (F,)'' $},$\cdots$,\kagi{$ (F,\!,) $},\kagi{$ (F,\!,)' $},\kagi{$ (F,\!,)'' $},$\cdots$
    }
\]
を持つ.\kagi{$,$}の数が$n$,\kagi{$'$}の数が$i$であるとき,$n$項述語記号の$i$番目を意味する.実際上は,変項は\kagi{$x$},\kagi{$y$},\kagi{$z$}等々で,原始的述語以外の述語記号は\kagi{$F$},\kagi{$G$},\kagi{$H$}等々で代用される.
次に,量化言語の文法は,原子式から複合式を構成する手段として,否定と条件法の真理関数\kagi{$\neg$}及び\kagi{$\supset$}と,普遍量化子\kagi{$(x)$},\kagi{$(x')$},\kagi{$(x'')$}などを持つ.
そして論理式は,次の3つの規則により再帰的に記述される.
\begin{enumerate}[label=(\arabic*)]
    \item \kagi{$ R\alpha_1\dots\alpha_n $}の\kagi{$ R $}に$0\neq n$項述語記号の$i$番目を,\kagi{$ \alpha_1\dots\alpha_n $}に$n$個の変項を並べて代入した結果は,論理式である.
    \item \kagi{$\neg P$}と\kagi{$(P\supset Q)$}において,\kagi{$P$}と\kagi{$Q$}に任意の論理式を代入した結果は,論理式である.
    \item \kagi{$(\alpha)P$}において,\kagi{$\alpha$}に任意の変項を,\kagi{$P$}に任意の論理式を代入した結果は,論理式である.
\end{enumerate}
連言\kagi{$ \con{1} $},選言\kagi{$ \lor $},双条件法\kagi{$ \equiv $}等の他の真理関数は否定と条件法から,存在量化子\kagi{$ (\exists x) $}は普遍量化子から定義できる.否定と条件法及び普遍量化以外の組み合わせを原始的として,他をそこから定義することもできる.

次に,1個の標準言語は,当該言語の述語として使用する述語記号を指定することによって規定される.例えば$\mathfrak{L}$では,$2$項述語記号の$0$番目\kagi{$ (F,\!,) $}が,要素関係を表わす\kagi{$ \in $}の形式的な表現であるとみなされる.すなわち,
\begin{itemize}
    \item \kagi{$(\alpha\in\beta)$}の\kagi{$\alpha$}と\kagi{$\beta$}に任意の変項を代入した結果は,\\\hfill
    \kagi{$ (F,\!,)\alpha\beta $}に同一の代入をした結果を表わす.
\end{itemize}
この文脈的定義により,\kagi{$ (F,\!,) $}は,$\mathfrak{L}$における実質的な原始的述語として(実際上は\kagi{$ \in $}で代用して),使用される.
\fndp{4}{4}の原初的言語は,\kagi{$ \in $}を唯一の原始的述語として持つ標準言語である.他方,$\mathfrak{L}$は原初的言語に多数の述語を追加したものであり,科学とテクノロジーに寄与する文が実質的に全て含まれる程度に包括的な標準言語である.
そして,$\mathfrak{L}$の原始的述語以外の述語記号が出現しない論理式が$\mathfrak{L}$の文である.他方,文でない論理式は文の論理構造を表わす図式(量化図式)となる($\mathfrak{L}$の原始的述語でない述語記号は$\mathfrak{L}$の開放文の位置を表わす).

なお,標準言語は変項以外の単称名辞を持たない.ある対象$z$の名前は,記述型\kagi{$ (\imath x)Ax $}の\kagi{$ Ax $}に$ z $についてのみ真である述語を代入した結果によって代用できる.そして,
\kagi{$ F(\imath x)Ax $}という文脈は,同一性を表わす\kagi{$ = $}がその言語で定義可能であれば,
\[
    (\exists y)[Fy\con{1}(x)(Ax\case{3}{0}{1}x = y)]
\]
の省略形とみなすことができる\footnote{クワイン~\cite[p.\,250]{クワインb}}.

文であれ図式であれ論理式はモデルによって解釈できる.1個のモデルは,空でない集合$u$(対象領域)と述語記号に$u$上の$n$項関係を割り当てる解釈関数$r$の組$\orp{u,r}$だ.
そして,論理式のクラス$x$が論理式$y$を含意するのは,
\[
   (m)(s)[m\text{ はモデル}\con{1}s\text{ は$\mathcal{L}\fap m$上の対象列}\case{1}{1}{1}
        (l)(l\in x\case{1}{1}{1}\orp{s,m}\text{ は$l$を充足})\case{1}{0}{1}\orp{s,m}\text{ は$y$を充足}
   ]
\]
であるとき,かつそのときに限られる.このような$x,y$のクラス(論理的含意関係)を\kagi{$ \mathrm{imp} $}で表わすと,1個の論理式が含意する関係は$ \brevel{(\lambda_x\classab{x})}\resl \mathrm{imp} $である.また,妥当式(任意のモデルにおいて真となる論理式)は$ \brevel{\mathrm{imp}}\img\classab{\Lambda} $である(\fndp{15}{15}).論理的真理つまり論理的に真である文は,妥当式である標準言語の文である.この点に関連して,標準言語$l\subseteq\mathrm{L}$と$ \alpha\subseteq l $について,クラス$\classab{p:\orp{\alpha,p}\in\mathrm{imp}}\cap l$を$\alpha$を公理集合とする$l$の理論と言う.量化理論は$\Lambda$を公理集合とする$\mathrm{L}$の理論である.
妥当式の全てが定理となるという意味で完全である,さまざまな量化理論の演繹体系が知られている\footnote{
    クワイン~\cite[pp.\,171--225]{クワインb}.
}.

さて,最適化対象である信念体系$\mathfrak{B}$と,それが規格化される先の標準言語(の文の集合)$\mathfrak{L}$を考える.規格化は,信念体系の問題の部分$ b\subseteq\mathfrak{B} $を,ある$l\subseteq\mathfrak{L}$と入れ替えることだ.つまり$\mathfrak{B}$を
\[
    \mathfrak{B}' = (\mathfrak{B}\cap\bar{b})\cup l
\]
に更新するのだ\footnote{メソッドの適用において,$b$と$l$を完全に特定化する必要はない.}.ただし,$l$のメンバーは$ \mathfrak{B}\cap\bar{b} $のどのメンバーの解釈でもない.
他方,$l$のメンバーは$b$のメンバーの解釈ではあり得る.しかし,ここでの解釈関係は同義性の関係ではないし,可及的に用法を保存するような関係ですらない.理論構築としての規格化は概念分析ではないのだ\footnote{
    理論構築としての規格化は,真理条件的な意味の理論を適用するために日常言語の文の論理形式を与えるような規格化(デイヴィドソン~\cite[pp.\,18--19]{デイヴィドソン})とも異なる.この意味での規格化は概念分析でもなく,文法的な規格化である.
}.さらに,仮に$\mathfrak{B}$のメンバーを永久文に限ったとしても,通常,
\[
    k\subseteq \mathfrak{L}\times\mathfrak{B}
\]
なる全ての(適切な)解釈関係$ k $について,$ l\nsubseteq k\img\univ\con{1}b\nsubseteq \breve{k}\img\univ $.つまり,$b$のどのメンバーの解釈でもない$l$のメンバー$q$が存在するし,逆に,$l$のどのメンバーにも解釈されないような$b$のメンバー$p$も存在する.
最適化において捨てられる信念$p$の代わりに,その信念の解釈であるような文を信念体系に取り込む必要はないということだ.仮に$p$の適切な解釈が$\mathfrak{L}$の中にあったとしてもである.
逆に最適化によって,これまでの信念体系に対応物のない新規の信念$q$が追加されてもよい.

\subsection{外延性}
\label{ssec:外延性}

標準言語に共通する特徴であって,メソッドに関連する重要な性質が外延性だ.信念体系を$\mathfrak{L}$に規格化することは,それに外延性を付与することを意味する.
$ \mathfrak{L} $の外延性は,任意の$ p,q,r\in \mathfrak{L} $と,$ r $における$ p $の出現のいくつかを$ q $に置換した式$ r' $について,
\[
    p\text{ の外延} = q\text{ の外延}\case{1}{1}{1}r\text{ の真理値} = r'\text{ の真理値}
\]
であることだ.しかし,この定式化に残る不明瞭さを除去する必要がある.

この点,\fndp{15}{15}におけるクラス$ \delta\exten\beta $は,要はモデル$\delta$に相対的な論理式$\beta$の外延である.これを$\beta$が閉鎖論理式であるケースにも拡張して,
\[
    \delta\parallel \beta = (\delta\exten\beta)\cup
    \classab{s\fap 0:
    % \classab{\mathcal{O}\fap(s\resl v):
    \mathrm{var}\,\beta = \Lambda\con{1}
    % v = \classab{\orp{0,0}}\con{1}
    \orp{s,\beta}\in\mathrm{SR}^\delta
    }
\]
と置く.$ \beta $が閉鎖論理式であるときは($ \mathrm{var}\,\beta = \Lambda $),
\begin{gather*}
    \beta\in\mathrm{T}^\delta\case{1}{1}{1}\delta\parallel\beta = \mathcal{L}\fap \delta,\\
    \beta\notin\mathrm{T}^\delta\case{1}{1}{1}\delta\parallel\beta = \Lambda.
\end{gather*}
ところで,\fndp{19}{19}において,$\breve{\epsilon}\fap 0$は$\mathfrak{L}$の標準的モデルとして規定された.これを前提すれば,任意の$ p,q,r\in \mathfrak{L} $と,$ r $における$ p $の出現のいくつかを$ q $に置換した式$ r' $について,
\[
   (\breve{\epsilon}\fap 0)\parallel p = (\breve{\epsilon}\fap 0)\parallel q \case{1}{1}{2}r\in \mathrm{T}^{(\breve{\epsilon}\fap 0)}\case{3}{1}{1}r'\in\mathrm{T}^{(\breve{\epsilon}\fap 0)}
\]
であることとして,$\mathfrak{L}$の外延性を説明できる.

しかし実際には,外延性を規定するのに外延を実体化する必要はない.さらに特定の言語やモデルにも依存せず,標準言語一般に適用可能な外延性の定式化が存在する.すなわち,型式
\[
   (x_1)\dots(x_n)(A\case{3}{0}{0}B)\case{1}{0}{1}C_A\case{3}{0}{0}C_B
\]
の\kagi{$ A $},\kagi{$ B $}及び\kagi{$ C_A $}にそれぞれ任意の論理式$ p,q,r $を代入し,\kagi{$ C_B $}には,$ r $における$ p $の$0\leq n$箇所の出現を$ q $で置き換えた論理式$r'$を代入する.\kagi{$ (x_1)\dots(x_n) $}には,$ p,q $で自由出現し,かつ,$ r,r' $で束縛出現する変項のすべてを順次代入し,そのような変項がなければ削除する.以上の操作の結果は妥当式である\footnote{
    清水~\cite[p.\,84]{清水}.
}.この事実が標準言語の外延性である.

日常言語においては,したがって最適化前の信念体系においても,外延性が成り立たない文脈は広範囲に存在する.
例えば,\kagi{$ OP $}を,\kagi{$ P $}の位置に来る文に演算子「ということは事実であるべきだ」を結合して得られる文の型とする.今,$x$は人を殺したが,$y$は誰も殺していないとしよう.また,\kagi{$ \alpha $}と\kagi{$ \beta $}で,それぞれ$ x $と$ y $を指示する単称記述の位置を表わす.すると次の(1)(2)は共に真となる.
\setcounter{equation}{0}
\begin{gather}
    (\exists z)(z\text{ は人}\con{1}\text{$\alpha$は$z$を殺す}),\\
    \neg(\exists z)(z\text{ は人}\con{1}\text{$\beta$は$z$を殺す}).
\end{gather}
しかし,\kagi{$ OP $}の\kagi{$ P $}に(1)を代入した結果は偽であり,(2)を代入した結果は真である.

次の例として,\kagi{$ \text{□}P $}を,\kagi{$ P $}の位置に来る文に演算子「ということは必然的である」を結合して得られる文の型とする.\kagi{$ \neg(\text{□}\neg P) $}は「$P$は可能である」に相当する.そして,次の(3)(4)は共に真であるが,\kagi{$ \text{□}P $}の\kagi{$ P $}に(3)を代入した結果は真であり,(4)を代入した結果は偽である.
\begin{gather}
    1+2 = 3,\\
    (\exists x)(x\text{ はマッコウクジラ}\con{1}x\text{ はアルビノである}).
\end{gather}

次に,\kagi{$ P\text{□→}Q $}の\kagi{$ P $}と\kagi{$ Q $}の位置に来る文に反事実的条件法の演算子「もし仮に〜ならば…だろう」を結合して得られる文の型とする.
$x$は水中にも冷蔵庫内にもない砂糖の塊であり,\kagi{$ \alpha $}でそれを指示する単称記述の位置を表わす.すると,(5)(6)は共に偽である.
\begin{gather}
    \alpha\text{ は水に入れられる},\\
    \alpha\text{ は冷蔵庫に入れられる}.
\end{gather}
しかし,\kagi{$ P\text{□→}(\alpha\text{ は溶ける}) $}の\kagi{$ P $}に(5)を代入した結果は真であるが,(6)を代入した結果は偽である.

\ref{sssec:信念体系の関与}で信念体系の部分が不快な感情に関与する仕方をパターン化した.これらのパターンの事例において最も致命的なケースは,以上で述べたような内包演算子が作り出す非外延的な文によるものである.内包性こそが悪の根なのだ.\ref{ssec:因果と規範}以降で内包的な文の集合の理論的代替物が扱われる.しかし,内包演算子の分析がこれ以上に行われることはない.内包性の理論的代替物は元の文集合に対して互換性がないからだ.理論構築としての規格化は,概念分析や文法的な規格化とは異なり後方互換性の確保を要請しない.つまり,古い文集合の機能や関連する傾向性が,理論的代替物においても実現される必要はない.内包的な文は単に打ち捨てられる.もっとも,メソッドの適用が最大限に効果を発揮した場合でさえ,おそらく内包演算子の日常的な使用を止めることはできないだろう.それらはコミュニケーションにおける技術的な有効性を持つからだ.
メソッドが目指すものは,内包的な文の集合の理論的代替物を構築することよって,それが信念体系に与える影響を最小化することだ.

\subsection{因果と規範}
\label{ssec:因果と規範}

反事実的条件法の理論的代替物は,解釈空間に相対化された因果の概念である(\fndpp{11}{11--20}).因果関係$\mathcal{C}^\epsilon$は,$ a\in x $であることが$ b\in y $であることを($\epsilon$に相対的に)因果的に決定するような$\orp{x,y}$のクラスだ.\kagi{$ \epsilon $}は解釈空間を指示するクラス抽象体の位置を表わす型文字であり,解釈空間は文脈によって異なり得る.なお,\kagi{$ \orp{\orp{a,x},\orp{b,y}}\in\mathcal{C}^\epsilon $}は\kagi{$ \orp{a,x}\to_{\epsilon}\orp{b,y} $}とも書かれる.
他方,価値と規範に関する理論的代替物は,この因果の概念に基づく規制の概念だ(\fndpp{21}{21--29}).
準拠領域$x$について,規制類型$y$に関する規制が成立するのは,$y$の適用条件$ \app{\epsilon}y $を充たすような$x$について,$y$の執行可能性$\enf{\epsilon}y$と制御可能性$\cty{\epsilon}y$という2個の因果的構造が成立する場合だ.ただし,制御方向$\tildel{\trgl{y}}\fap 1 = \Lambda$のときは制御可能性は不要である.つまり,
\[
    x\in\app{\epsilon}y\cap \enf{\epsilon}y\con{2}\tildel{\trgl{y}}\fap 1 \neq \Lambda\case{1}{1}{1}x\in\cty{\epsilon}y
\]
であるとき,かつそのときに限り,$\orp{x,y}$は規制である.以上の説明は大雑把な要約だ.因果と規制の概念の完全な定式化は上記の参照先にある.

メソッドの適用は,ケースごとにこれらの理論的概念を使用して事態を再記述する\footnote{
    因果と規制の概念は解釈空間に相対化されているから,再記述に使用される文は,特定化されない解釈空間を表示する変項の自由出現を持つと考えられる.
}.そして,古い信念体系を更新するのだ.物理的システムに還元されない価値と規範という混乱した観念は捨て去られ,規制の因果的構造の記述によって代替される.因果関係と蓋然性は,解釈空間に明示的に相対化されることによって,それらの記述が何に依存しているのかが明確化される.因果と規制の概念の開発から始まり長い道程を経て,ようやく情報処理の擾乱は一掃され,信念体系は浄化されるのだ.
以下で適用例を示そう.

\subsubsection{未来を心配すること}
\label{sssec:未来を心配する}

未来の抑制条件の実現が現在の状況に依存している場合,その現在の状況も修正条件になる.それによって生じるのが「不安」と呼ばれるような感情だ.
この修正条件間の因果関係を直接経験しなくても,それの言語的代替によって現在の状況が修正条件になる.言語能力を持つ人間が不安を感じるには,未来を思い描くだけでよいのだ.
例えば,「今のこの収入だから,近い将来金欠になるだろう.」,「今一人だから今後も一人だ.」,「今の健康状態だと,まもなく生活に支障が出るだろう.」.
このような因果関係の記述を,解釈空間に相対化された因果関係の記述,つまり,
\[
   \orp{a,x}\to_{\epsilon}\orp{b,y}
\]
という形式に置き換える意義はいろいろとある.解釈空間が定めるモデル間の比較類似性という因果の判定基準が一応存在すること自体がそうだ.また,背景条件$ x\bkg{\epsilon}y $を考察して,$\orp{a,b}$がそれを充たすかどうか検討することもできる(\fndp{17}{17}).さらに,蓋然性が因果的に決定される事態として,$ y = \prob{z}{(n\uphl\epsilon)} $等と置く場合は,危険測度の値$n$に自覚的になれる.将来の漠然とした因果記述によって強い不安を覚えるケースでは,危険測度が曖昧化されることで,実質的にはそれが過大に見積もられていることが少なくないと思われる.

しかし,このようなケースでより致命的な要素は,将来の状況またはそれが依存していると記述される現在の状況の重要度の評価の方だ.最適化前はこの評価は漠然とした価値判断,「大変だ」,「とんでもないことになる」,「ヤバいことになる」等々になっているはずだ.この種の価値判断では,状況の記述とそれに対する自分の反応の仕方の記述が混淆されており,実際に起きていること以上の被害を錯覚させる.それが不安の持続性と耐性を高めてしまうのだ.この部分の最適化は因果関係の記述を精錬するだけではできない.\fndp{31}{31}の評価構造を使う.評価構造は規制集合の工学的妥当性その他の条件の重要度を判定する道具である.
ここでの問題は,S類型$y$と$ x\in\cty{\epsilon}y $について,$ e = \tilde{x}\fap 0\con{1}d = \tildel{\trgl{y}}\fap 0 $である場合における,抑制条件$\orp{e,d}$の(時区間$t$における)重要度の判定だ.ユーザー集合$\kappa$は信念体系の主体$ \tilde{x}\fap 1 $の単一クラスである.他方,抑制条件のクラスを扱うならば,
\[
    e \subseteq \classab{\orp{\tilde{x}\fap 0,\tildel{\trgl{y}}\fap 0}:x\in\cty{\epsilon}y\con{1}\tilde{x}\fap 1 \in\kappa}\con{1}
    d \subseteq \mathcal{P}(\mathfrak{E})
\]
としてもよい.いずれにせよ,$ e\in d $であることが因果的に決定する$ \kappa $メンバーの促進条件のクラス$\zeta_1$を次のように規定する.
\begin{multline*}
    \zeta_1 = \classab{\orp{\tilde{a}\fap 0,\tildel{\trgl{b}}\fap 0}:
    \orp{e,d}\to_{\epsilon}\orp{\tilde{a}\fap 0,\tildel{\trgl{b}}\fap 0}\con{1}a\in\cty{\epsilon}b\con{1}
    a\fap 0\subseteq t\con{1}\tilde{a}\fap 1 \in \kappa\con{1}\\
    b\in\mathrm{Reg}\con{1}
    (\exists x)(\tildel{\trgl{b}}\fap 0 = x^{::\epsilon}\cap\mser{(\arg b)}{0})\con{1}
    \tildel{\trgl{b}}\fap 1 = \trgl{b}\fap 0
    }.
\end{multline*}
同じく抑制条件のクラス$\zeta_2$は,上の等式の\kagi{$ \tildel{\trgl{b}}\fap 1 = \trgl{b}\fap 0 $}を\kagi{$ \tildel{\trgl{b}}\fap 1 = \barl{(\trgl{b}\fap 0)}\cap\msec{(\arg b)}{0} $}に変更すれば得られる.
なお,$\zeta_2$のメンバーは,それを惹起することによって$x\in\cty{\epsilon}y$となるような基礎抑制条件とは限らない.つまり,次のような$b$の抑制条件とは限らない.
\[
    \mathrm{pur}\,b = \mathrm{pur}\,y\con{1}\tildel{(\tildel{\trgl{b}}\tbinom{0}{(\tildel{\trgl{y}}\fap 0)})} = \trgl{y}.
\]
次に,評価関数$\gamma$の独立変項は,手続上包括的に評価される$\zeta_1$の部分クラス,または,同じく$\zeta_2$の部分クラスである\footnote{
    修正条件$ \orp{e,d} $の単純な汎用性は,それが促進条件か抑制条件によって,$\gamma\img\mathcal{P}(\zeta_1)$または$\gamma\img\mathcal{P}(\zeta_2)$で測ることができる.一見すると,$\orp{e,d}$が修正条件となる他の規制類型のクラスも関係するように思われるが,他の修正条件を惹起するということは,その修正条件を持つ規制類型と類似する規制類型に対して修正条件になるということを意味するから,当該クラスを独立に考慮する必要はない.
}.つまり,
\[
    (w)(w\in\breve{\gamma}\img\univ\case{1}{1}{2}w\subseteq\zeta_1\case{2}{1}{1}w\subseteq\zeta_2).
\]
さらにここでの評価関数に固有の要素として,$ \gamma $の独立変項は,機能的に同一の生得的な促進条件のクラスであるか,機能的に同一の生得的な抑制条件のクラスである.生得的修正条件は,通常,他の非生得的な修正条件の危険惹起を媒介して危険惹起される.他方,$\gamma$の値は,当該規格で許容される媒介関数(\fndp{32}{32})が独立変項のメンバーに与える最大値と考えられる.そして,
\[
    (\gamma\img\mathcal{P}(\zeta_1)\text{ のメンバーの総和}) - (\gamma\img\mathcal{P}(\zeta_2)\text{ のメンバーの総和})
\]
の値が小さい程,当該抑制条件の重要度(要するに有害度)は高い(評価対象が促進条件の場合は逆である).
こうして,現在または未来の抑制条件に関する価値判断は,評価構造の記述に置き換えられる.結局,実際に起きていることは,様々な抑制条件の実現とそれの因果的関係でしかない.後は評価構造によるそれらの分類の仕方があるだけだ.それを超えて状況を価値的に表現することは,端的に誤った情報処理なのだ.

\subsubsection{過去に囚われること}
\label{sssec:過去に囚われること}

\ref{sssec:未来を心配する}とは逆に,現在の状況が過去に実現された抑制条件に依存しているパターンもある.現在の状況はそれ自体として抑制条件であるが,それを基礎づける制御可能性が過去の抑制条件に依存するわけではない.単に過去の抑制条件の実現が,現在の他の抑制条件の実現を因果的に決定しているということだ.
例えば,家族と死別するとか恋人と別れる等の喪失によって,それがなければ現在実現していたであろう促進条件が阻止されている.そして,促進条件の阻止が抑制条件となって落ち込みや憂鬱等の情動を生じさせるのだ.また,同じ状況が将来も継続すること(現在の状況によって今後も類似の状況が生じるという因果関係)を肯定する言語的傾向性によって,\ref{sssec:未来を心配する}のパターンにより,不安などが抱き合わせで生じることもよくある.

いずれにしても,過去・現在・未来の状況との間の因果関係の記述を最適化することが必要だ.しかし,\ref{sssec:未来を心配する}のケースと同様に,このケースにおいて最も致命的なのは,過去の喪失やそれに依存する現在の状況に対して,「もううんざりだ」,「もうやってられない」,「辛すぎる」などと心の内外で叫ぶことだ.この部分の最適化すなわち漠然とした価値的判断を評価構造に置き換えることが,結局最も重要になる.事実起きていることは,評価構造が分類するような仕方で抑制条件の因果的ネットワークが周囲に渦巻いているということ,単にそれだけのことなのだ.そこに価値的判断を上乗せすることは情報処理を混乱させる以外の意味を持たない.つまり全く無意味ということだ.

なお関連する事例として,過去の出来事によって生じるジレンマ状況について考察しよう.S類型$ y,y' $について,$y$の制御構造の実現によって$y'$の構成要件の実現が因果的に決定されているケースだ.つまり,
\setcounter{equation}{0}
\begin{gather}
    \orp{x,y},\orp{x',y'}\in\mathrm{REG}^\epsilon\con{1}x\in\cs{\epsilon}y\con{1}x'\in \exe{\epsilon}y',\\
    \orp{x,\app{\epsilon}y\cap\cs{\epsilon}y}\to_{\epsilon}\orp{x',\exe{\epsilon}y'}.
\end{gather}
典型的には,
\[
    (\exists z)(\exists w)[
        \orp{\orp{z,w},\orp{x,y}}\in\mathrm{IS}\con{1}
        \tildel{(x')}\uphr\bar{1}=\tilde{z}\uphr\bar{1}\con{1}\tildel{\trgl{(y')}}\uphr\bar{1}=\tildel{\trgl{w}}\uphr\bar{1}
    ]
\]
であり,かつ,$ x\in \exe{\epsilon}y $であることが作為行動であるケース.すると上記の遮断関係(\fndp{27}{27})により,$ x'\in \exe{\epsilon}y' $であることはそれに対応する不作為行動になる.
(1)(2)を充たす$ \orp{x,y},\orp{x',y'} $が仮に同一の法体系に含まれる法規制であるなら,一方が他方の機能を阻害しているから工学的妥当性に問題を生じるだろう.しかし,個人レベルの規制ではジレンマは頻発する.例えば,別れた恋人に連絡すれば復縁できるだろう状況において,あえて連絡しない不作為により分離状態が維持されているケース.この分離状態それ自体は抑制条件であり,それによって悲しみ等の情動も生じる.しかし,既に別の恋人がいるとかその他の事情を適用条件として,別れた恋人に連絡すると別のトラブルが発生する執行可能性が成り立つ.その結果,連絡する作為が遮断される.

このようなジレンマ状況においては自ら抑制条件を維持していると言えるが,それは抑制条件を阻止する行動が別の制御構造により遮断される結果にすぎない.当該抑制条件が抑制的であることに変わりはない.すなわち,
$ x'\notin\cs{\epsilon}y' $であるが,執行随伴性$ x'\in\exe{\epsilon}y'\cap\enf{\epsilon}y' $によって,ある$ x'' $について$ \cs{\epsilon}y' $の蓋然性が実現する限り,$ x'\in\cty{\epsilon}y' $.もちろん$ x' = x'' $でもよい.それゆえ,最適化前の信念体系の状態によっては,状況全体に対する悲観的な価値判断等によって,不必要に強い情動が生じ得る.メソッドによる最適化の必要性と可能性については,ジレンマ状況でもそれ以外の状況でも同じことだ.

\subsubsection{自己非難と他者非難}
\label{sssec:自己非難と他者非難}

\ref{sssec:未来を心配する}と\ref{sssec:過去に囚われること}で処理されたのは,特に限定のない何らかの状況が抑制条件となって情動を生成するケースだった.他方,ここで扱われるのは,他人や自分の行動またはその結果が抑制条件となって情動を生成するケースである.
他人や自分の行動それ自体は物理的な因果的構造だ.しかし,それに対する規範的判断が同期することによって,他人の行動の場合は「怒り」と呼ばれる情動が,自分の行動の場合は「落ち込み」とか呼ばれる情動が,(不適応な結果を招く程に)大きく増幅させられる.
規範的判断は主として,行動記述$p$に内包的演算子「〜であるべき」や「〜であってはならない」等を結合して作られる規範文を肯定する傾向性だ.$p$が行動記述でない拡張的な規範的判断も存在する.「世界はこうあるべきでない」,「世の中不公平だ」等々と類似のセリフがそれだ.規範的判断のクラスは最適化前の信念体系に深く食い入り我々を蝕んでいる.このような規範的判断のクラスを最適化することは,規制の概念を使用して規範的システムを再記述することを意味する(\ref{ssec:因果と規範}).言い換えれば,規範的判断を,規制の構成要件該当性と執行可能性及び制御可能性の記述に置き換えることだ.

この点について,既存の規範的判断の理論的代替物は,主として,倫理的または道徳的と分類される規制(倫理規制)の記述である.倫理規制は以下のような特徴を持つ.
まず,倫理規制の修正条件は,その構成要件が帰属される主体が属するコミュニティのメンバーによる非難/賞賛行動の結果,またはその集積と考えられる.
すなわち,コミュニティ$\kappa$の倫理的なS規制$ \orp{x,y} $について,
\begin{align*}
    \tilde{x}\fap 0 \subseteq\classab{\orp{\tilde{z}\fap 2,\tildel{\trgl{w}}\fap 2}:
        \tilde{z}\fap 1\in\kappa\con{1}z\in\app{\epsilon}w\con{1}w\text{ は非難行動のP類型}
    }\con{1}\tildel{\trgl{y}}\fap 0 \subseteq\mathcal{P}(\mathfrak{E}).
\end{align*}
他方,倫理的なP規制の場合は,上記の「非難」を「賞賛」に変更する.
次に,倫理規制では認定手続と執行手続は未分化であり,それらを合体させたプロセスが各実行規制(\fndp{35}{35})の結節点の内部で進行する.
この点,$\orp{x,y}$の修正条件の実行規制は,
\[
   v \subseteq \classab{\orp{z,w}:\orp{\tilde{z}\fap 2,\tildel{\trgl{w}}\fap 2}\in \tilde{x}\fap 0\con{1}
   \tilde{z}\fap 1\in\kappa\con{1}z\in\app{\epsilon}w\con{1}w\text{ は非難行動のP類型}}
\]
なる$ v $のメンバーであり\footnote{
    $v$のメンバーは,既に述べた非難または賞賛する行動のP規制であるが,その適用条件自体は希薄である.
},$ v\subseteq \classab{\orp{z,w}:z\in\exe{\epsilon}w} $であることによって,$ \orp{x,y} $の修正条件が直接実現される.
そして,倫理規制の構成要件実現に対して,それを非難/賞賛する人($z\in v\img\univ$の結節点$\tilde{z}\fap 1$)の内部で,構成要件該当性$x\in \exe{\epsilon}y$や制御可能性$ x\in\cty{\epsilon}y $の認定が行われる.そのプロセス(倫理手続)がなければ実行規制の執行可能性(例えば,違反者を攻撃→自己満足)は生じない.

また,倫理手続には正規性の認定も含まれ,執行可能性$ x\in\enf{\epsilon}y $の認定はこれで代替される.倫理規制の正規集合は単純であり制定階層は普通1段階しかない.すなわち,倫理規制は,その執行可能性がある評価構造による工学的妥当性を充たすという事実によって制定される.そして,その執行可能性が同じ評価構造による工学的妥当性を充たさないという事実によって廃止される.制定規制も廃止規制もN規制であり,この2種類のN規制を始祖として,あるコミュニティの倫理的な正規集合または規制体系が作られる.なお,ここでの評価構造は,既に述べたコミュニティ$\kappa$をユーザー集合とする.この評価構造による工学的妥当性が実際に成立していることによって,(証拠が生成され)倫理手続でそれを認定する制御構造が構築される.その結果,倫理規制の執行可能性が実現される\footnote{倫理学は評価構造を開発してそれを宣伝することによって,このプロセスに介入しようとする.}.
この点,他者や自己を義務や禁止に違反することで非難する場合,そのことを「正当化」する価値判断が同時に存在することが多い.倫理規制の正規性つまり評価構造による工学的妥当性の記述は,この規範的正当化の理論的代替物である.

さて,規範的判断に関連する信念体系の最適化によって気付かされることの1つは,我々が倫理規制の執行機関として自分や他人を非難する場合,その殆どの場合に正規性の認定を誤っているということだ.つまり,恣意的な評価構造を用いる場合は別として,殆どの場合に非難の執行可能性は工学的妥当性を持たない.誰かを咎める意味はないのだ.
我々は既存の意味での規範と価値を喪失する.規範は物理的システムの中に消え去り,また,あらゆるものが(その値がゼロであれ何であれ)等しい価値を持つようになる.それゆえ,賞賛に値するものも非難に値するものもない.